% !Mode:: "TeX:UTF-8"
%%%%%%%%%%%%%%%%%%%%%%%%%%%%%%%%%%%%%%%%%%%%%%
%                      Notations File Made by Gau, Syu                       %
%                                version = 2017.314                                 %
%                        contact = GauSyu@Gmail.com                        %
%%%%%%%%%%%%%%%%%%%%%%%%%%%%%%%%%%%%%%%%%%%%%%
%
%	\ensuremath{---}\xspace may be helpful
%
% @ common Operators
\DeclareMathOperator{\id}{id}			%	identity
\DeclareMathOperator{\Id}{Id}			%	Identity
\DeclareMathOperator{\obj}{obj}			%	identity
\DeclareMathOperator{\pr}{pr}			%	projection
\DeclareMathOperator{\pt}{pt}			%	point
\DeclareMathOperator{\res}{res}			%	restriction
%
% @ Operators in a category
\DeclareMathOperator{\coim}{coim}
\DeclareMathOperator{\Coim}{Coim}
\DeclareMathOperator{\coker}{coker}
\DeclareMathOperator{\Coker}{Coker}
\DeclareMathOperator{\im}{im}
\DeclareMathOperator{\Image}{Im}
\DeclareMathOperator{\Ker}{Ker}
\DeclareMathOperator{\Sat}{Sat}			%	Saturated closure
\DeclareMathOperator{\sk}{sk}
\DeclareMathOperator{\ob}{ob}%Object
%
% @ Operators in linear algebra
\DeclareMathOperator{\Ad}{Ad}			%	Adjunction
\DeclareMathOperator{\ad}{ad}			%	adjuntion
\DeclareMathOperator{\diag}{diag}
\DeclareMathOperator{\Det}{Det}			%	Determinant
\DeclareMathOperator{\Exp}{Exp}			%	Expotential
\DeclareMathOperator{\Hess}{Hess}		%	Hessian
\DeclareMathOperator{\Span}{span} 		%	Annoyingly, \span is already a command in TeX, and redefining it leads to other problems
\DeclareMathOperator{\Tr}{Tr}				%	trace
\DeclareMathOperator{\tr}{tr}				%	trace
\DeclareMathOperator{\wt}{wt}				%	trace
%
% @ Operators in sub(p)scripts
\DeclareMathOperator{\ab}{ab}		%	abelian
\DeclareMathOperator{\an}{an}		%	analytic
\DeclareMathOperator{\conv}{conv}%	convergent
\DeclareMathOperator{\ev}{ev}		%	evaluation
\DeclareMathOperator{\opp}{op}	%	opposite
\DeclareMathOperator{\pre}{pre}
\DeclareMathOperator{\rev}{rev}		%	reverse
\DeclareMathOperator{\sep}{sep}		%	separated
\DeclareMathOperator{\tor}{tor}		%	torsion
%
% @ Operators in commutative algebra
\DeclareMathOperator{\Ann}{Ann}			%	Annihilator
\DeclareMathOperator{\Char}{Char}		%	Characteristic
\DeclareMathOperator{\gr}{gr}			%	associated graded object
\DeclareMathOperator{\Gr}{Gr}			%	associated graded object
\DeclareMathOperator{\SGr}{\mathscr{G}\!\mathit{r}}%	associated graded object
\DeclareMathOperator{\lcm}{lcm}			%	least common multiple
\DeclareMathOperator{\ord}{ord}			%	order
\DeclareMathOperator{\rad}{rad}			%	radical
\DeclareMathOperator{\Rad}{Rad}			%	Radical
\DeclareMathOperator{\rank}{rank}		%	rank
\DeclareMathOperator{\mult}{mult}		%	multiplicity
\DeclareMathOperator{\Length}{Length}		%	length
\DeclareMathOperator{\sgn}{sgn}			%	signature
\DeclareMathOperator{\supp}{supp}	%	support
\DeclareMathOperator{\Supp}{Supp}	%	Support
\DeclareMathOperator{\vol}{vol}				%	volume
\DeclareMathOperator{\Vol}{Vol}				%	Volume
%
% @ Operators of constructions
\DeclareMathOperator{\Alt}{Alt}				%	Alternative
\DeclareMathOperator{\Sym}{Sym}			%	Symmetric
\DeclareMathOperator{\Ext}{Ext}				%	Extensions
\DeclareMathOperator*{\EExt}{\mathcal{E}xt}	%	Cal Ext
\DeclareMathOperator{\Proj}{Proj}			%	Proj Spectrum
\DeclareMathOperator{\Quot}{Qout}		%	Quotient
\DeclareMathOperator{\Spec}{Spec}		%	Spectrum
\DeclareMathOperator{\Spm}{Spm}		%	Maximal Spectrum
\DeclareMathOperator{\Tor}{Tor}				%	Torsors
%
\DeclareMathOperator{\Der}{Der}			%	Derivations
\DeclareMathOperator{\SDer}{\mathscr{D}\!\mathit{er}}			%	Sheaf Derivations
\DeclareMathOperator{\Diff}{Diff}			%	Differential operators
\DeclareMathOperator{\SDiff}{\mathscr{D}\!\mathit{iff}}			%	Differential operators
\DeclareMathOperator{\Gal}{Gal}			%	Galois group
\DeclareMathOperator{\Pic}{Pic}			%	Picard group
\DeclareMathOperator{\Res}{Res}			%	Residue
%
% @ Hom-like Operators
\DeclareMathOperator{\Aut}{Aut}		%	automorphisms
\DeclareMathOperator{\Bil}{Bil}				%	Bilinear maps
\DeclareMathOperator{\End}{End}		%	Endomorphisms
\DeclareMathOperator{\Fun}{Fun}			%	Functors
\DeclareMathOperator{\hFun}{hFun}			%	Functors
\DeclareMathOperator{\FFun}{\mathcal{F}un}			%	Functors
\DeclareMathOperator{\GrHom}{GrHom}	%	Graded Hom
\DeclareMathOperator{\Hom}{Hom}			%	Hom
\DeclareMathOperator{\inEnd}{\underline{End}}			%	Internal End
\DeclareMathOperator{\inHom}{\underline{Hom}}			%	Internal Hom
\DeclareMathOperator{\inGrHom}{\underline{GrHom}}	%	Internal Graded Hom
\DeclareMathOperator{\Nat}{Nat}				%	Nature transformations
\DeclareMathOperator{\Mor}{Mor}		%	Morphisms
\DeclareMathOperator{\Map}{Map}		%	Maps
\DeclareMathOperator{\HHom}{\mathcal{H}\!\mathit{om}}			%		Hom space
\DeclareMathOperator{\SHom}{\mathscr{H}\!\!\mathit{om}}			%	sheaf Hom
\DeclareMathOperator{\SEnd}{\mathscr{E}\!\mathit{nd}}			%	sheaf Hom
\def\SheafHom{\SHom}
\DeclareMathOperator{\cont}{\mathit{cont}}
%
% @ Limit Operators
\DeclareMathOperator*{\holim}{\mathrm{holim}}
\DeclareMathOperator*{\hocolim}{\mathrm{hocolim}}		
\DeclareMathOperator*{\dirlim}{\underrightarrow{\lim}}			%	direct limit
\DeclareMathOperator*{\colim}{\mathrm{colim}}			%	direct limit
\DeclareMathOperator*{\invlim}{\underleftarrow{\lim}}			%	inverse limit
\DeclareMathOperator*{\Rinvlim}{\underleftarrow{\lim}^1}	%	derived inverse limit
\DeclareMathOperator*{\Lan}{Lan}				%	Left Kan extension
\DeclareMathOperator*{\Ran}{Ran}				%	Right Kan extension
%
%
% @ Operators in number theory
\DeclareMathOperator{\Frob}{Frob}		%	Frobenuis
\DeclareMathOperator*{\sump}{\sum\nolimits^\prime}			%	sum outside origin
\renewcommand{\mod}{\mathop{\mathrm{mod}}}
\makeatletter
\renewcommand{\pod}[1]{\allowbreak\mathchoice
	{\if@display \mkern 18mu\else \mkern 8mu\fi (#1)}
	{\if@display \mkern 18mu\else \mkern 8mu\fi (#1)}
	{\mkern4mu(#1)}
	{\mkern4mu(#1)}
}
\makeatother