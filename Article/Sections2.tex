% !Mode:: "TeX:UTF-8"
%%%%%%%%%%%%%%%%%%%%%%%%%%%%%%%%%%%%%%%%%%%%%%
%                 Section Setting File Made by Gau, Syu                  %
%                                version = 2017.314                                 %
%                        contact = GauSyu@Gmail.com                        %
%%%%%%%%%%%%%%%%%%%%%%%%%%%%%%%%%%%%%%%%%%%%%%
%
% @ Section name
\usepackage{titlesec}
\titleformat{\section}
{\normalfont\Large\bfseries}
{\S\ \thesection}
{1em}
{}
%	\titleformat{<command>}[<shape>]{<format>}{<label>}{<sep>}{<before-code>}[<after-code>]
%%	<command>: the sectioning command to be redefined
%%	<shape>: (optional) paragraph shape
%%	<format>: the format to be applied to the whole title
%%	<label>: the label with numbering
%%	<sep>: the horizontal separation between label and title body
%%	<before-code>: code preceding the title body
%%	<after-code>: (optional) code following the title body
%
% @ Section Numbering
%\renewcommand{\thesubsection}{\thesection.\roman{subsection}}	%	abbrev and romanize the counter labels
%	The following code remove the whole label part of subsection if it doesn't have number, otherwise there will be an empty label at head
\makeatletter
\def\@subseccntformat#1{\csname #1ignore\expandafter\endcsname\csname the#1\endcsname\quad}
\let\subsectionignore\@gobbletwo
\let\latex@numberline\numberline
\def\numberline#1{\if\relax#1\relax\else\latex@numberline{#1}\fi}
\makeatother
%
% @ Set the depths
\setcounter{secnumdepth}{2}			%	depth of section numbering levels
%	0=chapter, 1=section, 2=subsection
\setcounter{tocdepth}{2}				% 	depth of contents