% !Mode:: "TeX:UTF-8"
%%%%%%%%%%%%%%%%%%%%%%%%%%%%%%%%%%%%%%%%%%%%%%
%               Theorems Setting File Made by Gau, Syu                %
%                                version = 2017.314                                 %
%                        contact = GauSyu@Gmail.com                        %
%%%%%%%%%%%%%%%%%%%%%%%%%%%%%%%%%%%%%%%%%%%%%%
%
% @	Known keys to \declaretheorem
%%	[parent=<counter>] (alias: numberwithin, within) : reset whenever that counter is incremented
%%	[sibling=<counter>] (alias: numberlike, sharenumber) : use this counter for numbering
%%	[title=<TEX>] (alias: name, heading) : title of the theorem, default is the name of the environment
%%	[numbered=<yes/no>] : will be numbered or not
%%	[style=<theoremstyle>] : use the settings of this style
%%	[refname=<string>] : the name of the theorem as used by \autoref, \cref and friends
%%	[refname=<string1,string2>] : the second is the plural form used by \cref
%%	[Refname=<string>] : the name of the theorem as used by \Autoref, \Cref and friends
%%	[Refname=<string1,string2>] : the second is the plural form used by \Cref
%
% @ Known keys to \declaretheoremstyle
%%	[spaceabove=<length>]
%%	[spacebelow=<length>]
%%	[headfont=<font switches>]
%%	[bodyfont=<font switches>]
%%	[headpunct=<TEX>] : theoremseparator
%%	[postheadspace=<length>] : horizontal space inserted after the entire head of the theorem, before the body
%%	[headindent=<length>] : horizontal space inserted before the head
%%	[headformat=<TEX> or <keywords>] : see manuel
%%	[headformat=margin] : number protude in the (left) margin
%%	[headformat=swapnumber] : puts number before the name
%
% @ Thm-type
\theoremstyle{margin}	%	Put the number at left margin
\numberwithin{equation}{subsection}	%	reset eq. number for each section
\Crefname{equation}{}{}
%
\declaretheorem[%
title=Theorem,%
numberlike=equation,%
refname={theorem,theorems},%
Refname={Theorem,Theorems}%
]{theorem}
%	theorem
%
\declaretheorem[%
title=Theorem,%
numbered=no%
]{Thm}
%	Thm: Theorem
%
\declaretheorem[%
title=Proposition,%
numberlike=equation,%
refname={proposition,propositions},%
Refname={Proposition,Propositions}%
]{proposition}
%	proposition
%
\declaretheorem[%
title=Proposition,%
numberlike=equation,%
refname={proposition,propositions},%
Refname={Proposition,Propositions}%
]{prop}
%	prop: Proposition
%
\declaretheorem[
title=Proposition,%
within=equation,%
refname={proposition,propositions},%
Refname={Proposition,Propositions}%
]{sprop}
%	sprop: (Sub)Proposition
%         
\declaretheorem[%
title=Proposition,%
numbered=no,%
refname={proposition,propositions},%
Refname={Proposition,Propositions}%
]{Prop}
%	prop: Proposition
%   
\declaretheorem[%
title=Lemma,%
numberlike=equation,%
refname={lemma,lemmas},%
Refname={Lemma,Lemmas}%
]{lemma}
%	lemma
%
\declaretheorem[%
title=Lemma,%
numberlike=equation,%
refname={lemma,lemmas},%
Refname={Lemma,Lemmas}%
]{lem}
%	lem: Lemma
%
\declaretheorem[%
title=Lemma,%
numberlike=sprop,%
refname={lemma,lemmas},%
Refname={Lemma,Lemmas}%
]{slem}
%	slem: (Sub)Lemma
%
\declaretheorem[%
title=Lemma,%
numbered=no,%
refname={lemma,lemmas},%
Refname={Lemma,Lemmas}%
]{Lem}
%	lem: Lemma
%
\declaretheorem[%
title=Corollary,%
numberlike=equation,%
refname={corollary,corollaries},%
Refname={Corollary,Corollaries}%
]{corollary}
%	corollary
%
\declaretheorem[%
title=Corollary,%
numberlike=equation,%
refname={corollary,corollaries},%
Refname={Corollary,Corollaries}%
]{cor}
%	cor: Corollary
%
\declaretheorem[%
title=Corollary,%
numberlike=sprop,%
refname={corollary,corollaries},%
Refname={Corollary,Corollaries}%
]{scor}
%	scor: (Sub)Corollary
%
% @ Defn-type
\theorembodyfont{\upshape}	%	following envs will use upshape bodyfont
%
\declaretheorem[%
title={},%
numberlike=equation%
]{para}
%
\declaretheorem[%
title={},%
numberlike=sprop%
]{subpara}
%	definition
%
\declaretheorem[%
title={Definition},%
numberlike=equation,%
refname={definition,definitions},%
Refname={Definition,Definitions}%
]{defn}
%	defn: Definition
%
\declaretheorem[%
title={Definition},%
numbered=no
]{Defn}
%	defn: Definition (nonum)
%
\declaretheorem[%
title=Example,%
numberlike=equation,%
refname={example,examples},%
Refname={Example,Examples}%
]{example}
%	example
%
\declaretheorem[%
title=Example,%
numberlike=equation,%
refname={example,examples},%
Refname={Example,Examples}%
]{eg}
%	eg: Example
%
\declaretheorem[%
title=Example,%
numberlike=sprop,%
refname={example,examples},%
Refname={Example,Examples}%
]{seg}
%	seg: (Sub)Example
%
\declaretheorem[%
title=Example,%
numbered=no%
]{Eg}
%	Eg: (unnumbered) Example
%
\declaretheorem[%
title=Remark,%
numberlike=theorem,%
refname={remark,remarks},%
Refname={Remark,Remarks}%
]{remark}
%	remark
%
\declaretheorem[%
title=Remark,%
numberlike=equation,%
refname={remark,remarks},%
Refname={Remark,Remarks}%
]{rem}
%	rem: Remark
%
\declaretheorem[%
title=Remark,%
numberlike=sprop,%
refname={remark,remarks},%
Refname={Remark,Remarks}%
]{srem}
%	srem: (Sub)Remark
%
\declaretheorem[%
title=Remark,%
numbered=no%
]{Rem}
%	Rem: (unnumbered) Remark
%
\declaretheorem[%
title=Recall,%
numbered=no%
]{rec}
%	rec: Recall
%
% @ Proof Env
\def\qedsymbol{\ensuremath{\Box}}	%	desige the qed symbol
%
\declaretheoremstyle[%
headfont=\sc,%
bodyfont=\normalfont,%
headpunct={:},%
qed=\qedsymbol%
]{nonumberproof}
%
\declaretheorem[%
title=Proof,%
numbered=no,%
style=proof%
]{proof}
%	proof




%%
%@@@THMitems
%%
\newlist{proplist}{enumerate}{1}
\setlist[proplist]{label=(\roman{proplisti}), ref=\theprop.(\roman{proplisti}),noitemsep}
\newcounter{listprop}
\Crefname{prop}{Proposition}{Propositions}
\Crefname{listprop}{Proposition}{Propositions}
\addtotheorempostheadhook[prop]{\crefalias{proplisti}{prop}}