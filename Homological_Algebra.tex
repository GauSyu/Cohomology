%!Mode:: "TeX:UTF-8"
%	Commone Head File
% !Mode:: "TeX:UTF-8"
%%%%%%%%%%%%%%%%%%%%%%%%%%%%%%%%%%%%%%%%%%%%%%
%              A Common Setting File Made by Gau, Syu               %
%                                version = 2018.April                     %
%                        contact = GauSyu@Gmail.com                        %
%%%%%%%%%%%%%%%%%%%%%%%%%%%%%%%%%%%%%%%%%%%%%%
%
% @ PRELOAD
%\RequirePackage{luatex85}	
%%	"luatex85" is needed for xy-envs if the typesetting engine is lualatex
%
% @ Document Class Settings
\documentclass[UTF-8,11pt,a4paper]{article}
%%		UTF-8: use UTF-8 encode
%%		11pt: main font size, 10pt is dedault
%%		a4paper: paper size
%
% @ Layout Settings
\usepackage{marginnote}
\usepackage{fancyhdr}		%	header and footer
\usepackage{indentfirst}	%	indent at each first line
\usepackage{pdfpages}		%	input pages from given pdf files.
% !Mode:: "TeX:UTF-8"
%%%%%%%%%%%%%%%%%%%%%%%%%%%%%%%%%%%%%%%%%%%%%%
%                Decorate Setting File Made by Gau, Syu                 %
%                                version = 2017.314                                 %
%                        contact = GauSyu@Gmail.com                        %
%%%%%%%%%%%%%%%%%%%%%%%%%%%%%%%%%%%%%%%%%%%%%%
%
% @ Colors
\usepackage{xcolor}		%	A packagefor colors
\newcommand{\black}{\color{black}}
\newcommand{\gray}{\color{gray}}
\newcommand{\red}{\color{red}}
\newcommand{\blue}{\color{blue}}
\newcommand{\green}{\color{green}}
\newcommand{\bfred}[1]{\textbf{\red #1}}
\newcommand{\bfblue}[1]{\textbf{\blue #1}}
\newcommand{\bired}[1]{\textbf{\it \red #1}}
\newcommand{\biblue}[1]{\textbf{\it \blue #1}}
\newcommand{\itred}[1]{\textit{\red #1}}
\newcommand{\itblue}[1]{\textit{\blue #1}}
%
% @ Typefaces
\usepackage{fontspec}	%	Provides an automatic and unified interface for loading fonts in LaTeX
%	Note that one may need the corresponding fonts in system to enable settings like the followings:
\newfontfamily{\Edward}{Edwardian Script ITC}
%\newfontfamily{\Frak}{Euclid Fraktur}
\newcommand{\Giant}{\fontsize{72pt}{\baselineskip}\selectfont}
\newcommand{\giant}{\fontsize{27pt}{\baselineskip}\selectfont}
%
% @ MultiColumn
\usepackage{multicol}	%	defines a multicols environment which typesets text in multiple columns
%
% @ Enumerate environment
\usepackage{enumerate}
%	This package gives the enumerate environment an optional argument which determines the style in which the counter is printed.
\usepackage[shortlabels]{enumitem}
\newlist{steps}{enumerate}{3}
\setlist[steps,1]{label=\emph{Step \arabic*.}}
\setlist[steps]{align=left, leftmargin=0pt, listparindent=\parindent, labelwidth=0pt, itemindent=!}
\setlist[steps,2]{label=\emph{Step \arabic{stepsi}.\arabic*.}}
\setlist[steps,3]{label=\emph{\alph*.}}	%	colors, typefaces etc.
% @ Latex Graphics
% !Mode:: "TeX:UTF-8"
%%%%%%%%%%%%%%%%%%%%%%%%%%%%%%%%%%%%%%%%%%%%%%
%           Latex Graphics Setting File Made by Gau, Syu            %
%                                version = 2017.314                                 %
%                        contact = GauSyu@Gmail.com                        %
%%%%%%%%%%%%%%%%%%%%%%%%%%%%%%%%%%%%%%%%%%%%%%
\usepackage{graphicx}	%	Enable to Insert Pictures
\usepackage{epic}			%	Extending Latex Graphics
%\usepackage{xy}   			% Enable to Draw Xy-Diagrams
%%	Start from 2017, I decive to use tikzcd instead of xy-enivs
\usepackage{tikz}			%	Enable to Draw Pictures 
\usetikzlibrary{cd}				%	Load the tikzcd library
\usetikzlibrary{arrows}		%	Load the arrows
\usetikzlibrary{tikzmark,quotes}	%	Enable subnode
\usepackage{braids}			%	An easy way to draw braids, need tikz
\usepackage{adjustbox}
%
% @ Indexes, Glossaries and Refs
% !Mode:: "TeX:UTF-8"
%%%%%%%%%%%%%%%%%%%%%%%%%%%%%%%%%%%%%%%%%%%%%%
%                   Index Setting File Made by Gau, Syu                   %
%                                version = 2017.314                                 %
%                        contact = GauSyu@Gmail.com                        %
%%%%%%%%%%%%%%%%%%%%%%%%%%%%%%%%%%%%%%%%%%%%%%
%
% @ MakeIndex
\makeindex
%
% @ Indexed Terminology
\usepackage{xifthen}			%	provides \isempty test
\newcommand{\termin}[2][]{%
\ifthenelse{\isempty{#1}}%
{\emph{\textbf{{#2}}}\index{#2}}% if #1 is empty
{\emph{\textbf{{#1}}}\index{#2}}% if #1 is not empty
}	%	Used as: \termin[arg1]{arg2}
%%		arg1 is optional, it is what this command display, default is same as arg2
%%		arg2 is what this command generates in the Index table
\newcommand{\bfredin}[2][]{%
\ifthenelse{\isempty{#1}}%
{\bfred{{#2}}\index{#2}}% if #1 is empty
{\bfred{{#1}}\index{#2}}% if #1 is not empty
}	%	Same as above, except the layout style
\newcommand{\bfbluein}[2][]{%
\ifthenelse{\isempty{#1}}%
{\bfblue{{#2}}\index{#2}}% if #1 is empty
{\bfblue{{#1}}\index{#2}}% if #1 is not empty
}	%	Same as above, except the layout style	%	index settings
%% !Mode:: "TeX:UTF-8"
%%%%%%%%%%%%%%%%%%%%%%%%%%%%%%%%%%%%%%%%%%%%%%
%               Glossaries Setting File Made by Gau, Syu               %
%                                version = 2017.314                                 %
%                        contact = GauSyu@Gmail.com                        %
%%%%%%%%%%%%%%%%%%%%%%%%%%%%%%%%%%%%%%%%%%%%%%
%
% @ MakeGlossaries
    \usepackage[
               symbols,                %list of symbols
              %nonumberlist,           %do not show page numbers
               seeautonumberlist,      %
               hyperfirst=false,       %
               toc,                    %show listings as entries in table of contents
               section=chapter,        %use section level for toc entries
               counter=section         %countered by section
               ]{glossaries}
    % & Add new glossary listing
        \altnewglossary{categories}{cat}{Categories}
    % & Enable hyperref on gls
        \glsenablehyper
    % & Make glossaries
        \makeglossaries

    % & <Example> Usage of glossaryentry
    %   \newglossaryentry{<\label>}
    %                    {name=<\what occurs in the glossary>,
    %                     description=<\>,
    %                     text=<\what occurs in the context>,
    %                     sort=<\How this term by sorted>,
    %                     type=<\>}

	%	glossary settings
% !Mode:: "TeX:UTF-8"
%%%%%%%%%%%%%%%%%%%%%%%%%%%%%%%%%%%%%%%%%%%%%%
%                 BibTeX Setting File Made by Gau, Syu                %
%                                version = 2017.314                                 %
%                        contact = GauSyu@Gmail.com                        %
%%%%%%%%%%%%%%%%%%%%%%%%%%%%%%%%%%%%%%%%%%%%%%
%
\usepackage[sorting=none,giveninits=true]{biblatex}
%% !Mode:: "TeX:UTF-8"
%%%%%%%%%%%%%%%%%%%%%%%%%%%%%%%%%%%%%%%%%%%%%%
%                 AMSRefs Setting File Made by Gau, Syu                %
%                                version = 2017.314                                 %
%                        contact = GauSyu@Gmail.com                        %
%%%%%%%%%%%%%%%%%%%%%%%%%%%%%%%%%%%%%%%%%%%%%%
%
% @ AMSRefs
\usepackage[alphabetic,abbrev]{amsrefs}	%	A Refs Package for AMS
%%	citation-order: ordered as the citation-order
%%	alphabetic: generates alphabetic labels
%%	y2k: use the full year instead of the last two digits of the year
%%	shortalphabetic: generates a shorter alphabetic label using only the first letter of each author name
%%	abbrev: abbrev: equivalent to requesting all four of the following options
%%		initials: replace the given names of all authors, editors, and translators with their initials
%%		short-journals: print short form instead of full form for journal names
%%		short-months: print short version of month names (e.g., Jan. instead of January)
%%		short-publishers: print short form instead of full form for publisher names
%%	backrefs: causes “back-references” to be printed at the end of each bibliography entry to show what page it was cited on
\newcommand{\CITE}[1]{%
\hyperlink{#1}{{\upshape [\textbf{\red{#1}}]}}
}
	% AMSRef settings, NEED manual modifications
%
% @ Chapters and Sections
% !Mode:: "TeX:UTF-8"
%%%%%%%%%%%%%%%%%%%%%%%%%%%%%%%%%%%%%%%%%%%%%%
%                 Section Setting File Made by Gau, Syu                  %
%                                version = 2017.314                                 %
%                        contact = GauSyu@Gmail.com                        %
%%%%%%%%%%%%%%%%%%%%%%%%%%%%%%%%%%%%%%%%%%%%%%
%
% @ Section name
\usepackage{titlesec}
\titleformat{\section}
{\normalfont\Large\bfseries}
{\S\ \thesection}
{1em}
{}
%	\titleformat{<command>}[<shape>]{<format>}{<label>}{<sep>}{<before-code>}[<after-code>]
%%	<command>: the sectioning command to be redefined
%%	<shape>: (optional) paragraph shape
%%	<format>: the format to be applied to the whole title
%%	<label>: the label with numbering
%%	<sep>: the horizontal separation between label and title body
%%	<before-code>: code preceding the title body
%%	<after-code>: (optional) code following the title body
%
% @ Section Numbering
%\renewcommand{\thesubsection}{\thesection.\roman{subsection}}	%	abbrev and romanize the counter labels
%	The following code remove the whole label part of subsection if it doesn't have number, otherwise there will be an empty label at head
\makeatletter
\def\@subseccntformat#1{\csname #1ignore\expandafter\endcsname\csname the#1\endcsname\quad}
\let\subsectionignore\@gobbletwo
\let\latex@numberline\numberline
\def\numberline#1{\if\relax#1\relax\else\latex@numberline{#1}\fi}
\makeatother
%
% @ Set the depths
\setcounter{secnumdepth}{2}			%	depth of section numbering levels
%	0=chapter, 1=section, 2=subsection
\setcounter{tocdepth}{2}				% 	depth of contents	%	section settings for articles
%
% @ Math Support
%\usepackage{stix}					%	The STIX fonts are a suite of uni­code OpenType fonts con­tain­ing a com­plete set of math­e­mat­i­cal glyphs
\usepackage[leqno]{amsmath}
%	Standard AMS Package & AMS Theorem Env
%%	leqno means put number of formulas on the left
%
\usepackage{mathtools}
%	provides a series of packages designed to enhance the appearance of documents containing a lot of mathematics
%
\usepackage{hyperref}	%	basic package for hyperrefs
% @@@@@@@@@@@@@ MUST Put hyperref AFTER amsmath and BEFORE cleveref
\usepackage[amsmath,thmmarks,hyperref]{ntheorem}
%	Pacakage for for handling theorem-like environments
%%	[thmmarks] enables the automatical placement of endmarks
%%	[thref] enables the extended reference features
%%	[hyperref] provides compability with the hyperref-package
%%	[amsthm] provides compatibility with the theorem-layout commands of the amsthm-package
%%	[standard] uses the amssymb and latexsym (automatically loaded) packages and defines the frequently used environments in english and german documents
%
\usepackage{thmtools}
%	a collection of packages that is designed to provide an easier interface to theorems, and to facilitate some more advanced tasks
%
%
%
% @ fix the bug in aliasctr.sty
\usepackage{etoolbox}
\makeatletter
\patchcmd{\@counteralias}
 {\@ifdefinable{c@#1}}
 {\expandafter\@ifdefinable\csname c@#1\endcsname}
 {}{}
\makeatother
%
%
%
% @ Math Fonts
\usepackage{amssymb}		%	defined math symbols in msam and mabm, amsfonts is automatically included
%\usepackage{eucal}				%	makes \mathcal use Euler script instead of the usual Computer Modern calligraphic alphabet
%\usepackage{mathrsfs}		%	support for using RSFS fonts (\mathscr) in maths
\usepackage[cal=euler,scr=boondox]{mathalfa}
%	make \mathcal use Euler script and \mathscr use boondoxo (from Stix with Calligraphic Oblique)
\usepackage{mathdots}		%	commands to produce dots in math that respect font size
\usepackage{upgreek}		%	upright Greek letters
\usepackage{manfnt}			%	Knuth's ‘manual’ font
\usepackage{stmaryrd}		%	St Mary Road symbols for theoretical computer science
%
% @ Abbrevated Command for Letters
% !Mode:: "TeX:UTF-8"
%%%%%%%%%%%%%%%%%%%%%%%%%%%%%%%%%%%%%%%%%%%%%%
%              Letter Abbreviation File Made by Gau, Syu              %
%                                version = 2017.314                                 %
%                        contact = GauSyu@Gmail.com                        %
%%%%%%%%%%%%%%%%%%%%%%%%%%%%%%%%%%%%%%%%%%%%%%
%
% @ \mathcal
\def\Aa{{\cal A}}
\def\Bb{{\cal B}}
\def\Cc{{\cal C}}
\def\Dd{{\cal D}}
\def\Ee{{\cal E}}
\def\Ff{{\cal F}}
\def\Gg{{\cal G}}
\def\Hh{{\cal H}}
\def\Ii{{\cal I}}
\def\Jj{{\cal J}}
\def\Kk{{\cal K}}
\def\Ll{{\cal L}}
\def\Mm{{\cal M}}
\def\Nn{{\cal N}}
\def\Oo{{\cal O}}
\def\Pp{{\cal P}}
\def\Qq{{\cal Q}}
\def\Rr{{\cal R}}
\def\Ss{{\cal S}}
\def\Tt{{\cal T}}
\def\Uu{{\cal U}}
\def\Vv{{\cal V}}
\def\Ww{{\cal W}}
\def\Xx{{\cal X}}
\def\Yy{{\cal Y}}
\def\Zz{{\cal Z}}			%	\mathcal{A} -> \Aa
% !Mode:: "TeX:UTF-8"
%%%%%%%%%%%%%%%%%%%%%%%%%%%%%%%%%%%%%%%%%%%%%%
%              Letter Abbreviation File Made by Gau, Syu              %
%                                version = 2017.314                                 %
%                        contact = GauSyu@Gmail.com                        %
%%%%%%%%%%%%%%%%%%%%%%%%%%%%%%%%%%%%%%%%%%%%%%
%
% @ \mathbb
\def\AA{{\mathbb A}}
\def\BB{{\mathbb B}}
\def\CC{{\mathbb C}}
\def\DD{{\mathbb D}}
\def\EE{{\mathbb E}}
\def\FF{{\mathbb F}}
\def\GG{{\mathbb G}}
\def\HH{{\mathbb H}}
\def\II{{\mathbb I}}
\def\JJ{{\mathbb J}}
\def\KK{{\mathbb K}}
\def\LL{{\mathbb L}}
\def\MM{{\mathbb M}}
\def\NN{{\mathbb N}}
\def\OO{{\mathbb O}}
\def\PP{{\mathbb P}}
\def\QQ{{\mathbb Q}}
\def\RR{{\mathbb R}}
\def\SS{{\mathbb S}}
\def\TT{{\mathbb T}}
\def\UU{{\mathbb U}}
\def\VV{{\mathbb V}}
\def\WW{{\mathbb W}}
\def\XX{{\mathbb X}}
\def\YY{{\mathbb Y}}
\def\ZZ{{\mathbb Z}}			%	\mathbb{A} -> \AA
% !Mode:: "TeX:UTF-8"
%%%%%%%%%%%%%%%%%%%%%%%%%%%%%%%%%%%%%%%%%%%%%%
%              Letter Abbreviation File Made by Gau, Syu              %
%                                version = 2017.314                                 %
%                        contact = GauSyu@Gmail.com                        %
%%%%%%%%%%%%%%%%%%%%%%%%%%%%%%%%%%%%%%%%%%%%%%
%
% @ \mathfrak (little)
\def\aa{\mathfrak{a}}
\def\bb{\mathfrak{b}}
\def\cc{\mathfrak{c}}
\def\dd{\mathfrak{d}}
\def\ee{\mathfrak{e}}
\def\ff{\mathfrak{f}}
\def\gg{\mathfrak{g}}  % Knuth uses $\gg$ for ``>>''.
\def\hh{\mathfrak{h}}
\def\ii{\mathfrak{i}}
\def\jj{\mathfrak{j}}
\def\kk{\mathfrak{k}}
\def\ll{\mathfrak{l}}  % Knuth uses $\ll$ for ``<<''.
\def\mm{\mathfrak{m}}
\def\nn{\mathfrak{n}}
\def\oo{\mathfrak{o}}
\def\pp{\mathfrak{p}}
\def\qq{\mathfrak{q}}
\def\rr{\mathfrak{r}}
\def\ss{\mathfrak{s}}
\def\tt{\mathfrak{t}}
\def\uu{\mathfrak{u}}
\def\vv{\mathfrak{v}}
\def\ww{\mathfrak{w}}
\def\xx{\mathfrak{x}}
\def\yy{\mathfrak{y}}
\def\zz{\mathfrak{z}}
			%	\mathfrak{a} -> \aa
% !Mode:: "TeX:UTF-8"
%%%%%%%%%%%%%%%%%%%%%%%%%%%%%%%%%%%%%%%%%%%%%%
%              Letter Abbreviation File Made by Gau, Syu              %
%                                version = 2017.314                                 %
%                        contact = GauSyu@Gmail.com                        %
%%%%%%%%%%%%%%%%%%%%%%%%%%%%%%%%%%%%%%%%%%%%%%
%
% @ \mathscr
\def\Aaa{\mathscr{A}}
\def\Bbb{\mathscr{B}}
\def\Ccc{\mathscr{C}}
\def\Ddd{\mathscr{D}}
\def\Eee{\mathscr{E}}
\def\Fff{\mathscr{F}}
\def\Ggg{\mathscr{G}}
\def\Hhh{\mathscr{H}}
\def\Iii{\mathscr{I}}
\def\Jjj{\mathscr{J}}
\def\Kkk{\mathscr{K}}
\def\Lll{\mathscr{L}}
\def\Mmm{\mathscr{M}}
\def\Nnn{\mathscr{N}}
\def\Ooo{\mathscr{O}}
\def\Ppp{\mathscr{P}}
\def\Qqq{\mathscr{Q}}
\def\Rrr{\mathscr{R}}
\def\Sss{\mathscr{S}}
\def\Ttt{\mathscr{T}}
\def\Uuu{\mathscr{U}}
\def\Vvv{\mathscr{V}}
\def\Www{\mathscr{W}}
\def\Xxx{\mathscr{X}}
\def\Yyy{\mathscr{Y}}
\def\Zzz{\mathscr{Z}}			%	\mathscr{A} -> \Aaa
%% !Mode:: "TeX:UTF-8"
%%%%%%%%%%%%%%%%%%%%%%%%%%%%%%%%%%%%%%%%%%%%%%
%              Letter Abbreviation File Made by Gau, Syu              %
%                                version = 2017.314                                 %
%                        contact = GauSyu@Gmail.com                        %
%%%%%%%%%%%%%%%%%%%%%%%%%%%%%%%%%%%%%%%%%%%%%%
%
% @ \mathscr
\def\scA{\mathscr{A}}
\def\scB{\mathscr{B}}
\def\scC{\mathscr{C}}
\def\scD{\mathscr{D}}
\def\scE{\mathscr{E}}
\def\scF{\mathscr{F}}
\def\scG{\mathscr{G}}
\def\scH{\mathscr{H}}
\def\scI{\mathscr{I}}
\def\scJ{\mathscr{J}}
\def\scK{\mathscr{K}}
\def\scL{\mathscr{L}}
\def\scM{\mathscr{M}}
\def\scN{\mathscr{N}}
\def\scO{\mathscr{O}}
\def\scP{\mathscr{P}}
\def\scQ{\mathscr{Q}}
\def\scR{\mathscr{R}}
\def\scS{\mathscr{S}}
\def\scT{\mathscr{T}}
\def\scU{\mathscr{U}}
\def\scV{\mathscr{V}}
\def\scW{\mathscr{W}}
\def\scX{\mathscr{X}}
\def\scY{\mathscr{Y}}
\def\scZ{\mathscr{Z}}		%	\mathscr{A} -> \scA
% !Mode:: "TeX:UTF-8"
%%%%%%%%%%%%%%%%%%%%%%%%%%%%%%%%%%%%%%%%%%%%%%
%              Letter Abbreviation File Made by Gau, Syu              %
%                                version = 2017.314                                 %
%                        contact = GauSyu@Gmail.com                        %
%%%%%%%%%%%%%%%%%%%%%%%%%%%%%%%%%%%%%%%%%%%%%%
%
% @ \mathfrak (capital)
\def\AAa{\mathfrak{A}}
\def\BBb{\mathfrak{B}}
\def\CCc{\mathfrak{C}}
\def\DDd{\mathfrak{D}}
\def\EEe{\mathfrak{E}}
\def\FFf{\mathfrak{F}}
\def\GGg{\mathfrak{G}}
\def\HHh{\mathfrak{H}}
\def\IIi{\mathfrak{I}}
\def\JJj{\mathfrak{J}}
\def\KKk{\mathfrak{K}}
\def\LLl{\mathfrak{L}}
\def\MMm{\mathfrak{M}}
\def\NNn{\mathfrak{N}}
\def\OOo{\mathfrak{O}}
\def\PPp{\mathfrak{P}}
\def\QQq{\mathfrak{Q}}
\def\RRr{\mathfrak{R}}
\def\SSs{\mathfrak{S}}
\def\TTt{\mathfrak{T}}
\def\UUu{\mathfrak{U}}
\def\VVv{\mathfrak{V}}
\def\WWw{\mathfrak{W}}
\def\XXx{\mathfrak{X}}
\def\YYy{\mathfrak{Y}}
\def\ZZz{\mathfrak{Z}}	%	\mathfrak{A} -> \AAa
% !Mode:: "TeX:UTF-8"
%%%%%%%%%%%%%%%%%%%%%%%%%%%%%%%%%%%%%%%%%%%%%%
%              Letter Abbreviation File Made by Gau, Syu              %
%                                version = 2017.314                                 %
%                        contact = GauSyu@Gmail.com                        %
%%%%%%%%%%%%%%%%%%%%%%%%%%%%%%%%%%%%%%%%%%%%%%
%
% @ \mathfrak (capital)
\def\frA{\mathfrak{A}}
\def\frB{\mathfrak{B}}
\def\frC{\mathfrak{C}}
\def\frD{\mathfrak{D}}
\def\frE{\mathfrak{E}}
\def\frF{\mathfrak{F}}
\def\frG{\mathfrak{G}}
\def\frH{\mathfrak{H}}
\def\frI{\mathfrak{I}}
\def\frJ{\mathfrak{J}}
\def\frK{\mathfrak{K}}
\def\frL{\mathfrak{L}}
\def\frM{\mathfrak{M}}
\def\frN{\mathfrak{N}}
\def\frO{\mathfrak{O}}
\def\frP{\mathfrak{P}}
\def\frQ{\mathfrak{Q}}
\def\frR{\mathfrak{R}}
\def\frS{\mathfrak{S}}
\def\frT{\mathfrak{T}}
\def\frU{\mathfrak{U}}
\def\frV{\mathfrak{V}}
\def\frW{\mathfrak{W}}
\def\frX{\mathfrak{X}}
\def\frY{\mathfrak{Y}}
\def\frZ{\mathfrak{Z}}	%	\mathfrak{A} -> \frA
% !Mode:: "TeX:UTF-8"
%%%%%%%%%%%%%%%%%%%%%%%%%%%%%%%%%%%%%%%%%%%%%%
%              Letter Abbreviation File Made by Gau, Syu              %
%                                version = 2017.314                                 %
%                        contact = GauSyu@Gmail.com                        %
%%%%%%%%%%%%%%%%%%%%%%%%%%%%%%%%%%%%%%%%%%%%%%
%
% @ \mathrm (little)
\def\aaa{\mathrm{a}}
\def\bbb{\mathrm{b}}
\def\ccc{\mathrm{c}}
\def\ddd{\mathrm{d}}
\def\eee{\mathrm{e}}
\def\fff{\mathrm{f}}
\def\ggg{\mathrm{g}}  % Knuth uses $\gg$ for ``>>''.
\def\hhh{\mathrm{h}}
\def\iii{\mathrm{i}}
\def\jjj{\mathrm{j}}
\def\kkk{\mathrm{k}}
\def\lll{\mathrm{l}}  % Knuth uses $\ll$ for ``<<''.
\def\mmm{\mathrm{m}}
\def\nnn{\mathrm{n}}
\def\ooo{\mathrm{o}}
\def\ppp{\mathrm{p}}
\def\qqq{\mathrm{q}}
\def\rrr{\mathrm{r}}
\def\sss{\mathrm{s}}
\def\ttt{\mathrm{t}}
\def\uuu{\mathrm{u}}
\def\vvv{\mathrm{v}}
\def\www{\mathrm{w}}
\def\xxx{\mathrm{x}}
\def\yyy{\mathrm{y}}
\def\zzz{\mathrm{z}}		%	\mathrm{a} -> \aaa
%
% @ Notations
% !Mode:: "TeX:UTF-8"
%%%%%%%%%%%%%%%%%%%%%%%%%%%%%%%%%%%%%%%%%%%%%%
%                      Notations File Made by Gau, Syu                       %
%                                version = 2017.314                                 %
%                        contact = GauSyu@Gmail.com                        %
%%%%%%%%%%%%%%%%%%%%%%%%%%%%%%%%%%%%%%%%%%%%%%
%
% @ Arrows
\def\acts{\curvearrowright}
\def\actson{\curvearrowright}
\def\actsby{\curvearrowleft}
\def\sendto{\rightsquigarrow}
%
\def\mono{\rightarrowtail}
\def\epi{\twoheadrightarrow}
\def\into{\hookrightarrow}
\def\onto{\twoheadrightarrow}
\def\isom{\overset{\sim}{\to}}
\def\Isom{\overset{\sim}{\longrightarrow}}
%
\def\from{\leftarrow}
\def\To{\longrightarrow}
\def\From{\longleftarrow}
\def\tofrom{\leftrightarrow}
\def\ToFrom{\longleftrightarrow}
\def\tto{\rightrightarrows}
\def\ffrom{\leftleftarrows}
%
\def\then{\Rightarrow}
\def\Then{\Longrightarrow}
\def\down{\downarrow}
\def\up{\uparrow}			%	Alternative commands for arrows
% !Mode:: "TeX:UTF-8"
%%%%%%%%%%%%%%%%%%%%%%%%%%%%%%%%%%%%%%%%%%%%%%
%                      Notations File Made by Gau, Syu                       %
%                                version = 2017.314                                 %
%                        contact = GauSyu@Gmail.com                        %
%%%%%%%%%%%%%%%%%%%%%%%%%%%%%%%%%%%%%%%%%%%%%%
%
% @ Functions
\newcommand{\norm}[1]{\lVert #1\rVert}    %	Norm
\newcommand{\dual}[1]{{#1}^{\wedge}}      %	Dual
\newcommand{\codual}[1]{{#1}^{\vee}}      %	Codual
\newcommand{\pfrac}[2]{\frac{\partial{#1}}{\partial{#2}}}		%	fraction of partial differentials
\newcommand{\bfrac}[2]{\left(\frac{#1}{#2}\right)}					%	Legendre cymbol
\newcommand{\udsum}[3]{\left.{#1}\right|^{#2}_{#3}}				%	1^2_3
\newcommand{\local}[2]{\left.{#1}\right|_{#2}}						%	Local #1 at #2
%
\newcommand{\markar}[1]{\stackrel{{#1}}{\longrightarrow}}	%	marked rightarrow
\newcommand{\markal}[1]{\stackrel{{#1}}{\longleftarrow}}	%	marked leftarrow
\newcommand{\iso}[1]{\stackrel{#1}{\cong}}								%	marked isomorphism
\newcommand{\defeq}{\stackrel{\text{def}}{=}}							%	definition equality
\newcommand{\defen}{\stackrel{\text{def}}{\iff}}						%	definition condition
%
\newcommand{\tdtlog}[1]{t\frac{\rm d}{{\rm d} t}\log\left({#1}\right)}
\newcommand{\kongwei}{\underline{\hphantom{X}}}
%
\newcommand{\tp}[1]{\prescript{\rm t}{}{#1}}
%
% @ Map Descriptions
\newcommand{\mapdes}[4]
{
	\begin{align*}
	#1 & \longrightarrow #2 \\
	#3 & \longmapsto #4
	\end{align*}
}
\newcommand{\longmapdes}[5]
{
	\begin{align*}
	#1\colon  #2 & \longrightarrow  #3 \\
	#4 & \longmapsto  #5
	\end{align*}
}
\newcommand{\isodes}[4]
{
	\begin{align*}
	#1 & \cong  #2 \\
	#3 & \leftrightarrow  #4
	\end{align*}
}		%	Functions and Map Descriptions
% !Mode:: "TeX:UTF-8"
%%%%%%%%%%%%%%%%%%%%%%%%%%%%%%%%%%%%%%%%%%%%%%
%                      Notations File Made by Gau, Syu                       %
%                                version = 2017.314                                 %
%                        contact = GauSyu@Gmail.com                        %
%%%%%%%%%%%%%%%%%%%%%%%%%%%%%%%%%%%%%%%%%%%%%%
%
%	\ensuremath{---}\xspace may be helpful
%
% @ common Operators
\DeclareMathOperator{\id}{id}			%	identity
\DeclareMathOperator{\Id}{Id}			%	Identity
\DeclareMathOperator{\obj}{obj}			%	identity
\DeclareMathOperator{\pr}{pr}			%	projection
\DeclareMathOperator{\pt}{pt}			%	point
\DeclareMathOperator{\res}{res}			%	restriction
%
% @ Operators in a category
\DeclareMathOperator{\coim}{coim}
\DeclareMathOperator{\Coim}{Coim}
\DeclareMathOperator{\coker}{coker}
\DeclareMathOperator{\Coker}{Coker}
\DeclareMathOperator{\im}{im}
\DeclareMathOperator{\Image}{Im}
\DeclareMathOperator{\Ker}{Ker}
\DeclareMathOperator{\Sat}{Sat}			%	Saturated closure
\DeclareMathOperator{\sk}{sk}
\DeclareMathOperator{\ob}{ob}%Object
%
% @ Operators in linear algebra
\DeclareMathOperator{\Ad}{Ad}			%	Adjunction
\DeclareMathOperator{\ad}{ad}			%	adjuntion
\DeclareMathOperator{\diag}{diag}
\DeclareMathOperator{\Det}{Det}			%	Determinant
\DeclareMathOperator{\Exp}{Exp}			%	Expotential
\DeclareMathOperator{\Hess}{Hess}		%	Hessian
\DeclareMathOperator{\Span}{span} 		%	Annoyingly, \span is already a command in TeX, and redefining it leads to other problems
\DeclareMathOperator{\Tr}{Tr}				%	trace
\DeclareMathOperator{\tr}{tr}				%	trace
\DeclareMathOperator{\wt}{wt}				%	trace
%
% @ Operators in sub(p)scripts
\DeclareMathOperator{\ab}{ab}		%	abelian
\DeclareMathOperator{\an}{an}		%	analytic
\DeclareMathOperator{\conv}{conv}%	convergent
\DeclareMathOperator{\ev}{ev}		%	evaluation
\DeclareMathOperator{\opp}{op}	%	opposite
\DeclareMathOperator{\pre}{pre}
\DeclareMathOperator{\rev}{rev}		%	reverse
\DeclareMathOperator{\sep}{sep}		%	separated
\DeclareMathOperator{\tor}{tor}		%	torsion
%
% @ Operators in commutative algebra
\DeclareMathOperator{\Ann}{Ann}			%	Annihilator
\DeclareMathOperator{\Char}{Char}		%	Characteristic
\DeclareMathOperator{\gr}{gr}			%	associated graded object
\DeclareMathOperator{\Gr}{Gr}			%	associated graded object
\DeclareMathOperator{\SGr}{\mathscr{G}\!\mathit{r}}%	associated graded object
\DeclareMathOperator{\lcm}{lcm}			%	least common multiple
\DeclareMathOperator{\ord}{ord}			%	order
\DeclareMathOperator{\rad}{rad}			%	radical
\DeclareMathOperator{\Rad}{Rad}			%	Radical
\DeclareMathOperator{\rank}{rank}		%	rank
\DeclareMathOperator{\mult}{mult}		%	multiplicity
\DeclareMathOperator{\Length}{Length}		%	length
\DeclareMathOperator{\sgn}{sgn}			%	signature
\DeclareMathOperator{\supp}{supp}	%	support
\DeclareMathOperator{\Supp}{Supp}	%	Support
\DeclareMathOperator{\vol}{vol}				%	volume
\DeclareMathOperator{\Vol}{Vol}				%	Volume
%
% @ Operators of constructions
\DeclareMathOperator{\Alt}{Alt}				%	Alternative
\DeclareMathOperator{\Sym}{Sym}			%	Symmetric
\DeclareMathOperator{\Ext}{Ext}				%	Extensions
\DeclareMathOperator*{\EExt}{\mathcal{E}xt}	%	Cal Ext
\DeclareMathOperator{\Proj}{Proj}			%	Proj Spectrum
\DeclareMathOperator{\Quot}{Qout}		%	Quotient
\DeclareMathOperator{\Spec}{Spec}		%	Spectrum
\DeclareMathOperator{\Spm}{Spm}		%	Maximal Spectrum
\DeclareMathOperator{\Tor}{Tor}				%	Torsors
%
\DeclareMathOperator{\Der}{Der}			%	Derivations
\DeclareMathOperator{\SDer}{\mathscr{D}\!\mathit{er}}			%	Sheaf Derivations
\DeclareMathOperator{\Diff}{Diff}			%	Differential operators
\DeclareMathOperator{\SDiff}{\mathscr{D}\!\mathit{iff}}			%	Differential operators
\DeclareMathOperator{\Gal}{Gal}			%	Galois group
\DeclareMathOperator{\Pic}{Pic}			%	Picard group
\DeclareMathOperator{\Res}{Res}			%	Residue
%
% @ Hom-like Operators
\DeclareMathOperator{\Aut}{Aut}		%	automorphisms
\DeclareMathOperator{\Bil}{Bil}				%	Bilinear maps
\DeclareMathOperator{\End}{End}		%	Endomorphisms
\DeclareMathOperator{\Fun}{Fun}			%	Functors
\DeclareMathOperator{\hFun}{hFun}			%	Functors
\DeclareMathOperator{\FFun}{\mathcal{F}un}			%	Functors
\DeclareMathOperator{\GrHom}{GrHom}	%	Graded Hom
\DeclareMathOperator{\Hom}{Hom}			%	Hom
\DeclareMathOperator{\inEnd}{\underline{End}}			%	Internal End
\DeclareMathOperator{\inHom}{\underline{Hom}}			%	Internal Hom
\DeclareMathOperator{\inGrHom}{\underline{GrHom}}	%	Internal Graded Hom
\DeclareMathOperator{\Nat}{Nat}				%	Nature transformations
\DeclareMathOperator{\Mor}{Mor}		%	Morphisms
\DeclareMathOperator{\Map}{Map}		%	Maps
\DeclareMathOperator{\HHom}{\mathcal{H}\!\mathit{om}}			%		Hom space
\DeclareMathOperator{\SHom}{\mathscr{H}\!\!\mathit{om}}			%	sheaf Hom
\DeclareMathOperator{\SEnd}{\mathscr{E}\!\mathit{nd}}			%	sheaf Hom
\def\SheafHom{\SHom}
\DeclareMathOperator{\cont}{\mathit{cont}}
%
% @ Limit Operators
\DeclareMathOperator*{\holim}{\mathrm{holim}}
\DeclareMathOperator*{\hocolim}{\mathrm{hocolim}}		
\DeclareMathOperator*{\dirlim}{\underrightarrow{\lim}}			%	direct limit
\DeclareMathOperator*{\colim}{\mathrm{colim}}			%	direct limit
\DeclareMathOperator*{\invlim}{\underleftarrow{\lim}}			%	inverse limit
\DeclareMathOperator*{\Rinvlim}{\underleftarrow{\lim}^1}	%	derived inverse limit
\DeclareMathOperator*{\Lan}{Lan}				%	Left Kan extension
\DeclareMathOperator*{\Ran}{Ran}				%	Right Kan extension
%
%
% @ Operators in number theory
\DeclareMathOperator{\Frob}{Frob}		%	Frobenuis
\DeclareMathOperator*{\sump}{\sum\nolimits^\prime}			%	sum outside origin
\renewcommand{\mod}{\mathop{\mathrm{mod}}}
\makeatletter
\renewcommand{\pod}[1]{\allowbreak\mathchoice
	{\if@display \mkern 18mu\else \mkern 8mu\fi (#1)}
	{\if@display \mkern 18mu\else \mkern 8mu\fi (#1)}
	{\mkern4mu(#1)}
	{\mkern4mu(#1)}
}
\makeatother		%	Math Operators
% !Mode:: "TeX:UTF-8"
%%%%%%%%%%%%%%%%%%%%%%%%%%%%%%%%%%%%%%%%%%%%%%
%                      Notations File Made by Gau, Syu                       %
%                                version = 2017.314                                 %
%                        contact = GauSyu@Gmail.com                        %
%%%%%%%%%%%%%%%%%%%%%%%%%%%%%%%%%%%%%%%%%%%%%%
%
% @ Some notations
\newcommand{\Calrm}[2]{\mathcal{#1}\mathrm{#2}}
\newcommand{\<}{\langle}
\renewcommand{\>}{\rangle}
\renewcommand{\le}{\leqslant}
\renewcommand{\ge}{\geqslant}
\def\di{\mathrm{d}}
\def\dt{\frac{\di}{\di t}}
\def\st{\textrm{ s.t. }}
%
% @ Spaces
\def\RP{\mathbb{R}\mathbf{P}}
\def\CP{\mathbb{C}\mathbf{P}}
\def\Real{\mathbb{R}}
\def\dR{\mathrm{dR}}
\def\DB{\mathscr{D}\mathscr{b}}
%
% @ Groups
\def\GL{\mathrm{GL}}	%	general linear group
\def\SL{\mathrm{SL}}	%	special linear group
\def\GO{\mathrm{O}}		%	general orthogonal group
\def\SO{\mathrm{SO}}	%	special orthogonal group
\def\GU{\mathrm{U}}		%	general unitary group
\def\SU{\mathrm{SU}}	%	special unitary group
\def\Sp{\mathrm{Sp}}	%	symplectic group
\def\Spin{\mathrm{Spin}}%	Spin group
\def\PGL{\mathrm{PGL}}	%	proj GL
\def\PSL{\mathrm{PSL}}	%	proj SL
%
% @ Lie algebras
\def\gl{\mathfrak{gl}}		%	general linear Lie algebra
\def\sl{\mathfrak{sl}}		%	special linear Lie algebra
\def\go{\mathfrak{o}}	%	orthogonal
\def\so{\mathfrak{so}}	%	special orthogonal
\def\sp{\mathfrak{sp}}	%	symplectic
\def\gu{\mathfrak{u}}	%	unitary
\def\su{\mathfrak{su}}	%	special unitary
%
% @ Number theory staffs
\def\dedekind{{\scriptstyle\mathcal{O}}}
\def\Dedekind{\mathcal{O}}
%
% @ Categories
\def\Set{\mathbf{Set}}
\def\Sub{\mathbf{Sub}}
\def\Cat{\mathbf{Cat}}
\def\Top{\mathbf{Top}}
\def\Et{\mathbf{Et}}
\def\EtBund{\mathbf{EtBund}}
\def\PSh{\mathbf{PSh}}
\def\Sh{\mathbf{Sh}}
\def\Sch{\mathbf{Sch}} %Schemes
\def\Aff{\mathbf{Aff}}
\def\El{\mathbf{El}}
\def\Fin{\mathbf{Fin}}
\def\Alg{\mathbf{Alg}}
\def\Vect{\mathbf{Vect}}
\def\CRing{\mathbf{CRing}}
\def\Ring{\mathbf{Ring}}%Ring
\def\Ab{\mathbf{Ab}}%Abelian group 
\def\PAb{\mathbf{PAb}}
\def\Grp{\mathbf{Grp}}%Group
\def\Grpd{\mathbf{Grpd}}%Groupoid
\def\Mon{\mathbf{Mon}}%Monoid
\def\CMon{\mathbf{CMon}}%Monoid
\def\CoMon{\mathbf{CoMon}}%Monoid
\def\BiMon{\mathbf{BiMon}}%Monoid
\def\Mod{\mathbf{Mod}}%Module
\def\PMod{\mathbf{PMod}}%Module
\def\delooping{\mathbf{B}}
\def\Rep{\mathbf{Rep}}%Representations
\def\Loc{\mathbf{Loc}}
\def\SpSeq{\mathbf{SpSeq}}
\def\QCoh{\mathbf{QCoh}}
\def\Coh{\mathbf{Coh}}
\def\SmoothMfd{\mathbf{SmoothMfd}}
\def\Fil{\mathrm{Fil}}
\def\fib{\mathrm{fib}}	%Fibrations
\def\cofib{\mathrm{cofib}}	%Cofibrations
\def\Fib{\mathrm{Fib}}	%Fibrations
\def\Cofib{\mathrm{Cofib}}	%Cofibrations
\def\Cyl{\mathrm{Cly}}	%Cylinder
\def\Path{\mathrm{Path}}	%Path
\def\unit{\mathbf{1}}
\def\MC{\mathbf{MC}}
\def\Ch{\mathbf{Ch}}

\def\Ho{\mathrm{Ho}}%Homotopy
\def\h{\mathrm{h}}%Homotopy
\def\s{\mathrm{s}}%Homotopy
\def\dg{\mathrm{dg}}

% @ Miscellana
\def\vac{|0\>}
\newcommand{\nlab}[1]{\href{http://ncatlab.org/nlab/show/#1}{{#1}} }
\newcommand{\arxiv}[1]{\href{https://arxiv.org/abs/#1}{{#1}} }
\def\qbface{({\red·}‿{\red·})}			%	Math Names
%\input{Letters_and_Notations}

%	Theorem Environments
% !Mode:: "TeX:UTF-8"
%%%%%%%%%%%%%%%%%%%%%%%%%%%%%%%%%%%%%%%%%%%%%%
%               Theorems Setting File Made by Gau, Syu                %
%                                version = 2018April                       %
%                        contact = GauSyu@Gmail.com                        %
%%%%%%%%%%%%%%%%%%%%%%%%%%%%%%%%%%%%%%%%%%%%%%
%
% @	Known keys to \declaretheorem
%%	[parent=<counter>] (alias: numberwithin, within) : reset whenever that counter is incremented
%%	[sibling=<counter>] (alias: numberlike, sharenumber) : use this counter for numbering
%%	[title=<TEX>] (alias: name, heading) : title of the theorem, default is the name of the environment
%%	[numbered=<yes/no>] : will be numbered or not
%%	[style=<theoremstyle>] : use the settings of this style
%%	[refname=<string>] : the name of the theorem as used by \autoref, \cref and friends
%%	[refname=<string1,string2>] : the second is the plural form used by \cref
%%	[Refname=<string>] : the name of the theorem as used by \Autoref, \Cref and friends
%%	[Refname=<string1,string2>] : the second is the plural form used by \Cref
%
% @ Known keys to \declaretheoremstyle
%%	[spaceabove=<length>]
%%	[spacebelow=<length>]
%%	[headfont=<font switches>]
%%	[bodyfont=<font switches>]
%%	[headpunct=<TEX>] : theoremseparator
%%	[postheadspace=<length>] : horizontal space inserted after the entire head of the theorem, before the body
%%	[headindent=<length>] : horizontal space inserted before the head
%%	[headformat=<TEX> or <keywords>] : see manuel
%%	[headformat=margin] : number protude in the (left) margin
%%	[headformat=swapnumber] : puts number before the name
%
% @ Thm-type
\theoremstyle{margin}	%	Put the number at left margin
\numberwithin{equation}{section}	%	reset eq. number for each section
%
\declaretheorem[title=Theorem,numberlike=equation,%
refname={theorem,theorems},Refname={Theorem,Theorems}%
]{theorem}
\def\thm{\theorem}
\declaretheorem[title=Theorem,numbered=no]{Theorem}
\def\Thm{\Theorem}
%
\declaretheorem[title=Lemma,numberlike=equation,%
refname={lemma,lemmas},Refname={Lemma,Lemmas}%
]{lemma}
\def\lem{\lemma}
\declaretheorem[title=Lemma,within=equation,%
refname={lemma,lemmas},Refname={Lemma,Lemmas}%
]{sublemma}
\def\slem{\sublemma}
\declaretheorem[title=Lemma,numbered=no]{Lemma}
\def\Lem{\Lemma}
%
\declaretheorem[title=Proposition,numberlike=equation,%
refname={proposition,propositions},Refname={Proposition,Propositions}%
]{proposition}
\def\prop{\proposition}
\declaretheorem[title=Proposition,numberlike=sublemma,%
refname={proposition,propositions},Refname={Proposition,Propositions}%
]{subproposition}
\def\sprop{\subproposition}
\declaretheorem[title=Proposition,numbered=no]{Proposition}
\def\Prop{\Proposition}
%
\declaretheorem[title=Corollary,numberlike=equation,%
refname={corollary,corollaries},Refname={Corollary,Corollaries}%
]{corollary}
\def\cor{\corollary}
\declaretheorem[title=Corollary,numberlike=sublemma,%
refname={corollary,corollaries},Refname={Corollary,Corollaries}%
]{subcorolloary}
\def\scor{\subcorollary}
\declaretheorem[title=Corollary,numbered=no]{Corollary}
\def\Cor{\Croposition}
%
% @ Defn-type
\theorembodyfont{\upshape}	%	following envs will use upshape bodyfont
%
\declaretheorem[title={},numberlike=equation]{para}
\declaretheorem[title={},numberlike=sublemma]{subpara}
%
\declaretheorem[title={Definition},numberlike=equation,%
refname={definition,definitions},Refname={Definition,Definitions}%
]{definition}
\def\defn{\definition}
\declaretheorem[title={Definition},numbered=no]{Definition}
\def\Defn{\Definition}
%
\declaretheorem[title=Example,numberlike=equation,%
refname={example,examples},Refname={Example,Examples}%
]{example}
\def\eg{\example}
\declaretheorem[title=Example,numberlike=sublemma,%
refname={example,examples},Refname={Example,Examples}%
]{subexample}
\def\seg{\subexample}
\declaretheorem[title=Example,numbered=no]{Example}
\def\Eg{\Example}
%
\declaretheorem[title=Remark,numberlike=theorem,%
refname={remark,remarks},Refname={Remark,Remarks}%
]{remark}
\def\rem{\remark}
\declaretheorem[title=Remark,numberlike=sublemma,%
refname={remark,remarks},Refname={Remark,Remarks}%
]{subremark}
\def\srem{\subremark}
\declaretheorem[title=Remark,numbered=no]{Remark}
\def\Rem{\Remark}
%
\declaretheorem[title=Recall,numbered=no]{recall}
%
% @ Proof Env
\def\qedsymbol{\ensuremath{\Box}}	%	desige the qed symbol
%
\declaretheoremstyle[%
headfont=\sc,%
bodyfont=\normalfont,%
headpunct={:},%
qed=\qedsymbol%
]{nonumberproof}
%
\declaretheorem[%
title=Proof,%
numbered=no,%
style=proof%
]{proof}
%	proof
	
% !Mode:: "TeX:UTF-8"
%%%%%%%%%%%%%%%%%%%%%%%%%%%%%%%%%%%%%%%%%%%%%%
%              A Common Setting File Made by Gau, Syu               %
%                                version = 2018.April                     %
%                        contact = GauSyu@Gmail.com                        %
%%%%%%%%%%%%%%%%%%%%%%%%%%%%%%%%%%%%%%%%%%%%%%
%
% @ Thm Ref
\usepackage{nameref}
\usepackage[nameinlink]{cleveref}
\newcommand{\crefrangeconjunction}{--}
\Crefname{equation}{}{}
%
% !Mode:: "TeX:UTF-8"
%%%%%%%%%%%%%%%%%%%%%%%%%%%%%%%%%%%%%%%%%%%%%%
%               Theorems Setting File Made by Gau, Syu                %
%                                version = 2018April                       %
%                        contact = GauSyu@Gmail.com                        %
%%%%%%%%%%%%%%%%%%%%%%%%%%%%%%%%%%%%%%%%%%%%%%
%
% @ refer items in lists inside thm envs
%
\newlist{thmlist}{enumerate}{1}
\setlist[thmlist]{label=(\roman{thmlisti}), ref=\thetheorem.(\roman{thmlisti}),noitemsep}
\newcounter{listthm}
\Crefname{listthm}{Theorem}{Theorems}
\addtotheorempostheadhook[theorem]{\crefalias{thmlisti}{theorem}}
\newlist{proplist}{enumerate}{1}
\setlist[proplist]{label=(\roman{proplisti}), ref=\theproposition.(\roman{proplisti}),noitemsep}
\newcounter{listprop}
\Crefname{listprop}{}{}
\addtotheorempostheadhook[proposition]{\crefalias{proplisti}{proposition}}
%
%	PDF File Information
%
\hypersetup{
pdftitle={Homological Algebra},%标题
pdfauthor={Xu Gao},            %作者
%pdfproducer={XeLaTeX},        %制作工具
bookmarksopen=true,         %书签自动打开
colorlinks=true,                       %是否采用彩色超链接
citecolor=red,                       %文献引用的颜色
filecolor=black,                       %文件链接颜色
linkcolor=black,                       %内部链接颜色
urlcolor=darkgray                         %网页与电邮链接颜色
}
%	Bib
\bibliography{references}
%
%	TITLE
%
\title{
Note on\\ \texttt{\Huge Homological Algebra}
}
\author{Xu Gao}
\date{
Last update:
\today
}

\begin{document}
\maketitle
\begin{abstract}
This note is on homological algebra with a 
homotopy-theoretical perspective and aims to 
introduce a framework for homotopy theory 
based on the notion of dg-categories. 
Such a framework, as I know, is a special case 
of the full general machinery of infinite-category 
theory and thus should be thought as well-known 
fact or even common sense. 
\end{abstract}
\tableofcontents

%	MainText
\clearpage

\section{Homotopy theory for topological spaces}
Before going to the main topics of this note, 
let's take a glance to the homotopy theory. 
One can refer to either a standard textbook on algebraic 
topology like \cite{Hatcher}, or a homotopy-first 
textbook like \cite{ATfromHomotopy}, or the wonderful 
textbook \cite{may1999concise}. 
For further reading, refer \cite{may2012more}.

\para{
Let $f,g\colon X\to Y$ be two (continues) maps between 
topological spaces, a \bfblue{(left) homotopy} 
$\Phi\colon f\then g$ is a commutative diagram 
(in the category of topological spaces) of the form 
\[
\begin{tikzcd}
X\ar[dr,"{f}"]\ar[d,"{(\id,\delta_0)}"']&\\
X\times I\ar[r,"{\Phi}"description]&Y\\
X\ar[ur,"{g}"']\ar[u,"{(\id,\delta_1)}"]
\end{tikzcd}
\]
where $I$ is the unit interval $[0,1]$ and $\delta_0$ 
(resp. $\delta_1$) is the inclusion $\{0\}\into I$ 
(resp. $\{1\}\into I$).
If such a homotopy exists, then we say $f$ and $g$ are 
\bfblue{homotopic}, denoted by $f\simeq g$. 
Let $x_0\in X$ and $y_0\in Y$ be base points and suppose 
$f$ and $g$ preserve the base point. Then $\Phi$ 
is called a \bfblue{based homotopy} if $\Phi(x_0,t)=y_0$ 
for all $t\in I$. 
More generally, let $A\subset X$ and $B\subset Y$ be 
subspaces and $\local{f}{A}=\local{g}{A}$ and 
$f(A)\subset B$. Then $\Phi$ is called a 
\bfblue{relative homotopy} or \bfblue{homotopy rel $A$} 
if $\Phi(x,t)=f(x)$ for all $x\in A$.
To emphasize the base point $x_0$, or the subspace $A$, 
we use the notations $f\simeq_{x_0}g$ or $f\simeq_{A}g$ 
to denote that $f$ and $g$ are \bfblue{based homotopic} 
or \bfblue{homotopic rel $A$}.

The set $\Map(X,Y)$ of all continues maps from $X$ 
to $Y$, equipped with the compact-open topology, is 
called the \bfblue{mapping space} from $X$ to $Y$. 
If $X$ is a good topological space, for instant 
a locally compact Hausdorff space, then there is a 
natural bijection 
\[
\Map(Z\times X,Y)\cong\Map\big(Z,\Map(X,Y)\big),
\] 
where $Z\times X$ carries the product topology. 
If this is the case, then the exponential law implies that 
there is a natural bijection between the set of homotopy 
classes of maps $X\to Y$ and the set of path-components of 
$\Map(X,Y)$. This set will be denoted by $[X,Y]$, 
called the \bfblue{free homotopy class set}.

Let $A\subset X$ and $B\subset Y$ be subspaces. The 
\bfblue{product} of the pairs $(X,A)$ and $(Y,B)$ is 
the pair $(X\times Y,X\times B\cup A\times Y)$.
The subspace $\Map(X,A;Y,B)$ of $\Map(X,Y)$ consists of 
those maps $f\colon X\to Y$ satisfying $f(A)\subset B$. 
It is called the \bfblue{(relative) mapping space} 
from $(X,A)$ to $(Y,B)$. 
There is a special subspace of it, which consists of those 
factoring through $B$, thus can be identified to 
$\Map(X,B)$.
Again, if $(X,A)$ is good enough, 
then there is a natural bijection 
\[
\Map\big(Z\times X,Z\times A\cup C\times X;Y,B\big)\cong
\Map\big(Z,C;\Map(X,A;Y,B),\Map(X,B)\big). 
\]
Let $(Z.C)$ be $(I,\varnothing)$, 
then we see that if $(X,A)$ is good enough, then 
there is a natural bijection between the set of 
relative homotopy classes of maps $(X,A)\to (Y,B)$ 
and the set of path-components of $\Map(X,A;Y,B)$. 
This set is denoted by $[X,A;Y,B]$, 
called the \bfblue{relative homotopy class set}.

Let $(X,x_0)$ and $(Y,y_0)$ are pointed spaces, i.e. 
topological spaces with a base point. 
The subspace $\Map(X,x_0;Y,y_0)$ is simply denoted by 
$\Map_{\ast}(X,Y)$, called the \bfblue{(based) mapping space}. 
(In many case, the base point is clear or irrelevant to the 
discussion, we should simplify our notation by just write $X$ 
instead of $(X,x_0)$.) 
If $X$ is good enough, from the previous paragraph, 
there is a natural bijection between the set of 
based homotopy classes of based maps $X\to Y$ 
and the set of path-components of $\Map_{\ast}(X,Y)$. 
This set will be denoted by $[X,Y]_{\ast}$, or $\<X,Y\>$, 
called the \bfblue{based homotopy class set}. 
Beside the Cartesian product, 
there is another \emph{tensor product} of pointed spaces, 
which is the \bfblue{smash product} $X\wedge Y$: 
it is precisely the pointed space obtained from the pair
$(X\times Y,X\vee Y)$ by modulo the later, 
where $X\vee Y$ is the wedge sum. 
There is a natural base point of $\Map_{\ast}(X,Y)$, that 
is the map $\widetilde{y_0}\colon X\to\{y_0\}$. 
In the case $X$ is good enough, there is a natural bijection 
\[
\Map_{\ast}(Z\wedge X,Y)\cong\Map_{\ast}(Z,\Map_{\ast}(X,Y)).
\]

}

\para{
Before going further, notice that the natural objection
\[
\Map(X\times I,Y)\cong\Map\big(X,\Map(I,Y)\big)
\]
gives another equivalent definition of homotopy: 
let $f,g\colon X\to Y$ be two maps between 
topological spaces, a \bfblue{right homotopy} 
$\Phi\colon f\then g$ is a commutative diagram of the form 
\[
\begin{tikzcd}
& Y\\
X\ar[r,"{\Phi}"description]\ar[ur,"{f}"]\ar[dr,"{g}"']&
\Map(I,Y)\ar[u,"{\ev_0}"']\ar[d,"{\ev_1}"]\\
& Y
\end{tikzcd}
\]
where $\ev_0$ (resp. $\ev_1$) is the evaluation at $0\in I$ 
(resp. $1\in I$).
}

\para{
One can also define the notion of based homotopy using pure 
diagrammatic language. 
Write $Y_{+}$ for the pointed space obtained as the union of 
$Y$ and a disjoint base point $\ast$. Note that if $X$ is a 
pointed space, then $X\wedge Y_{+}$ can be identified with 
the one obtained from the pair $(X\times Y,\{\ast\}\times Y)$ 
and $\Map_{\ast}(Y_{+},X)$ can be identified as $\Map(Y,X)$ 
specified the base point to be the map collapsing to the 
base point of $X$.
Let $f,g\colon X\to Y$ be two based maps between pointed spaces. 
A \bfblue{based homotopy} $\Phi\colon f\then g$ can be defined 
as a commutative diagram of the form
\[
\begin{tikzcd}
X\ar[dr,"{f}"]\ar[d,"{\id\wedge\delta_0}"']&\\
X\wedge I_{+}\ar[r,"{\Phi}"description]&Y\\
X\ar[ur,"{g}"']\ar[u,"{\id\wedge\delta_1}"]
\end{tikzcd}
\]
where inclusions $\delta_i$ are viewed as $\{i\}_{+}\into I_{+}$. 
Using the natural bijection for pointed spaces, a 
\bfblue{based right homotopy} can be defined as the same 
commutative diagram for right homotopy with additional requirement 
that all maps involved must be based.
\begin{Rem}
The functor $Y\mapsto Y_{+}$ is in fact the left adjoint of 
the forgetful functor from $\Top$ to $\Top_{\ast}$, the category 
of pointed spaces with based maps. 
\end{Rem}
}

\para{
The topological spaces with continuous maps form a category $\Top$.
However, this category lost informations since it 
ignores the topologies on the mapping spaces. 
A better category is the one obtained by replacing every 
mapping space by the corresponding homotopy class set\footnote{
There is an issue that the notion of homotopy class sets, 
although can be defined for arbitrary topological spaces, 
does not behave well unless the topological space is good 
enough. Therefore, it is better to work on a subcategory 
of $\Top$ consisting of \emph{good topological spaces}, or on 
a \emph{convenient category of topological spaces} instead of $\Top$. 
For the purpose of this note, we ignore this issue.
}.
This can be done since homotopy respect the composition 
of maps. The result category is called 
\bfblue{the homotopy category $\Hh$}. 
Two topological spaces are said to be 
\bfblue{(strong) homotopy equivalent} 
if they are isomorphic in $\Hh$.

Similar discussion apply to relative and pointed spaces.
}

\para{
Let $(X,x_0)$ be a pointed space. Then a 
\bfblue{(based) loop} on $(X,x_0)$ is a 
base-point-preserving map from $(S^1,\ast)$, where 
$\ast$ is a fixed base point of $S^1$, to it. 
$\Map_{\ast}(S^1,X)$ is called the 
\bfblue{loop space} on it, denoted by $\Omega(X,x_0)$ 
or simply $\Omega X$. 
There is a natural ``multiplication'' on this space: 
any two such loops can be concatenated to obtain a 
third loop. Although this ``multiplication'' is not 
associative, it does induce an associative 
multiplication on the quotient set $\pi_1(X,x_0)$ 
of it by modulo the based homotopies. 
The set $\pi_1(X,x_0)$ then carries a group structure 
and is called the \bfblue{fundamental group} of $(X,x_0)$.

Similarly, one can define the \bfblue{$n$-th homotopy group} 
as $\pi_n(X,x_0)=[S^n,X]_{\ast}$ with the addition induced 
by $c\colon S^n\to S^n\vee S^n$ where $c$ collapses 
a equator $S^{n-1}$ (containing the base point) in $S^n$ 
to the base point.
As the notation suggests, $\pi_0(X,x_0)$ should be 
$[S^0,X]_{\ast}$, where $S^0$ is the $0$-sphere, 
i.e. the set of two points with one of them being 
the base point. Note that there is no natural group structure 
on it anymore. Since $S^0$ is merely a set of two points 
and one of them must be mapped to $x_0$, the space 
$\Map_{\ast}(S^0,X)$ is homeomorphic to $\Map(\pt,X)$ and 
hence $X$ itself. Thus $\pi_0(X,x_0)$ actually has nothing 
to do with $x_0$ and is precisely the set of path-components 
of $X$. 
%For this reason, this set is often simply denoted by $\pi_0(X)$. $\pi_n(X,x_0)$ to $\pi_n(X)$.

Note that for $(X,A)$ a pair of space and subspace and $(Y,y_0)$ 
a pointed space, there is a canonical bijection 
$[X,A;Y,y_0]\cong[X/A,[A];Y,y_0]$. 
Thus the $n$-the homotopy group can also be defined as 
$[I^n,\partial I^n;X,x_0]$ with the addition induced by 
concatenation (there are $n$ different ways to do this, 
but by the \emph{Eckmann-Hilton argument}, they all give the 
same commutative binary operation on the homotopy class set).
This characterization is easier to compute. 
}

\para{
Note that we have a natural bijection 
\[
\Map_{\ast}(X\wedge S^1,Y) \cong 
\Map_{\ast}(X,\Omega Y)
\]
for any pointed spaces $X$ and $Y$. 
Let $\Sigma X$ denote the pointed space $X\wedge S^1$. 
It is called the \bfblue{suspension} of $X$. 
From this we get
\[
\pi_n(X) = [\Sigma^nS^0,X]_{\ast} = \pi_0(\Omega^nX).
\]

}

\para{
We can always view the loop space $\Omega(X,x_0)$ as a subspace of 
$\Map(I,X)$ by identify it as $\Map(I,\partial I;X,x_0)$. 
Note that there are two canonical maps from $\Map(I,X)$ to 
$X$: one maps $f\colon I\to X$ to $f(0)$, another to $f(1)$. 
If we ignore the issue that concatenation is not strict 
associative, those data defines a \emph{topological groupoid}. 
To fix this issue, we can consider $[I,X]$ instead of
$\Map(I,X)$. Then the result construction is a \emph{groupoid}, 
called the \bfblue{fundamental groupoid} of $X$ and denoted 
by $\Pi_1(X)$. If $X$ is good enough 
(locally path-connected and locally simply-connected),
then $[I,X]$ has a natural topology on it and 
$\Pi_1(X)$ becomes a \emph{topological groupoid}.

In any case, using those two maps, we obtain a bundle 
$[I,X]\to X\times X$ whose fiber at any point $(x_0,x_0)$
in the diagonal is precisely $\pi_1(X,x_0)$. 
Thus, if we pullback it along the diagonal map 
$\Delta\colon X\to X\times X$, we obtain a bundle above $X$, 
or equivalently a sheaf on $X$. 
This is another realization of the notion of 
\emph{fundamental groupoid}. 

It is clear that the fundamental groupoid $\Pi_1(X)$ encodes 
the information of homotopies between points, i.e. paths 
connecting them, and is essentially 
(up to equivalences of categories)
determined by $\pi_0(X)$ and $\pi_1(X,x_0)$ with $x_0$ go 
through a presenting system of $\pi_0(X)$.
}

\para{
Then one may try to obtain a higher analogy of fundamental 
groupoids. That is a \emph{functorial} construction $\Pi(X)$ 
for each topological space $X$, which encodes the information 
of not only homotopies between points, but homotopies between 
homotopies, homotopies between those between homotopies and so on.
Moreover, $\Pi(X)$ must be essentially determined by 
$\pi_0(X)$ and $\pi_n(X,x_0)$ for all $n$ with $x_0$ go through 
a presenting system of $\pi_0(X)$. 
This object is called the \bfblue{homotopy type} 
or \bfblue{fundamental $\infty$-groupoid} of $X$. 

The later terminology suggests it should be an $\infty$-groupoid, 
i.e. an $\infty$-category with all morphisms invertible. 
Ideally, for a given topological space $X$, its points should 
be the objects of $\Pi(X)$, homotopies between them should be 
$1$-morphisms of $\Pi(X)$, homotopies between $1$-morphisms 
should be $2$-morphisms and so on.  
Conversely, there is a requirement of $\infty$-category theory 
called the \bfblue{homotopy hypothesis}, which states that 
the $\infty$-category of $\infty$-groupoid is equivalent 
(in the sense of $\infty$-category theory)
to the $\infty$-category of homotopy types. 

A na\"{i}ve approach is just define an $\infty$-groupoid as 
a topological space and an $\infty$-category as 
a category enriched over $\Hh$. However, this does not work 
due to the reason below.
}

\para{
Let $f\colon X\to Y$ be a map between topological spaces. 
We can view it as a based map by choosing a base point 
$x_0$ of $X$. Then, by composing with $f$, we obtain 
natural maps $\Map_{\ast}(S^n,X)\to\Map_{\ast}(S^n,Y)$ 
and hence homomorphisms $f_\ast\colon\pi_n(X)\to\pi_n(Y)$ 
for all $n$.
$f$ is called a \bfblue{weak homotopy equivalence} if 
$f_\ast$ is an isomorphism for all $n$ and 
all choices of base point. 
Two topological spaces are said to be 
\bfblue{weak homotopy equivalent}, or have the same 
\bfblue{(weak) homotopy type} if there is a zigzag 
of weak homotopy equivalences between them. 
By the homotopy hypothesis, if $f\colon X\to Y$ 
is weak homotopy equivalence, then the induces morphism 
$f_{\ast}\colon\Pi(X)\to\Pi(Y)$ must be an equivalent 
of $\infty$-groupoid, or an isomorphism in $\Hh$.

It is not difficult to show that homotopy equivalences are 
weak homotopy equivalences. However, the converse is not true. 
Therefore to get the correct $\infty$-category theory, 
the homotopy category $\Hh$ should be modified such that 
two topological spaces are weak homotopy equivalent if 
and only if they are isomorphic in $\Hh$.

One way to do this is to restrict $\Hh$ to a suitable subcategory 
such that: 
\begin{enumerate}[1)]
\item in this subcategory, every weak homotopy equivalence 
becomes an isomorphism;
\item every topological space is weak homotopy equivalent 
to an object in this subcategory.
\end{enumerate}
}

\para{
There is a special class of topological spaces called 
\bfblue{CW complexes}. For which we have 
\begin{description}
\item[Whitehead theorem] Every weak homotopy equivalence 
between CW complexes is a strong homotopy equivalence.
\item[CW approximation] Every topological space admits a 
weak homotopy equivalence from a CW complex to it.
\item[Cellular approximation] Every maps of CW complexes 
is homotopic to a cellular map, i.e. preserving the skeletons.
\end{description}

Thus, a good modification of $\Hh$ is to restrict it to 
the subcategory of CW complexes.

With this modification, we can built an $\infty$-category 
theory satisfying the homotopy hypothesis. 
To summary\footnote{
But there is still some pathological issue in this framework. 
A really workable definition needs to replace $\Top$ by a 
convenient category of topological spaces.
}:
\begin{enumerate}
\item \bfblue{The homotopy category $\Hh$} is the category 
of CW complexes whose morphisms are homotopy classes of 
maps between CW complexes. Furthermore, such a morphism 
can be presented by a cellular map.
\item Hence, an \bfblue{$\infty$-groupoid} is a CW complex and 
an \bfblue{$\infty$-category} is a category enriched over $\Hh$. 
This definition gives naturally a notion of 
\bfblue{homotopy category} of an $\infty$-category, 
which is the ordinary category obtained by apply the 
\emph{change of base categories} $\pi_0\colon\Hh\to\Set$.
\item The \bfblue{fundamental $\infty$-groupoid} of a topological 
space is then the CW approximation of it.
\end{enumerate}

The above version of $\infty$-category theory provided a 
good framework to study homotopy theory and has the advantage 
that it is pretty geometric. 
However, it also has some disadvantages: it is not algebraic 
enough for general application and the constructions in CW 
complex theory involves cumbersome and irrelevant choices. 
Another well-developed $\infty$-category theory can be find in 
\cite{HTT}. An axiomatic approach to $\infty$-category theory 
can be find in a book in progress \cite{ICWM}.
}

\para{
Leaving the general $\infty$-category theory aside, let's 
return to the homotopy theory of topological spaces. 
First of first, the category $\Top$ of topological spaces 
now can be viewed as an $\infty$-category. 
Note that, in our setting, the \bfblue{Hom space} from $X$ to $Y$ 
is not $\Map(X,Y)$, but its CW approximation. 
Let's denote it by $\HHom(X,Y)$.
}

\para{
A significant feature of $\infty$-category theory is it admits 
\bfblue{homotopy limits} and \bfblue{homotopy colimits}. 
To see the difference between those notions and 
\emph{limits/colimits}, 
let's consider a simple diagram: $\bullet\to\bullet$.
A digram of this shape in $\Top$ is just a continuous map 
$f\colon X\to Y$. It is easy to see that the limit (resp. colimit) 
of it is just $X$ (resp. $Y$). 

However, when consider homotopy limit of it, one looks at 
the category of \emph{homotopy triangle above} $f$.
An object of this category is a space $T$ (called the \emph{vertex})
together with a triangle 
\[
\begin{tikzcd}
T\ar[d]\ar[dr,""{name=U,below}]
&\\
X\ar[Rightarrow,from=U]\ar[r,"{f}"']&Y
\end{tikzcd}
\]
where the bold arrow denoted a homotopy. 
If $S\to T$ is a continuous map, then by composing it with a 
homotopy triangle above $f$ with vertex $T$, we obtain a homotopy 
triangle above $f$ with vertex $S$. A \emph{morphism} between 
homotopy triangles is such a continuous map.
Then the \emph{homotopy limit} of the diagram $X\markar{f}Y$
is the terminal object in this category.

To spell out the homotopy limit, we translate the homotopy triangles 
into usual commutative diagrams
\[
\begin{tikzcd}
& Y \\
T\ar[d]\ar[ur]\ar[r] & 
\Map(I,Y)\ar[u,"{\ev_0}"']\ar[d,"{\ev_1}"] \\
X\ar[r,"{f}"] & Y
\end{tikzcd}
\]
which is equivalent to the following diagram.
\[
\begin{tikzcd}
T\ar[d]\ar[r] & 
\Map(I,Y)\ar[d,"{\ev_1}"] \\
X\ar[r,"{f}"] & Y
\end{tikzcd}
\]
Therefore, the homotopy limit of the diagram $X\markar{f}Y$ is 
the pullback of $\ev_1\colon\Map(I,Y)\to Y$ along $f$. 
More concretely, it is the space
\[
Nf:=\big\{(x,\gamma)\in X\times\Map(I,Y):f(x)=\gamma(1)\big\}
\]
equipped with the subspace topology. 
This space is called the \bfblue{mapping path space} of $f$.
It is clear that $Nf$ is not homeomorphic to $X$ in general. 
However, they are homotopy equivalent.

The similar story happens to the dual situation, where the homotopy 
triangle is eventually translated into the following diagram.
\[
\begin{tikzcd}
X\ar[r,"{f}"]\ar[d,"{\delta_0}"'] & Y\ar[d]\\
X\times I\ar[r] & T \\
\end{tikzcd}
\]
Therefore, the homotopy colimit of the diagram $X\markar{f}Y$ is 
the pushout of $\delta_0\colon X\to X\times I$ along $f$.
More concretely, it is the quotient space
\[
\Cyl(f):=X\times I\coprod Y/\sim,
\]
where $\sim$ is generated by $(x,0)\sim f(x)$.
This space is called the \bfblue{mapping cylinder} of $f$.
It is clear that $\Cyl(f)$ is not homeomorphic to $X$ in general. 
However, they are homotopy equivalent.
}
\begin{Rem}
Note that in the above diagrams, one can invert the orientation of $I$, 
i.e. switch $\ev_0$ and $\ev_1$ (resp. $\delta_0$ and $\delta_1$), 
while the resulting space is homeomorphic to the one defined there.
\end{Rem}

\para{
However, the \emph{homotopy limits/colimits} are even not 
limits/colimits in the homotopy category. 
To see this, let's consider the diagram 
$\bullet\to\bullet\from\bullet$. 
A diagram of this shape in $\Top$ is a pair of continuous maps 
$X\markar{f}Y\markal{g}Z$. Then a \emph{homotopy square to} it 
is such a diagram
\[
\begin{tikzcd}
T\ar[d]\ar[r] & Z\ar[d,"{g}"]
\ar[dl,Rightarrow,dashed] \\
X\ar[r,"{f}"'] & Y
\end{tikzcd}
\]
where the dashed arrow denote a homotopy. 
Such a homotopy diagram is equivalent to the 
following commutative diagram.
\[
\begin{tikzcd}
Z\ar[r,"{g}"]& Y \\
T\ar[d]\ar[u]\ar[r] & 
\Map(I,Y)\ar[u,"{\ev_0}"']\ar[d,"{\ev_1}"] \\
X\ar[r,"{f}"] & Y
\end{tikzcd}
\]
Hence, the homotopy limit of the diagram $X\markar{f}Y\markal{g}Z$ 
is the fiber product of $Nf$ and $Ng$ over $\Map(I,Y)$, that is 
the space
\[
X\times_Y^hZ:=
\big\{(x,\gamma,z)\in X\times\Map(I,Y)\times Z:
f(x)=\gamma(1),g(z)=\gamma(0)\big\}.
\]
This space is called the \bfblue{homotopy fiber product}, 
or the \bfblue{homotopy pullback} of $g$ along $f$.

Dually, one can consider the digram $X\markal{f}Y\markar{g}Z$ 
and the homotopy colimit of it is the fiber coproduct of 
$\Cyl(f)$ and $\Cyl(g)$ under $Y\times I$, which is 
the quotient space
\[
X\amalg_Y^hZ:=X\coprod(Y\times I)\coprod Z/\sim,
\]
where $\sim$ is generated by $f(y)\sim(y,0)$ and $(y,1)\sim g(y)$. 
This is called the \bfblue{homotopy fiber coproduct}. the \bfblue{homotopy pushout} of $f$ along $g$.
}

\para{
Now, let's consider a special case of previous constructions: 
where $Z$ is the singleton $\pt$.
In this case, we can identify $X\times_Y^h\pt$ with the space
\[
\Fib(f):=
\big\{(x,\gamma)\in X\times\Map(I,Y):
f(x)=\gamma(1),\gamma(0)=\ast\big\},
\]
where $\ast$ is the image of $\pt$ in $Y$. 
This space is called the \bfblue{homotopy fiber} of $f$ 
at the point $\ast\in Y$. 
We can identify $X\amalg_Y^h\pt$ as the quotient space
\[
\Cofib(f):=X\coprod(Y\times I)/\sim,
\]
where $\sim$ is generated by $f(y)\sim(y,0)$ and $(y,1)\sim(y',1)$. 
This space is called the \bfblue{homotopy cofiber} of $f$, 
or the \bfblue{mapping cone} of $f$ with notation $Cf$.
}

\para{
Let's consider a even more special case: both $X$ and $Z$ are singleton 
$\pt$ and mapping to the same point $\ast$ of $Y$. 
In this case, we can surprisingly identify $\pt\times_Y^h\pt$ with 
the loop space $\Omega Y$ by viewing $Y$ as the pointed space with 
the base point $\ast$. It is clear that the loop space of a topological 
space is in general not contractible. 

Besides, we can identify $\pt\amalg_Y^h\pt$ as the quotient space
\[
SY:=Y\times I/\sim,
\]
where $\sim$ is generated by $(y,i)\sim(y',i)$ for $i=0,1$. 
This space is called the \bfblue{unreduced suspension} of $Y$. 
Let $Y=S^1$, it is clear that $SS^1=S^2$, which is not contractible. 
Note that if $Y$ is pointed as a base point $\ast$, then $SY$ admits 
a distinguish subspace $\{\ast\}\times I$ and the quotient by modulo 
this subspace is the pointed space $\Sigma Y$.
}

\para{
Recall that if $D\colon\Ii\to\Cc$ is a diagram in a category $\Cc$, 
then there are natural isomorphisms of sets
\begin{align*}
\Hom_{\Cc}(-,\lim D) &\cong \lim\Hom_{\Cc}(-,D), \\
\Hom_{\Cc}(\colim D,-) &\cong \lim\Hom_{\Cc}(D,-).
\end{align*}
Analogously, if $D\colon\Ii\to\Cc$ is a diagram in a $\infty$-category 
$\Cc$, then there should be natural equivalences of 
(functors to) $\infty$-groupoids\footnote{
However, the right-hand side is not a CW complex in general. 
Hence one needs to replace it by its CW approximation and makes 
the statements meaningful only for weak homotopy equivalences. 
Consequently, the notions of homotopy limits/colimits make sense 
only up to weak homotopy equivalences.
}
\begin{align*}
\HHom_{\Cc}(-,\holim D) &\simeq \holim\HHom_{\Cc}(-,D), \\
\HHom_{\Cc}(\hocolim D,-) &\simeq \holim\HHom_{\Cc}(D,-).
\end{align*}
Therefore, since we have worked out the homotopy limits/colimits 
of previous diagrams, we can make the following definitions in an 
arbitrary $\infty$-category $\Cc$.
\begin{proplist}
\item Let $f\colon X\to Y$ be a morphism in $\Cc$. 
Then a \bfblue{mapping path object} is an object $Nf$ inducing a 
natural equivalence of $\infty$-groupoids
\[
\HHom_{\Cc}(T,Nf) \simeq N\HHom_{\Cc}(T,f),
\]
where the continuous map 
$\HHom_{\Cc}(T,f)\colon\HHom_{\Cc}(T,X)\to\HHom_{\Cc}(T,Y)$ 
is given by composing with $f$, for each object $T$ of $\Cc$. 
Dually, a \bfblue{mapping cylinder object} is an object $\Cyl(f)$ 
inducing a 
natural equivalence of $\infty$-groupoids
\[
\HHom_{\Cc}\big(\Cyl(f),T\big) \simeq P\HHom_{\Cc}(f,T)
\]
for each object $T$ of $\Cc$. 
\item Let $X\markar{f}Y\markal{g}Z$ be two morphisms in $\Cc$. 
Then a \bfblue{homotopy fiber product} is a object $X\times_Y^hZ$ 
inducing a natural equivalence of $\infty$-groupoids
\[
\HHom_{\Cc}(T,X\times_Y^hZ) \simeq 
\HHom_{\Cc}(T,X)\times_{\HHom_{\Cc}(T,Y)}^h\HHom_{\Cc}(T,Z),
\]
where the right-hand side is obtained from the diagram
\[
\HHom_{\Cc}(T,X)\markar{f_{\ast}}
\HHom_{\Cc}(T,Y)\markal{g_{\ast}}\HHom_{\Cc}(T,Z),
\]
for each object $T$ of $\Cc$.
\item As special cases of previous, we have the notions of 
\bfblue{homotopy fiber} and \bfblue{loop space object} 
(also called \bfblue{looping}) in $\Cc$.
\item Let $X\markal{f}Y\markar{g}Z$ be two morphisms in $\Cc$. 
Then a \bfblue{homotopy fiber coproduct} is a object $X\amalg_Y^hZ$ 
inducing a natural equivalence of $\infty$-groupoids
\[
\HHom_{\Cc}(X\amalg_Y^hZ,T) \simeq 
\HHom_{\Cc}(X,T)\times_{\HHom_{\Cc}(Y,T)}^h\HHom_{\Cc}(Z,T),
\]
where the right-hand side is obtained from the diagram
\[
\HHom_{\Cc}(X,T)\markar{f^{\ast}}
\HHom_{\Cc}(Y,T)\markal{g^{\ast}}\HHom_{\Cc}(Z,T),
\]
for each object $T$ of $\Cc$.
\item As special cases of previous, we have the notions of 
\bfblue{homotopy cofiber} and \bfblue{suspension object} in $\Cc$.
\end{proplist}
}

\para{
%From now on, without specification otherwise, the base point of a 
%pointed space is denoted by $\ast$.
Apply the previous to 
the $\infty$-category $\Top_{\ast}$, we obtain the 
following constructions.
\begin{proplist}
\item The \bfblue{mapping path space} 
of a based map $f\colon (X,x_0)\to (Y,y_0)$ 
is the same space as $Nf$ with the 
base point $(x_0,\widetilde{y_0})$, 
where $\widetilde{y_0}$ is the constant path at $y_0$.
\item The \bfblue{(reduced) mapping cylinder} 
of a based map $f\colon (X,x_0)\to (Y,y_0)$ is the quotient space
\[
\Cyl(f):=X\times I\coprod Y/\sim,
\]
where $\sim$ is generated by $(x,0)\sim f(x)$ and 
$(x_0,t)\sim(x_0,t')$, 
with the base point the class of $(x_0,0)$.
\item The \bfblue{homotopy fiber product} 
of a pair of based maps $(X,x_0)\markar{f}(Y,y_0)\markal{g}(Z,z_0)$ is 
the same space as $X\times_Y^hZ$ with the base point 
$(x_0,\widetilde{y_0},z_0)$.
\item In particular, the \bfblue{homotopy fiber} of a based map 
$f\colon (X,x_0)\to (Y,y_0)$ is the same space as $\Fib(f)$ with 
the base point $(x_0,\widetilde{y_0})$.
\item In particular, the \bfblue{looping} of pointed space $(X,x_0)$ 
is the loop space $\Omega X$ with the based point the constant loop 
at $x_0$.
\item The \bfblue{(reduced) homotopy fiber coproduct} 
of based maps $(X,x_0)\markal{f}(Y,y_0)\markar{g}(Z,z_0)$ is 
the quotient space
\[
X\amalg_Y^hZ:=X\coprod(Y\times I)\coprod Z/\sim,
\]
where $\sim$ is generated by $f(y)\sim(y,0)$, $(y,1)\sim g(y)$ 
and $(y_0,t)\sim(y_0,t')$, with the base point the class of $(y_0,t)$.
\item In particular, the \bfblue{(reduced) homotopy cofiber} of a based map 
$f\colon (X,x_0)\to (Y,y_0)$ is the quotient space
\[
\Cofib(f):=X\coprod(Y\times I)/\sim,
\]
where $\sim$ is generated by $f(y)\sim(y,0)$, $(y,1)\sim(y',1)$ 
and $(y_0,t)\sim(y_0,t')$, with the base point the class of $(y_0,t)$.
\item In particular, the \bfblue{(reduced) suspension} 
of pointed space $(X,x_0)$ is the suspension $\Sigma X$.
\end{proplist}
}

\para{
Let $f\colon X\to Y$ be a map between topological spaces. 
The preimage $f^{-1}(y_0)$ of $y_0\in Y$ is called the 
\bfblue{fiber} of $X$ at the point $y_0$. 
Viewing $f$ as a based map by specifying $y_0$ as the base 
point of $Y$, the notion of fiber 
is similar to the notion of kernel: 
let $f\colon A\to B$ be a homomorphism between abelian groups, 
then the kernel is the preimage $f^{-1}(0)$.

Note that in the category of pointed spaces, the singleton 
$\pt$ is both an initial and terminal object, hence is 
a \emph{zero object}. 
Let $\mathcal{C}$ be a category having pullbacks and 
a zero object ${\bf0}$. 
For $f\colon A\to B$ a morphism in $\Cc$, its \bfblue{kernel} 
is the pullback of the zero morphism ${\bf0}\to B$ along $f$.
\[
\begin{tikzcd}
\Ker(f)\ar[d]\ar[r] & {\bf0}\ar[d] \\
A\ar[r,"{f}"] & B
\end{tikzcd}
\]
Dually, if $\mathcal{C}$ has pushouts, the \bfblue{cokernel} 
of $f$ is the pushout of the zero morphism $A\to{\bf0}$ along $f$. 
\[
\begin{tikzcd}
A\ar[d]\ar[r,"{f}"] & B\ar[d]\\
{\bf0}\ar[r] & \Coker(f)
\end{tikzcd}
\]
A sequence $A\markar{f}B\markar{g}C$ is called a 
\bfblue{left exact sequence} if $A$ is the kernel of $g$, a 
\bfblue{right exact sequence} if $C$ is the cokernel of $f$ 
and a \bfblue{short exact sequence} if both of previous are true.

In the category of pointed sets, or pointed spaces, we further have 
the notion of \emph{exact sequence}: 
a sequence $X\markar{f}Y\markar{g}Z$ is said to be \bfblue{exact}
at $Y$ if $\im(f)=\ker(g)$.
}

\para{
Let $\Cc$ be a category with terminal object $\pt$. 
Then the category under $\pt$ has a zero object $\pt\to\pt$. 
This category is denoted by \bfblue{$\Cc_{\ast}$}. 
An object $x_0\colon\pt\to X$ in $\Cc_{\ast}$ is called a 
\bfblue{pointed object} in $\Cc$, 
viewed as an object $X$ in $\Cc$ with the \bfblue{base point} $x_0$. 
A morphism in $\Cc_{\ast}$ is called a \bfblue{based morphism}. 

Suppose $\Cc$ has limits and colimits. Then we have the followings. 
\begin{proplist}
\item 
The forgetful functor sending each pointed object $(X,x_0)$ to $X$ has 
a left adjoint $+\colon\Cc\to\Cc_{\ast}$ sending each object $X$ to 
the pointed object $(X_{+},\ast)$, where $X_{+}$ is the coproduct of 
$X$ and $\pt$ and $\ast$ is the morphism $\pt\to X\amalg\pt$.
\item 
Therefore the limits of pointed objects can be computed in the category 
$\Cc$: it is precisely the limit together with the unique morphism 
obtained from the base points by the universal property.
\item 
Secondly, the colimits of pointed objects are obtained by apply the 
functor $+$ to the colimits of their underlying objects.
\item 
The coproduct of two pointed objects $X,Y$ is called the 
\bfblue{wedge sum} of them, denoted by $X\vee Y$. 
Clearly, there is canonical morphism $X\times Y\to X\vee Y$. 
The cokernel of this morphism is called the \bfblue{smash product} 
and denoted by $X\wedge Y$. 
\end{proplist}

Suppose $\Cc$ is further \emph{Cartesian closed}, i.e. the functor
$X\times-$ has a right adjoint $[X,-]$. 
\begin{proplist}[resume]
\item Then the smash product gives $\Cc_{\ast}$ a closed symmetric 
monoidal structure: the unit is $\pt_{+}$ and the internal Hom object 
$[X,Y]_{\ast}$ is obtained as the pullback of the morphism 
$\pt\to[\pt,Y]$ along $[X,Y]\to[\pt,Y]$ with the base point obtained 
from the morphism $\pt\to[X,Y]$ whose adjunct is the composition 
$\pt\times X\to\pt\to Y$.
\end{proplist}
}

\para{
Now, let $\Cc$ be a $\infty$-category having terminal object $\pt$. 
Then we can define the $\infty$-category $\Cc_{\ast}$ of 
pointed objects as previous. 
Suppose $\Cc$ has homotopy pullbacks and homotopy pushouts. 
A sequence $X\markar{f}Y\markar{g}Z$ of is called a 
\bfblue{fibration sequence} if $X$ is a homotopy fiber of $g$ and a 
\bfblue{cofibration sequence} if $Z$ is a homotopy cofiber of $f$. 
Unlike left/right exact sequences, fibration/cofibration sequences are 
automatically long.

Indeed, 
let $f\colon X\to Y$ be a based morphism of pointed objects in $\Cc$.
Then, we have the fibration sequence 
\[
\Fib(f)\markar{i}X\markar{f}Y.
\]
Consider the \emph{reversed} homotopy fiber $\bar\Fib(i)$ of $i$. 
To see what does this means and why we need this, 
look at the following diagram
\[
\begin{tikzcd}
\bar\Fib(i)\ar[r]\ar[d]&
\Fib(f)\ar[r]\ar[d,"{i}"']\ar[dl,Rightarrow,dashed]&
\pt\ar[d]\ar[dl,Rightarrow,dashed]\\
\pt\ar[r]&X\ar[r,"{f}"]&Y
\end{tikzcd}
\]
where the right square exhibits $\Fib(f)$ as the homotopy fiber of $f$ 
while the left square, 
instead of exhibiting $\bar\Fib(i)$ as the homotopy fiber of $i$ which 
is the homotopy pullback of $\pt\to X$ along $i$, 
exhibits $\bar\Fib(i)$ as the homotopy pullback of $i$ along 
$\pt\to X$. 
Note that, by pasting the two squares, the rectangle becomes a homotopy 
square and exhibits $\bar\Fib(i)$ as the homotopy pullback of 
$\pt\to Y$ along itself, i.e. the \emph{loop space object} $\Omega Y$. 
Note that, by our construction, 
the reversed homotopy fiber and the homotopy fiber are 
canonically isomorphic\footnote{
In fact, since the notions of homotopy limits only make sense 
up to weak homotopy equivalences, 
the statement here is literally wrong. 
However, it is true that the constructions of 
reversed homotopy fiber 
(which is given by just invert $I$ in the construction of 
the homotopy fiber) 
and the homotopy fiber given in 
$\Top$ and $\Top_{\ast}$ are canonically homeomorphic.}.
Therefore we have anther fibration sequence
\[
\Omega Y\To\Fib(f)\markar{i}X.
\]
If we keep going, obtaining the following diagram
\[
\begin{tikzcd}
\Omega X\ar[r]\ar[d,"{-\Omega f}"']&
\pt\ar[d]\ar[dl,Rightarrow,dashed]&\\
\Omega Y\ar[r]\ar[d]&
\Fib(f)\ar[r]\ar[d,"{i}"']\ar[dl,Rightarrow,dashed]&
\pt\ar[d]\ar[dl,Rightarrow,dashed]\\
\pt\ar[r]&X\ar[r,"{f}"]&Y
\end{tikzcd}
\]
where the $-\Omega f$ denotes the \emph{reversed} loop morphism. 
The reversion appear due to the reversed homotopy in the left-below 
square.

Therefore, if we have a fibration sequence $X\markar{f}Y\markar{g}Z$, 
then we have a \emph{long fibration sequence}
\[
\cdots\To
\Omega X\markar{-\Omega f}\Omega Y\markar{-\Omega g}\Omega Z
\To X\markar{f}Y\markar{g}Z.
\]

The similar story applies to cofibration sequences. 
If we have a cofibration sequence $X\markar{f}Y\markar{g}Z$, 
then we have a \emph{long cofibration sequence}
\[
X\markar{f}Y\markar{g}Z\To
\Sigma X\markar{-\Sigma f}\Sigma Y\markar{-\Sigma g}\Sigma Z
\To\cdots.
\]

}

\para{
Let $f\colon X\to Y$ be a morphism in $\Cc_{\ast}$. 
The adjunction of $\Sigma$ and $\Omega$ gives rise to the 
following commutative diagram.
\[
\adjustbox{scale=0.9,center}{
\begin{tikzcd}
& \Sigma\Omega\Fib(f)\ar[r]\ar[d] & 
\Sigma\Omega X\ar[r]\ar[d] & 
\Sigma\Omega Y\ar[r]\ar[d] & 
\Sigma\Omega\Cofib(f)\ar[r]\ar[d] & 
\Sigma\Omega\Sigma X\ar[d] \\
\Omega Y\ar[r]\ar[d] & 
{\Fib(f)}\ar[r]\ar[d] & 
X\ar[r]\ar[d] & 
Y\ar[r]\ar[d] & 
{\Cofib(f)}\ar[r]\ar[d] & 
{\Sigma X} \\
\Omega\Sigma\Omega Y\ar[r] & 
\Omega\Sigma\Fib(f)\ar[r] & 
\Omega\Sigma X\ar[r] & 
\Omega\Sigma Y\ar[r] & 
\Omega\Sigma\Cofib(f) &
\end{tikzcd}
}
\]
Considering the following homotopy commutative diagram:
\[
\begin{tikzcd}
\Fib(f)\ar[r]\ar[d]&
\pt\ar[d]\ar[dl,Rightarrow,dashed]\\
X\ar[r]\ar[d]&
Y\ar[d]\ar[dl,Rightarrow,dashed]\\
\pt\ar[r]&
\Cofib(f)
\end{tikzcd}
\]
one see that there are homotopy equivalence:
\[
\Fib(f)\markar{\sim}\Omega\Cofib(f),\qquad
\Sigma\Fib(f)\markar{\sim}\Cofib(f).
\]
Together with the triangle identities for the $\Sigma\dashv\Omega$, 
we obtain the following commutative diagram
\[
\adjustbox{scale=0.9,center}{
\begin{tikzcd}
& \Sigma\Omega\Fib(f)\ar[r]\ar[d] & 
\Sigma\Omega X\ar[r]\ar[d] & 
\Sigma\Omega Y\ar[r]\ar[d] & 
\Sigma\Fib(f)\ar[r]\ar[d] & 
\Sigma X\ar[d,equal] \\
\Omega Y\ar[r]\ar[d,equal] & 
{\Fib(f)}\ar[r]\ar[d] & 
X\ar[r]\ar[d] & 
Y\ar[r]\ar[d] & 
{\Cofib(f)}\ar[r]\ar[d] & 
{\Sigma X} \\
\Omega Y\ar[r] & 
\Omega\Cofib(f)\ar[r] & 
\Omega\Sigma X\ar[r] & 
\Omega\Sigma Y\ar[r] & 
\Omega\Sigma\Cofib(f) &
\end{tikzcd}
}
\]
where the top row is the suspension of a fiber sequence 
and the bottom row is the looping of a cofiber sequence.
}

\para{
It turns out that the functor 
$[Z,-]_{\ast}\colon\Top_{\ast}\to\Set_{\ast}$ 
is left exact for any pointed space $Z$. 
In particular, $\pi_0$ is left exact. 
So, if we have a fiber sequence of pointed spaces
\[
\cdots\To\Omega^2 Z\To
\Omega X\markar{-\Omega f}\Omega Y\markar{-\Omega g}\Omega Z
\To X\markar{f}Y\markar{g}Z.
\]
Notice that $\pi_0(\Omega^nX)=\pi_n(X)$.
Then we get a long exact sequence of pointed sets
\begin{multline*}
\cdots\To
\pi_2(X)\markar{f_{\ast}}\pi_2(Y)\markar{g_{\ast}}\pi_2(Z)\To \\
\pi_1(X)\markar{f_{\ast}}\pi_1(Y)\markar{g_{\ast}}\pi_1(Z)
\To \pi_0(X)\markar{f_{\ast}}\pi_0(Y)\markar{g_{\ast}}\pi_0(Z).
\end{multline*}
Moreover, since $\pi_0$ is left exact, the above maps 
preserve group structures if there exists one.

For $\Cc$ an $\infty$-category and $C$ any object in $\Cc_{\ast}$, 
the functor $\HHom_{\Cc_{\ast}}(C,-)$ is left exact, i.e. preserves 
homotopy limits. 
Hence, if we have a fiber sequence of pointed objects
\[
\cdots\To\Omega^2 Z\To
\Omega X\markar{-\Omega f}\Omega Y\markar{-\Omega g}\Omega Z
\To X\markar{f}Y\markar{g}Z.
\]
Then we get a fiber sequence of pointed spaces
\begin{multline*}
\cdots\To\HHom_{\Cc_{\ast}}(C,\Omega^2 Z)\To 
\HHom_{\Cc_{\ast}}(C,\Omega X)\markar{f_{\ast}}
\HHom_{\Cc_{\ast}}(C,\Omega Y)\markar{g_{\ast}}\\
\HHom_{\Cc_{\ast}}(C,\Omega Z)
\To \HHom_{\Cc_{\ast}}(C,X)\markar{f_{\ast}}
\HHom_{\Cc_{\ast}}(C,Y)\markar{g_{\ast}}
\HHom_{\Cc_{\ast}}(C,Z),
\end{multline*}
and thus a long exact sequence of pointed sets
\begin{multline*}
\cdots\To\pi_0\HHom_{\Cc_{\ast}}(C,\Omega^2 Z)\To\\
\pi_0\HHom_{\Cc_{\ast}}(C,\Omega X)\markar{f_{\ast}}
\pi_0\HHom_{\Cc_{\ast}}(C,\Omega Y)\markar{g_{\ast}}
\pi_0\HHom_{\Cc_{\ast}}(C,\Omega Z)\\
\To\pi_0\HHom_{\Cc_{\ast}}(C,X)\markar{f_{\ast}}
\pi_0\HHom_{\Cc_{\ast}}(C,Y)\markar{g_{\ast}}
\pi_0\HHom_{\Cc_{\ast}}(C,Z),
\end{multline*}
where the maps preserve (possibly exist) group structures. 
To simplify notation, denote $\pi_0\HHom_{\Cc_{\ast}}(-,-)$ by 
$\<-,-\>$ if there is no ambiguity.
Dually, the functor $\HHom_{\Cc_{\ast}}(-,C)$ 
sends homotopy colimits to homotopy limits. 
Hence, if we have a cofiber sequence of pointed objects
\[
X\markar{f}Y\markar{g}Z\To
\Sigma X\markar{-\Sigma f}\Sigma Y\markar{-\Sigma g}\Sigma Z
\To\Sigma^2 X\To\cdots.
\]
Then we get a fiber sequence of pointed spaces
\begin{multline*}
\cdots\To\HHom_{\Cc_{\ast}}(\Sigma^2 X,C)\To 
\HHom_{\Cc_{\ast}}(\Sigma Z,C)\markar{g^{\ast}}
\HHom_{\Cc_{\ast}}(\Sigma Y,C)\markar{f^{\ast}}\\
\HHom_{\Cc_{\ast}}(\Sigma X,C)
\To \HHom_{\Cc_{\ast}}(Z,C)\markar{g^{\ast}}
\HHom_{\Cc_{\ast}}(Y,C)\markar{f^{\ast}}
\HHom_{\Cc_{\ast}}(X,C),
\end{multline*}
and thus a long exact sequence of pointed sets
\begin{multline*}
\cdots\To
\<\Sigma^2 X,C\>\To
\<\Sigma Z,C\>\markar{g^{\ast}}
\<\Sigma Y,C\>\markar{f^{\ast}}\\
\<\Sigma X,C\>\To
\<Z,C\>\markar{g^{\ast}}
\<Y,C\>\markar{f^{\ast}}
\<X,C\>,
\end{multline*}
where the maps preserve (possibly exist) group structures.
The above two exact sequences are related by the following
identification of pointed sets
\[
\pi_0\HHom_{\Cc_{\ast}}(\Sigma^n X,Y)=
\pi_0\HHom_{\Cc_{\ast}}(X,\Omega^n Y)=
\pi_n\HHom_{\Cc_{\ast}}(X,Y).
\]

}

\para{
Let $\Cc$ be an $\infty$-category. 
An \bfblue{$\Omega$-spectrum} $\EE$ is 
a sequence of pointed objects $\{E_n\}_{n\in\NN}$ 
together with weak equivalences $f_n\colon E_n\to\Omega E_{n+1}$. 
Here $f_n$ is a weak equivalence means 
$f_{n\ast}\colon 
\pi_n\HHom_{\Cc_{\ast}}(C,E_n)\to
\pi_n\HHom_{\Cc_{\ast}}(C,\Omega E_{n+1})$ 
are isomorphisms for all $n$ and pointed object $C$.

Then, once we have a cofibration sequence $X\to Y\to Z$, we have 
long exact sequences of pointed sets
\begin{multline*}
\cdots\To
\<\Sigma Z,E_n\>\To
\<\Sigma Y,E_n\>\To
\<\Sigma X,E_n\>\\
\To
\<Z,E_n\>\To
\<Y,E_n\>\To
\<X,E_n\>,
\end{multline*}
for all $n$. 
By the adjunction $\Sigma\dashv\Omega$ and the definition of 
$\Omega$-spectrum, we deduce a long exact sequence of abelian groups
\begin{multline*}
\cdots\To
\<Z,E_{n-1}\>\To
\<Y,E_{n-1}\>\To
\<X,E_{n-1}\>\\
\To
\<Z,E_n\>\To
\<Y,E_n\>\To
\<X,E_n\>\To\\
\<Z,E_{n+1}\>\To
\<Y,E_{n+1}\>\To
\<X,E_{n+1}\>\To\cdots.
\end{multline*}
Let $H^n(X,\EE)$ denote $\<X,E_n\>$. Then using above discussion, 
it is easy to show that $H^n(-,\EE)$ defines a generalized cohomology 
theory, i.e it satisfies analogy of Eilenberg-Steenrod axioms.
This functor is called the \bfblue{intrinsic cohomology} with 
coefficient $\EE$.

Note that at the beginning of the long exact sequence we have
\[
H^0(Z,\EE)\To H^0(Y,\EE)\To H^0(X,\EE)\To\cdots
\]
but that not all, we further have
\begin{multline*}
\cdots\To
\<\Sigma Z,E_0\>\To
\<\Sigma Y,E_0\>\To
\<\Sigma X,E_0\>\\
\To
H^0(Z,\EE)\To H^0(Y,\EE)\To H^0(X,\EE).
\end{multline*}
Note that 
\[
\<\Sigma^n(-),E_0\> = 
\pi_n\HHom_{\Cc_{\ast}}(-,E_0).
\]
Hence we conclude that if we want to extend the intrinsic cohomology 
to negative degrees so that we have long exact sequence tending to both 
directions, then we have to put
\[
H^{-n}(-,\EE) = \pi_n\HHom_{\Cc_{\ast}}(-,E_0).
\]
In other words, \emph{negative cohomology groups are homotopy groups}.
}









\clearpage
\section{Chain complexes}
\para{
Let $I$ be a set and $\Cc$ a category. 
An \bfblue{$I$-graded object} in $\Cc$ is a functor from $I$, 
viewed as a discrete category, to $\Cc$. 
Hence the category of $I$-graded objects is denoted by $\Cc^{I}$. 
In plain words, an $I$-graded object is 
a family of objects $\{X_i\}_{i\in I}$ in $\Cc$ indexed by $I$. 
We denote it by $X_{\bullet}$ or simply $X$ if there is no ambiguity. 
A $\ZZ$-graded object is simply called a \bfblue{graded object} 
and the category $\Cc^{\ZZ}$ will be denoted by $\Gr(\Cc)$. 
A \bfblue{morphism} between $I$-graded objects $f\colon X\to Y$ is thus 
a family of morphisms $\{f\colon X_i\to Y_i\}_{i\in I}$ 
in $\Cc$ indexed by $I$. In other words, 
\[
\Hom_{\Cc^I}(X,Y)=\prod_{i\in I}\Hom_{\Cc}(X_i,Y_i).
\]
Let $\iota\colon\Cc\to\Cc^I$ be the functor sending each object 
$Y$ to the $I$-graded object $\underline{Y}$ whose each degree is $Y$. 
Then we have a functor 
\[
\Hom_{\Cc^I}(X,\iota)\colon\Cc\To\Set.
\]
Suppose $\Cc$ has direct sums, then 
the above functor can be represented by the direct sum
\[
\bigoplus_{i\in I}X_i.
\]
We call it the representative of $X$ and denoted also by $X$.
}

\para{
Now, suppose $G$ is a commutative monoid. 
Let $X$ be a $G$-graded object and $g$ an element of $G$. 
The \bfblue{$g$-twisted object} of $X$ is the $G$-graded object 
$X(g)$ defined as
\[
X(g)_u:=X_{g+u},\quad\forall u\in G.
\]
Let $X, Y$ be two $G$-graded objects. 
A morphism from $X$ to $Y(g)$ is called a
\bfblue{$g$-twisted morphism} from $X$ to $Y$. 
The $0$-twisted morphisms are the usual morphisms can called 
\bfblue{homogeneous morphisms}.
The $G$-graded set defined by
\[
\Hom(X,Y)_g := \Hom_{\Cc^G}(X,Y(g))
\]
is called the \bfblue{$G$-graded Hom}.
}

\para{
Now, suppose $\Aa$ is an \emph{abelian tensor category}. 
For $A,B$ two $G$-graded objects in $\Aa$, 
their \bfblue{tensor product} is defined by
\[
(A\otimes B)_{g} := \bigoplus_{u+v = g}(A_u\otimes B_v),
\quad\forall g\in G.
\]
In this way, $\Aa^G$ becomes an abelian tensor category. 
If furthermore $\Aa$ is \emph{closed}, admitting internal Hom 
bifunctor $[-,-]\colon\Aa^{\opp}\times\Aa\to\Aa$. 
Then $\Aa^G$ can be viewed as a $\Aa$-enriched category by setting the \emph{Hom-object} as
\[
\inHom_{\Aa^G}(A,B) := \prod_{g\in G}[A_g,B_g].
\]
Moreover, we define the \bfblue{internal $G$-graded Hom-object} by
\[
[A,B]_g := \inHom_{\Aa^G}(A,B(g)).
\]
The internal $G$-graded Hom-objects turn to be the 
\emph{internal Hom-objects} in $\Aa^G$ and we have the following 
(enriched) adjunctions:
\begin{align*}
\Hom_{\Cc^G}(A\otimes B,C) & \cong \Hom_{\Cc^G}(A,[B,C]), \\
\inHom_{\Cc^G}(A\otimes B,C) & \cong \inHom_{\Cc^G}(A,[B,C]), \\
[A\otimes B,C] & \cong [A,[B,C]].
\end{align*}
(However, to prove the above statements, one needs to deal with 
$\Aa^{G}$-enrichment first and then apply the obverse 
\emph{change of base categories} $\Aa^{G}\to\Aa$.)
}

\para{
Let $\Cc$ be a category admitting a \emph{zero object} $0$. 
\begin{proplist}
\item 
A \bfblue{chain complex} in $\Cc$ is 
a graded object endowed with a $(-1)$-twisted endomorphism $\partial$, 
called the \bfblue{boundary operator} or \bfblue{codifferential},
such that $\partial\circ\partial=0$.
We use the notation $X_{\bullet}$ to indicate it is 
a chain complex.
\item 
Dually, a \bfblue{cochain complex} in $\Cc$ is 
a graded object endowed with a $1$-twisted endomorphism $\di$, 
called the \bfblue{differential} or \bfblue{coboundary operator},
such that $\di\circ\di=0$.
We use the notation $X^{\bullet}$ to indicate it is 
a cochain complex.
\item 
Let $X_{\bullet}$, $Y_{\bullet}$ be two chain complexes. 
A \bfblue{chain morphism} 
$f\colon X_{\bullet}\to Y_{\bullet}$ between them
is a homogeneous morphism such that the following diagrams commute.
\[
\begin{tikzcd}
\cdots\ar[r] & X_n\ar[r,"{\partial_n}"]\ar[d,"{f_n}"] &
X_{n-1}\ar[d,"{f_{n-1}}"]\ar[r] & \cdots \\
\cdots\ar[r] & Y_n\ar[r,"{\partial_n}"] & Y_{n-1} \ar[r] & \cdots
\end{tikzcd}
\]
\item 
Dually, let $X^{\bullet},Y^{\bullet}$ be two cochain complexes. 
A \bfblue{cochain morphism} 
$f\colon X^{\bullet}\to Y^{\bullet}$ between them 
is a homogeneous morphism such that the following diagrams commute.
\[
\begin{tikzcd}
\cdots\ar[r] & X^n\ar[r,"{\di^n}"]\ar[d,"{f^n}"] &
X^{n+1}\ar[d,"{f^{n+1}}"]\ar[r] & \cdots \\
\cdots\ar[r] & Y^n\ar[r,"{\di^n}"] & Y^{n+1} \ar[r] & \cdots
\end{tikzcd}	
\]
\end{proplist}

The category of chain complexes (resp. cochain complexes) in $\Cc$ 
with chain morphisms (resp. cochain morphisms) between them is 
denoted by $\Ch_{\ast}(\Cc)$ (resp. $\Ch^{\ast}(\Cc)$).
Note that this category also has a zero object $\underline{0}$
whose each degree is $0$.
}

\para{
A chain complex $X_{\bullet}$ is said to be 
\begin{itemize}
\item 
\bfblue{connective} if $X_n=0$ for all $n<0$;
\item 
\bfblue{coconnective} if $X_n=0$ for all $n>0$;
\item 
\bfblue{bounded above} if $X_n=0$ for sufficiently large $n$;
\item 
\bfblue{bounded below} if $X_n=0$ for sufficiently small $n$;
\item 
\bfblue{bounded} if it is both bounded above and bounded below.
\end{itemize}
The full subcategory of $\Ch_{\ast}(\Cc)$ 
spanned by connective 
(resp. coconnective, 
bounded above, 
bounded below, 
bounded) chain complexes 
is denoted by 
$\Ch_c(\Cc)$ or $\Ch_{\ge0}(\Cc)$ 
(resp. $\Ch_{\le0}(\Cc)$, 
$\Ch_-(\Cc)$, 
$\Ch_+(\Cc)$, 
$\Ch_b(\Cc)$).

Dually, a cochain complex $X^{\bullet}$ is said to be 
\begin{itemize}
\item 
\bfblue{coconnective} if $X^n=0$ for all $n<0$;
\item 
\bfblue{connective} if $X^n=0$ for all $n>0$;
\item 
\bfblue{bounded above} if $X^n=0$ for sufficiently large $n$;
\item 
\bfblue{bounded below} if $X^n=0$ for sufficiently small $n$;
\item 
\bfblue{bounded} if it is both bounded above and bounded below.
\end{itemize}
The full subcategory of $\Ch^{\ast}(\Cc)$ 
spanned by connective 
(resp. coconnective, 
bounded above, 
bounded below, 
bounded) chain complexes 
is denoted by 
$\Ch^c(\Cc)$ or $\Ch^{\le0}(\Cc)$ 
(resp. $\Ch^{\ge0}(\Cc)$, 
$\Ch^-(\Cc)$, 
$\Ch^+(\Cc)$, 
$\Ch^b(\Cc)$).
}

\para{
Any chain complex $X_{\bullet}$ can be transformed into  
a cochain complex by 
\[
X^n:=X_{-n},\qquad
\di^n:=\partial_{-n}
\]
and \emph{vice versa}. 
Thus we can identify the following two categories
\[
\Ch_{\ast}(\Cc) \cong \Ch^{\ast}(\Cc)
\]
and safely use the notation \bfblue{$\Ch(\Cc)$} 
instead of $\Ch_{\ast}(\Cc)$ or $\Ch^{\ast}(\Cc)$ 
to denote those categories.
In this sense, we can safely use the terminology \bfblue{complex} 
to indicate both chain complexes and cochain complexes, 
and \bfblue{morphism of complexes} to indicate
both chain morphisms and cochain morphisms.

On the other hand, one can see that chain complexes in $\Cc$ 
are the same as cochain complexes in $\Cc^{\opp}$, hence
\[
\Ch_{\ast}(\Cc)^{\opp} = \Ch^{\ast}(\Cc^{\opp}).
\]
So we can canonically identify $\Ch(\Cc^{\opp})$ and $\Ch(\Cc)^{\opp}$.

Restricting the full subcategories mentioned before, 
we have the following natural isomorphisms
\begin{align*}
\Ch_{\ge0}(\Cc)^{\opp} &= \Ch^{\ge0}(\Cc^{\opp}) \cong \Ch_{\le0}(\Cc^{\opp}),
\\
\Ch_{\le0}(\Cc)^{\opp} &= \Ch^{\le0}(\Cc^{\opp}) \cong \Ch^{\ge0}(\Cc^{\opp}).
\end{align*}
Therefore, we can identify 
connective (resp. coconnective) chain complexes with 
connective (resp. coconnective) cochain complexes 
and call simply call them 
\emph{connective (resp. coconnective) complexes}. 
In practice, the terminology 
\bfblue{connective complexes} 
often refers to connective chain complexes 
while \bfblue{coconnective complexes} 
to coconnective cochain complexes.

We also have the following natural isomorphisms
\begin{align*}
\Ch_-(\Cc)^{\opp} &= \Ch^-(\Cc^{\opp}) \cong \Ch_+(\Cc^{\opp}),
\\
\Ch_+(\Cc)^{\opp} &= \Ch^+(\Cc^{\opp}) \cong \Ch_-(\Cc^{\opp}),
\\
\Ch_b(\Cc)^{\opp} &= \Ch^b(\Cc^{\opp}) \cong \Ch_b(\Cc^{\opp}).
\end{align*}
Hence, we can identify 
bounded above (resp. bounded below) chain complexes with 
bounded below (resp. bounded above) cochain complexes.
In this sense bounded above and bounded below chain complexes are 
dual notions while the notion of bounded complexes is self-dual.

We say a complex $X_{\bullet}$ is \bfblue{concentrated} 
at degree $n_1,\cdots,n_k$ if $X_i=0$ unless $i=n_1,\cdots,n_k$. 
It is clear that concentrated complexes are bounded complexes 
and \emph{vice versa}.
}

\para{
There are many ways to embed $\Cc$ into the category $\Ch(\Cc)$. 
Let $X$ be an object in $\Cc$. 
\begin{proplist}
\item 
The complex $\underline{X}_{\bullet}$ has $X$ at its every degree 
and $0$ as its boundary operator.
\item 
The complex $X[n]$ concentrated at degree $-n$ with component $X$. 
\item 
We simply denote $X[0]$ by $X$ if there is no ambiguity.
\end{proplist}
The notation $X[n]$ suggests that this complex is obtained by apply 
a \bfblue{translation of degree $n$} functor to the complex $X$. 

In the case $\Cc$ is an additive category, 
the functor $[n]$ is defined as follows. 
Let $X_{\bullet}$ be a complex. 
Then the complex $X[n]_{\bullet}$ is defined by
\[
X[n]_i:=X_{n+i},\qquad
\partial_{X[n]}:=(-1)^n\partial_{X},\qquad
\forall i\in\ZZ.
\]
Let $f$ be a chain morphism. 
Then the chain morphism $f[n]$ is defined by 
$f[n]_i=f_{n+i}$ for all $i\in\ZZ$.
}
\begin{Rem}
Note that the functor $[n]$ on $\Ch^{\ast}(\Cc)$ is 
usually defined by
\[
X[n]^i:=X^{n+i},\qquad
\di^{X[n]}:=(-1)^n\di^{X},\qquad
\forall i\in\ZZ.
\]
Under this setting, we encounter that
\[
X[n]^{\bullet} \neq X[n]_{-\bullet},
\marginnote{\dbend}
\]
which goes against our identification! 

One should rather think the functor $[n]$ as an 
extension of $(n)$ from the category of graded objects 
(which can be viewed as complexes with zero differentials) 
to the category of complexes. 
So $X[n]$ is not a complex unless we specify it is 
a chain complex or cochain complex.
\end{Rem}

\para{
When $\Cc=\Ab$, the category of abelian groups, 
we simply denote $\Ch(\Ab)$ by $\Ch$. 
More generally, let $k$ be a ring and 
$\Cc=k\Mod$, the category of $k$-modules, 
we simply denote $\Ch(k\Mod)$ by $\Ch(k)$. 
The notations for subcategories $\Ch_{?}$ and $\Ch^{?}$ 
($?$ equals $c,\ge0,\le0,+,-,b$) are similar.
}

\para{
From now on, let $\Aa$ be an abelian category. 
When $\Aa$ is $\Ab$ or $k\Mod$, 
we can talk about \emph{elements} of an object. 
For general abelian tensor category, 
a \bfblue{global element} of an object 
refers to a morphism from the unit to it, 
and a \bfblue{(general) element} refers to a morphism from 
arbitrary object.

Let $(C_{\bullet},\partial)$ be a chain complex in $\Aa$.
\begin{proplist}
\item 
The $n$-th \bfblue{cycle object} of $C_{\bullet}$ 
is $Z_n(C) := \Ker\partial_n$, 
whose elements are called \bfblue{$n$-cycles}.
\item 
The $n$-th \bfblue{boundary object} of $C_{\bullet}$ 
is $B_n(C) := \Image\partial_{n+1}$, 
whose elements are called \bfblue{$n$-boundaries}.
\end{proplist}
Since $\partial\circ\partial = 0$, 
the inclusion $B_n(C)\mono C_n$ factors through $Z_n(C)$. 
\begin{proplist}[resume]
\item 
The cokernel of the resulted inclusion $B_n(C)\mono Z_n(C)$ 
is called the $n$-th \bfblue{homology object} of $C_{\bullet}$ 
and denoted by $H_n(C)$. 
The elements of $H_n(C)$ are called \bfblue{homology classes}. 
\end{proplist} 

Dually, let $(C^{\bullet},\di)$ be a cochain complex in $\Aa$.
\begin{proplist}[resume]
\item 
The $n$-th \bfblue{cocycle object} of $C^{\bullet}$ 
is $Z^n(C) := \Ker\di_n$, 
whose elements are called \bfblue{$n$-cocycles}.
\item 
The $n$-th \bfblue{coboundary object} of $C^{\bullet}$ 
is $B^n(C) := \Image\di_{n-1}$, 
whose elements are called \bfblue{$n$-coboundaries}.
\end{proplist}
Since $\di\circ\di = 0$, 
the inclusion $B^n(C)\mono C^n$ factors through $Z^n(C)$. 
\begin{proplist}[resume]
\item 
The cokernel of the resulted inclusion $B^n(C)\mono Z^n(C)$ 
is called the $n$-th \bfblue{cohomology object} 
of $C^{\bullet}$ 
and denoted by $H^n(C)$. 
The elements of $H^n(C)$ are called \bfblue{cohomology classes}. 
\end{proplist} 
The above constructions extend to the following additive functors
\begin{align*}
Z_{\bullet}, B_{\bullet}, H_{\bullet}\colon&
\Ch_{\ast}(\Aa)\To\Aa^{\ZZ}, \\
Z^{\bullet}, B^{\bullet}, H^{\bullet}\colon&
\Ch^{\ast}(\Aa)\To\Aa^{\ZZ}.
\end{align*}
In particular, any chain morphism 
$f\colon C_{\bullet}\to D_{\bullet}$ 
(resp. cochain morphism $f\colon C^{\bullet}\to D^{\bullet}$)
induces a homogeneous morphism 
\[
H(f)\colon H_{\bullet}(C)\to H_{\bullet}(D).\qquad
(\text{resp. }H(f)\colon H^{\bullet}(C)\to H^{\bullet}(D))
\]
Obviously, if $f$ is an isomorphism, then so is $H(f)$. But the converse may not be true.
A chain morphism (resp. cochain morphism) $f$ is called a \bfblue{quasi-isomorphism} if 
$H(f)$ is an isomorphism. 
A chain complex $C_{\bullet}$ (resp. cochain complex $C^{\bullet}$) 
is said to be \bfblue{acyclic} if 
it is \emph{quasi-isomorphic} to $0$.
}

\para{
Since complexes is a special kind of diagrams, 
the limits and colimits in $\Ch(\Aa)$ are computed degree-wisely. 
Note that 
filtered colimits commute with finite limits and all colimits, 
hence by the construction of the functors 
$B_{\bullet}, Z_{\bullet}$ and $H_{\bullet}$ 
(resp. $B^{\bullet}, Z^{\bullet}$ and $H^{\bullet}$), 
they preserve filtered colimits.
}

\para{
Suppose $\Aa$ is an abelian tensor category. 
Let $C_{\bullet}$, $D_{\bullet}$ be two complexes. 
Then there exists a natural boundary operator $\partial$ on 
the tensor product $(C\otimes D)_{\bullet}$ of their underlying 
graded objects. 
The resulted complex is called the \bfblue{Koszul product} of 
$C_{\bullet}$ and $D_{\bullet}$. 
By its construction, we only need to define the following morphisms
\[
C_{p}\otimes D_{q} \markar{\partial_{p,q}^{(1)}} C_{p-1}\otimes D_{q},\qquad
C_{p}\otimes D_{q} \markar{\partial_{p,q}^{(2)}} C_{p}\otimes D_{q-1}.
\]
Note that the condition $\partial\circ\partial = 0$ requires that the following two morphisms must be negative to each other.
\begin{align*}
C_{p}\otimes D_{q} &\markar{\partial_{p,q}^{(2)}} C_{p}\otimes D_{q-1} \markar{\partial_{p,q-1}^{(1)}} C_{p-1}\otimes D_{q-1},\\
C_{p}\otimes D_{q} &\markar{\partial_{p,q}^{(1)}} C_{p-1}\otimes D_{q} \markar{\partial_{p-1,q}^{(2)}} C_{p-1}\otimes D_{q-1}.
\end{align*}
The common convention is
\[
\partial_{p,q}^{(1)} := \partial_{p}\otimes\id_{D_q},\qquad
\partial_{p,q}^{(2)} := (-1)^{p}\id_{C_p}\otimes\partial_{q}.
\]
In element notation, it reads
\[
\partial(x\otimes y) = 
\partial x\otimes y + (-1)^{|x|}x\otimes\partial y,
\]
where $|x|$ denotes the degree of $x$.
Then one can verify that the above construction makes $\Ch(\Aa)$ 
into an abelian tensor category with the unit $\unit$,  
which is $\unit[0]_{\bullet}$ with $\unit$ the unit of $\Aa$, 
and with the non-trivial braiding 
$\gamma(C,D)_{\bullet}\colon
(C\otimes D)_{\bullet}\to (D\otimes C)_{\bullet}$ 
whose component in each degree is 
\[
(-1)^{pq}\gamma(C_p,D_q)\colon C_p\otimes D_q\To D_q\otimes C_p,
\]
where $\gamma$ is the braiding in $\Aa$.
}
\begin{Rem}
One can see that $C[n]_{\bullet}$ 
is precisely $(\unit[n]\otimes C)_{\bullet}$.
This could be a reason why one may dislike the common convention. 
However, if we use $(C\otimes D)_{\bullet}$ 
to denote what usually means $(D\otimes C)_{\bullet}$, 
then (using the element notation) the boundary operator reads as 
\[
\partial(x\otimes y) = 
(-1)^{|y|}\partial x\otimes y + x\otimes\partial y.
\]
In a middle way, we use the notation 
\bfblue{$(C\otimes^{\gamma}D)_{\bullet}$} 
to denote $(D\otimes C)_{\bullet}$.
To illustrate how the braiding 
$(C\otimes D)_{\bullet}\to (C\otimes^{\gamma}D)_{\bullet}$ 
works, let's accept the following formal rule for element notation
\[
\blue
x\otimes^{\gamma}y:=\gamma(x\otimes y)=(-1)^{|x||y|}y\otimes x,
\]
It is often the case that elements of $C\otimes D$ are written as $xy$. 
If this is that case, elements of $C\otimes^{\gamma}D$ can be 
written as $x^{\gamma}y$ and the rule above reads 
\[
x^{\gamma}y=(-1)^{|x||y|}yx.
\]

Since the two tensor structures 
$\otimes$ and $\otimes^{\gamma}$ are isomorphic, 
it doesn't matter which we use as long as we don't mix them.
The $\otimes$-convention is intuitive when 
you do algebraic calculation while the 
$\otimes^{\gamma}$-convention is convenient 
to spell out formulas in homotopy theory.

Note that, under $\otimes^{\gamma}$-convention, 
we have $C[n]_{\bullet}=(C\otimes^{\gamma}\unit[n])_{\bullet}$.
\end{Rem}
\begin{Rem}
The \bfblue{Koszul product} of two cochain complexes
$C^{\bullet}$ and $D^{\bullet}$ is 
\begin{align*}
(C\otimes D)^n &=\bigoplus_{p+q=n}C^p\otimes D^q,\\
\di(x\otimes y) &= \di x\otimes y + (-1)^{|x|}x\otimes\di y.
\end{align*}
We have $C[n]^{\bullet} = (\unit[n]\otimes C)^{\bullet}$, 
where $\unit[n]$ is a cochain complex version.
\end{Rem}

\para{
Suppose further $\Aa$ is a closed abelian tensor category. 
Let $C_{\bullet}$, $D_{\bullet}$ be two complexes. 
Then there exists a natural boundary operator $\partial$ on 
the internal Hom $[C,D]_{\bullet}$ of their underlying 
graded objects. 
The resulted complex is called the \bfblue{Koszul Hom complex} of 
$C_{\bullet}$ and $D_{\bullet}$. 
By its construction, we only need to define the following morphisms
\[
[C_{p},D_{q}] \markar{\partial_{-p,q}^{(1)}} [C_{p+1},D_{q}],\qquad
[C_{p},D_{q}] \markar{\partial_{-p,q}^{(2)}} [C_{p},D_{q-1}].
\]
Note that the condition $\partial\circ\partial = 0$ requires that the following two morphisms must be negative to each other.
\begin{align*}
[C_{p},D_{q}] &\markar{\partial_{-p,q}^{(2)}} [C_{p},D_{q-1}] \markar{\partial_{-p,q-1}^{(1)}} [C_{p+1},D_{q-1}],\\
[C_{p},D_{q}] &\markar{\partial_{-p,q}^{(1)}} [C_{p+1},D_{q}] \markar{\partial_{-p-1,q}^{(2)}} [C_{p+1},D_{q-1}].
\end{align*}
The common convention is
\[
\partial_{-p,q}^{(1)} := -(-1)^{-p+q}[\partial_{p+1},D_q],\qquad
\partial_{-p,q}^{(2)} := [C_{p},\partial_{q}].
\]
In element notation, it reads
\[
(\partial f)(x) = \partial f(x) - (-1)^{|f|}f(\partial x).
\]
Then one can verify that this construction together 
with previous ones makes $\Ch(\Aa)$ 
a closed abelian tensor category.
}
\begin{Rem}
The functor $-\otimes^{\gamma}C$, i.e. $C\otimes-$ admits 
a right adjoint $\<C,-\>$ which gives another, 
although equivalent to the above one, 
closed abelian tensor category structure.
The complex $\<C,D\>_{\bullet}$ 
(called the \bfblue{left Koszul Hom complex}) is defined as follows. 
Its components are the same as $[C,D]_{\bullet}$ and the boundary 
operator reads
\[
(\partial f)(x) = (-1)^{|x|}\big(\partial f(x) - f(\partial x)\big).
\]
The Koszul Hom complex has the advantages that the signature 
is independent on where the ``function'' $f$ acts on, 
hence is conventional if one only focus on those elements. 
On the other hand, the left Koszul Hom complex 
has the advantages that the signature only depends on 
where ``function'' $f$ acts on, 
hence is conventional if one only focus on how those functions act. 
Note that those two complexes are canonically isomorphic 
since we have the braiding isomorphism  $\gamma$. 
Indeed, the isomorphism reads
\[
f\longmapsto
\big(
x\mapsto (-1)^{|f||x|}f(x)
\big).
\]
So we are free to use one of them for best fulfill our purpose.
\end{Rem}
\begin{Rem}
The \bfblue{Koszul Hom complex} $[C,D]^{\bullet}$ of two cochain complexes
$C_{\bullet}$ and $D_{\bullet}$ is
\begin{align*}
[C,D]^n &= \prod_{-p+q=n}[C^p,D^q] \\
(\di f)(x) &= \di f(x)- (-1)^{|f|}f(\di x),
\end{align*}
and the differential for \bfblue{left Koszul Hom complex} $\<C,D\>^{\bullet}$ 
is
\[
(\di f)(x) = (-1)^{|x|}(\di f(x) - f(\di x)).
\]
\end{Rem}

\para{
Let $\Aa$ be an abelian tensor category. 
We have seen that so is $\Ch(\Aa)$. 
Moreover, since the full subcategories $\Ch_{?}(\Aa)$ and $\Ch^{?}(\Aa)$ 
with $?$ equals $+,-,b$ are closed under 
finite limits and colimits and Koszul product, 
they are also abelian tensor categories. 
As for the full subcategories $\Ch_{?}(\Aa)$ and $\Ch^{?}(\Aa)$ 
with $?$ equals $\ge0,\le0$, we use the following proposition.
}

\begin{prop}
%\addcontentsline{toc}{subsection}{{\protect\numberline{}}Subcategories of special complexes}
Let $\Aa$ be an abelian category. Then
\begin{proplist}
\item the inclusion $\Ch_{\ge0}(\Aa)\into\Ch(\Aa)$ admits a left adjoint $\sk_{\ge0}$ and a right adjoint $\tau_{\ge0}$ and hence is exact;
\item the inclusion $\Ch_{\le0}(\Aa)\into\Ch(\Aa)$ admits a right adjoint $\sk_{\le0}$ and a left adjoint $\tau_{\le0}$ and hence is exact.
\end{proplist}
In particular, $\Ch_{\ge0}(\Aa)$ and $\Ch_{\le0}(\Aa)$ are abelian categories.
\end{prop}
\begin{proof}
The functors $\sk_{\ge0}$ and $\tau_{\ge0}$ are defined as follows.
\begin{align*}
\sk_{\ge0}(C)_n & =
\begin{cases}
C_n & n\ge0,\\
0 & n<0;
\end{cases} \\
\tau_{\ge0}(C)_n & =
\begin{cases}
C_n & n>0,\\
Z_0(C) & n=0,\\
0 & n<0.
\end{cases}
\end{align*}
Notice that, for any chain complex $C_{\bullet}$,
we have canonical chain morphisms
\[
\pi'_C\colon C_{\bullet}\To\sk_{\ge0}(C)_{\bullet},\qquad\qquad
i_C\colon \tau_{\ge0}(C)_{\bullet}\To C_{\bullet}.
\]
On the other hand, for any connective chain complex $C_{\bullet}$,
we have $Z_0(C)=C_0$. Therefore
\[
\sk_{\ge0}(C)_{\bullet} = \tau_{\ge0}(C)_{\bullet} = C_{\bullet}.
\]
Those identities and previous chain morphisms give rise to the 
units and counits for the adjunctions.

The functors $\sk_{\le0}$ and $\tau_{\le0}$ are defined as follows.
\begin{align*}
\sk_{\le0}(C)_n & =
\begin{cases}
C_n & n\le0,\\
0 & n>0;
\end{cases} \\
\tau_{\le0}(C)_n & =
\begin{cases}
C_n & n<0,\\
C_0/B_0(C) & n=0,\\
0 & n>0.
\end{cases}
\end{align*}
Notice that, for any chain complex $C_{\bullet}$,
we have canonical chain morphisms
\[
i'_C\colon \sk_{\le0}(C)_{\bullet}\To C_{\bullet},\qquad\qquad
\pi_C\colon C_{\bullet}\To\tau_{\le0}(C)_{\bullet}.
\]
On the other hand, for any connective chain complex $C_{\bullet}$,
we have $B_0(C)=0$. Therefore
\[
\sk_{\le0}(C)_{\bullet} = \tau_{\le0}(C)_{\bullet} = C_{\bullet}.
\]
Those identities and previous chain morphisms give rise to the 
units and counits for the adjunctions.
\end{proof}
\begin{Rem}
The complex $\tau_{\ge0}(C)_{\bullet}$ (resp. $\tau_{\ge0}(C)_{\bullet}$) 
is called the 
\bfblue{$0$-th truncation from below} 
(resp. \bfblue{$0$-th truncation from above})
of $C_{\bullet}$. 
One can similarly define $n$-th truncation functors 
$\tau_{\ge n}$ and $\tau_{\le n}$.
\end{Rem}

\para{
Let $\unit$ be the unit of $\Aa$. 
Consider the chain complex $I_{\bullet}$ defined as
\[
\cdots\To0\To\unit\markar{(-\id,\id)}\unit\oplus\unit\To0\To\cdots
\]
where $\unit\oplus\unit$ is of degree $0$.
This complex is called the \bfblue{standard interval complex}.
To justify this terminology and give an intuition, consider that 
the topological interval $[0,1]$ admits 
the following cellular decomposition:
it has a $1$-cell \emph{the interior} $e=(0,1)$ and 
two $0$-cells \emph{the endpoints} $v_0=0$ and $v_1=1$. 
Then the associated cellular chain complex is the connective complex
\[
\cdots\To0\To\ZZ e\markar{\partial}\ZZ v_0\oplus\ZZ v_1,
\]
where $\partial(e)=v_1-v_0$.
To illustrate, we formally write the complex $I_{\bullet}$ as
\[
\cdots\To0\To\unit{e}\markar{\partial^I}\unit{v_0}\oplus\unit{v_1}
\To0\To\cdots.
\]

Let $C_{\bullet}$ be a complex. 
Let's spell out the complex $(I\otimes C)_{\bullet}$.
First,
\[
(I\otimes C)_{n}=
C_{n-1}{e}\oplus C_n{v_0}\oplus C_n{v_1}.
\]
To illustrate, an element $(f,x,y)$ of this object is written as 
$f\colon x\sendto y$, called a \bfblue{copath} in $C_{n}$. 
Then the boundary operator $\partial^{I\otimes C}_n$ is induced by 
\[
\partial^I\otimes\id_C, \quad
\id_{I_0}\otimes\partial^C_n, \quad\text{and}\quad
-\id_{I_1}\otimes\partial^C_{n-1}.
\]
To spell out this boundary operator more concretely, 
let's use the following notation.
Let $A_j,B_i$ ($1\le j\le n, 1\le i\le m$) 
be objects in $\Aa$, then the matrix
\[
\begin{pmatrix}
f_{11} & \cdots & f_{1n} \\
\vdots & \ddots & \vdots \\
f_{m1} & \cdots & f_{mn}
\end{pmatrix}
\]
denotes the morphism 
$\bigoplus_{1\le j\le n}A_j\to \bigoplus_{1\le i\le m}B_i$ 
induced by the following morphisms
\[
f_{ij}\colon A_j\to B_i,\qquad
1\le j\le n, 1\le i\le m.
\]
Using this notation, the boundary operators can be written as
\[
\partial^{I\otimes C}_n=
\begin{pmatrix}
-\partial^C_{n-1} & 0 & 0 \\
-1 & \partial^C_n & 0 \\
1 & 0 & \partial^C_n
\end{pmatrix}.
\]
Withing element notation, this reads
\[
\partial(f\colon x\sendto y) = 
\big(
-\partial f\colon 
-f + \partial x \sendto
f + \partial y
\big).
\]

On the other hand, let's spell out the complex 
$\<I,C\>_{\bullet}$. First,
\[
\<I,C\>_{n}=
[\unit{e},C_{n+1}]\oplus[\unit{v_0},C_n]\oplus[\unit{v_1},C_n]=:
C_{n+1}{e^{\ast}}\oplus C_n{v_0^{\ast}}\oplus C_n{v_1^{\ast}}.
\]
To illustrate, an element $(f,x,y)$ of this object is written as 
$f\colon x\sendto y$, called a \bfblue{path} in $C_{n}$. 
Then the boundary operator $\partial^{\<I,C\>}_n$ is induced by 
\[
-[I_1,\partial^C_{n+1}],\quad
[I_0,\partial^C_{n}],\quad\text{and}\quad
[\partial^I,C_n].
\]
Using matrix notation, the boundary operators can be written as
\[
\partial^{\<I,C\>}_n=
\begin{pmatrix}
-\partial^C_{n+1} & -1 & 1 \\
0 & \partial^C_n & 0 \\
0 & 0 & \partial^C_n
\end{pmatrix}.
\]
Withing element notation, this reads
\[
\partial(f\colon x\sendto y) = 
\big(-\partial f - x + y \colon 
\partial x\sendto\partial y\big).
\]

}

\para{
Dually, one can consider the 
\bfblue{co-interval complex} $\hat{I}^{\bullet}$.
It is actually motivated by the cellular cochain complex of 
the interval $[0,1]$:
\[
\ZZ v_0^{\ast}\oplus\ZZ v_1^{\ast}
\markar{\di}\ZZ e^{\ast}\To0\To\cdots,
\]
where $\di$ is the morphism $(-\id,\id)$.
To illustrate, we formally write the complex $\hat{I}^{\bullet}$ as
\[
\unit v_0^{\ast}\oplus\unit v_1^{\ast}
\markar{\di_I}\unit e^{\ast}\To0\To\cdots,
\]
Apply the equivalence between cochain complexes and chain complexes, 
one can see that this complex is precisely $\<I,\unit\>^{\bullet}$, 
i.e. it is the \emph{weak dual} of $I^{\bullet}$.

Moreover, let $C^{\bullet}$ be a complex. 
Then the complex $(\hat{I}\otimes C)^{\bullet}$ is
\[
(\hat{I}\otimes C)^{n}=
C^{n-1}{e^{\ast}}\oplus C^n{v_0^{\ast}}\oplus C^n{v_1^{\ast}}
\]
with differential 
\[
\di_{\hat{I}\otimes C}^n=
\begin{pmatrix}
-\di_C^{n-1} & -1 & 1 \\
0 & \di_C^n & 0 \\
0& 0 & \di_C^n
\end{pmatrix}.
\]
Withing element notation, this reads
\[
\di(f\colon x\sendto y) = 
\big(-\di f -x + y \colon 
\di x\sendto\di y\big).
\]
Apply the equivalence between cochain complexes and chain complexes, 
one can see that this complex is precisely $\<I,C\>^{\bullet}$.

On the other hand, the complex $\<\hat{I},C\>^{\bullet}$ is 
\[
\<\hat{I},C\>^n=
C^{n+1}{e^{\ast\ast}}\oplus C^n{v_0^{\ast\ast}}\oplus C^n{v_1^{\ast\ast}}
\]
with differential 
\[
\di_{\<\hat{I},C\>}^n=
\begin{pmatrix}
-\di_C^{n+1} & 0 & 0 \\
-1 & \di_C^n & 0 \\
1 & 0 & \di_C^n
\end{pmatrix}.
\]
Withing element notation, this reads
\[
\di(f\colon x\sendto y) = 
\big(-\di f \colon 
-f + \di x\sendto f + \di y\big).
\]
Apply the equivalence between cochain complexes and chain complexes, 
one can see that this complex is precisely $(I\otimes C)^{\bullet}$.

The reason for the aboves is that $\hat{I}^{\bullet}$ is indeed 
the \emph{strong dual} of $I_{\bullet}$. 
To see this, let's translate $\hat{I}^{\bullet}$ into a chain complex. 
Then the chain complex $(\hat{I}\otimes I)_{\bullet}$ is 
concentrated at degree $1,0,-1$ with components
\begin{align*}
(\hat{I}\otimes I)_{1} &= 
\unit{v_0^{\ast} e}\oplus\unit{v_1^{\ast} e}, \\
(\hat{I}\otimes I)_{0} &= 
\unit{e^{\ast} e}\oplus
\unit{v_0^{\ast} v_0}\oplus\unit{v_1^{\ast} v_0}\oplus
\unit{v_0^{\ast} v_1}\oplus\unit{v_1^{\ast} v_1}, \\
(\hat{I}\otimes I)_{-1} &= 
\unit{e^{\ast} v_0}\oplus\unit{e^{\ast} v_1}.
\end{align*}
The boundary operators are presented by matrices as
\[
\partial_1=
\begin{pmatrix}
-1 & 1  \\
-1 & 0  \\
0 & -1  \\
1 & 0  \\
0 & 1 
\end{pmatrix},\qquad
\partial_0=
\begin{pmatrix}
1 & -1 & 1 & 0 & 0 \\
-1 & 0 & 0 & -1 & 1
\end{pmatrix}.
\]
Then the \bfblue{evaluation} 
$\ev\colon \hat{I}\otimes I\to\unit$ 
is the chain morphism given by 
\[
\ev_0=
\begin{pmatrix}
-1 & 1 & 0 & 0 & 1
\end{pmatrix},
\]
which can be illustrated by the rule
\[
\ev(x^{\ast}y)=(-1)^{|x||y|}\delta_{x,y}:=
\begin{cases}
(-1)^{|x||y|} & x=y,\\
0 & x\neq y.
\end{cases}
\]
The \bfblue{unit morphism} 
$\iota'\colon \unit\to I\otimes^{\gamma}\hat{I}$ 
is the chain morphism given by 
\[
\iota'_0=
{\begin{pmatrix}
1 & 1 & 0 & 0 & 1
\end{pmatrix}}^{\rm t},
\]
where ${\rm t}$ denotes the transpose of a matrix. 

On the other hand, 
since braiding $\hat{I}\otimes I\to I\otimes\hat{I}$ can be 
illustrated by the following rule
\[
\gamma(x^{\ast}y) = (-1)^{|x||y|}yx^{\ast},
\]
it follows that 
the chain complex $(I\otimes\hat{I})_{\bullet}$ is 
concentrated at degree $1,0,-1$ with components
\begin{align*}
(I\otimes\hat{I})_{1} &= 
\unit{ev_0^{\ast}}\oplus\unit{ev_1^{\ast}}, \\
(I\otimes\hat{I})_{0} &= 
\unit{ee^{\ast}}\oplus
\unit{v_0v_0^{\ast}}\oplus\unit{v_1v_0^{\ast}}\oplus
\unit{v_0v_1^{\ast}}\oplus\unit{v_1v_1^{\ast}}, \\
(I\otimes\hat{I})_{-1} &= 
\unit{v_0e^{\ast}}\oplus\unit{v_1e^{\ast}},
\end{align*}
and the boundary operators are presented by matrices as
\[
\partial_1=
\begin{pmatrix}
1 & -1  \\
-1 & 0  \\
1 & 0  \\
0 & -1  \\
0 & 1 
\end{pmatrix},\qquad
\partial_0=
\begin{pmatrix}
-1 & -1 & 0 & 1 & 0 \\
1 & 0 & -1 & 0 & 1
\end{pmatrix}.
\]
Then the \bfblue{evaluation} 
$\ev'\colon \hat{I}\otimes^{\gamma}I\to\unit$ 
is the chain morphism given by 
\[
\ev'_0=
\begin{pmatrix}
1 & 1 & 0 & 0 & 1
\end{pmatrix},
\]
which can be illustrated by the rule
\[
\ev'(xy^{\ast})=\delta_{x,y}:=
\begin{cases}
1 & x=y,\\
0 & x\neq y.
\end{cases}
\]
The \bfblue{unit morphism} 
$\iota\colon \unit\to I\otimes\hat{I}$ 
is the chain morphism given by 
\[
\iota_0=
{\begin{pmatrix}
-1 & 1 & 0 & 0 & 1
\end{pmatrix}}^{\rm t},
\]
where ${\rm t}$ denotes the transpose of a matrix. 
Then one can verify that the data
\emph{$(\ev,\iota)$ exhibits $\hat{I}$ 
as a strong dual of $I$} while 
\emph{$(\ev',\iota')$ exhibits $I$ 
as a strong dual of $\hat{I}$}.
}
\begin{Rem}
In a tensor category $\Cc$, an object $X$ is \bfblue{dualizable} if 
it has a \bfblue{strong dual} $X^{\ast}$, 
which is another object in $\Cc$, and a \bfblue{(strong) duality}, 
which is a pair of morphisms $\ev\colon X^{\ast}\otimes X\to\unit$ 
(called the \bfblue{evaluation}) and 
$\iota\colon\unit\to X\otimes X^{\ast}$ satisfying the following 
commutative diagrams
\[
\adjustbox{scale=0.9,center}{
\begin{tikzcd}
X^{\ast}\otimes (X\otimes X^{\ast})\ar[r,"{\cong}"]& 
(X^{\ast}\otimes X)\otimes X^{\ast}\ar[d,"{\ev\otimes\id}"]\\
X^{\ast}\otimes\unit\ar[u,"{\id\otimes\iota}"]\ar[r,"{\cong}"]& 
\unit\otimes X^{\ast}
\end{tikzcd}\qquad
\begin{tikzcd}
(X\otimes X^{\ast})\otimes X\ar[r,"{\cong}"]& 
X\otimes (X^{\ast}\otimes X)\ar[d,"{\id\otimes\ev}"]\\
\unit\otimes X\ar[u,"{\iota\otimes\id}"]\ar[r,"{\cong}"]& 
X\otimes\unit
\end{tikzcd}
}
\]
where the horizontal isomorphisms are the canonical ones.

Suppose $\Cc$ is further closed. 
Then the \bfblue{weak dual} of an object $X$ is precisely the 
object $[X,\unit]$. If $X$ is dualizable, then 
the weak dual is also the strong dual $X^{\ast}$. 
If this is the case, then for any object $Y$, we have a
canonical isomorphism
\[
Y\otimes X^{\ast} \Isom [X,Y].
\]

\end{Rem}

\para{
There are two natural chain morphisms from $\unit$ to $I$: 
$s_i$ ($i=0,1$) sends $\unit$ to the factor $\unit{v_i}$ in 
the $0$-th degree of $I$.
Then for any complex $C$, we have canonical morphisms of complexes
\begin{align*}
\iota_i\colon& C\To(I\otimes C)\quad(i=0,1),\\
\ev_i\colon& \<I, C\>\To
\<\unit,C\>\cong C \quad(i=0,1).
\end{align*}
To illustrate, let's spell out them by element notation:
\begin{gather*}
\iota_0(x) = (0\colon x\sendto 0),\qquad
\iota_1(y) = (0\colon 0\sendto y),\\
\ev_0(f\colon x\sendto y) = x,\qquad
\ev_1(f\colon x\sendto y) = y.
\end{gather*}

}

\para{
Let $f,g\colon C\to D$ be two morphisms of complexes.
As in algebraic topology, 
a \bfblue{(left) homotopy} $\Phi\colon f\then g$ between them 
is a commutative diagram of complexes as left hand side 
and a \bfblue{right homotopy} is a commutative diagram as right hand side.
\[
\begin{tikzcd}
C\ar[dr,"{f}"]\ar[d,"{\iota_0}"']&\\
(I\otimes C)\ar[r,"{\Phi}"description]&D\\
C\ar[ur,"{g}"']\ar[u,"{\iota_1}"]
\end{tikzcd}
\qquad\qquad
\begin{tikzcd}
& D\\
C\ar[r,"{\Phi}"description]\ar[ur,"{f}"]\ar[dr,"{g}"']&
{\<I,D\>}\ar[u,"{\ev_0}"']\ar[d,"{\ev_1}"]\\
& D
\end{tikzcd}
\]

Applying the previous conventions to chain complexes, 
a left homotopy is of the form
\[
\Phi_n=
\begin{pmatrix}
\phi_{n-1} & f_n & g_n
\end{pmatrix},
\]
and the fact $\Phi$ is a chain morphism is then equivalent to 
that the $1$-twisted morphism $\phi_{\bullet}$ 
satisfies the following equality:
\[
g_n - f_n = \partial_n\circ\phi_n + \phi_{n-1}\circ\partial_n.
\]
Dually, a right homotopy $\Phi\colon f\then g$ is of the form
\[
\Phi_n = 
{\begin{pmatrix}
\phi_{n} & f_n & g_n
\end{pmatrix}}^{\rm t},
\]
and the fact that $\Phi$ is a chain morphism is then equivalent to 
that the $1$-twisted morphism $\phi_{\bullet}$ 
satisfies the same equality as above.
This equality can be illustrated as the following diagram.
\[
\begin{tikzcd}[column sep=large, row sep=huge]
\cdots\ar[r] & 
C_{n+1}\ar[r,"{\partial_{n+1}}"]
\ar[d,shift right,"{f_{n+1}}"']
\ar[d,shift left,pos=0.4,"{g_{n+1}}"] &
C_n\ar[r,"{\partial_n}"]
\ar[d,shift right,"{f_n}"']
\ar[d,shift left,"{g_n}"]
\ar[dl,"{\phi_n}"description] &
C_{n-1}\ar[r]
\ar[d,shift right,pos=0.6,"{f_{n-1}}"']
\ar[d,shift left,"{g_{n-1}}"] 
\ar[dl,"{\phi_{n-1}}"description] & \cdots \\
\cdots\ar[r] & 
D_{n+1} \ar[r,"{\partial_{n+1}}"] & 
D_n \ar[r,"{\partial_n}"] & 
D_{n-1} \ar[r] & \cdots
\end{tikzcd}
\]
A $1$-twisted morphism $\phi_{\bullet}$ as above 
is called a \bfblue{chain homotopy}
from $f$ to $g$, also denoted by $\phi\colon f\then g$.

Similarly, applying to cochain complexes, 
a left homotopy is of the form
\[
\Phi^n=
\begin{pmatrix}
\phi^{n+1} & f^n & g^n
\end{pmatrix},
\]
and the fact $\Phi$ is a chain morphism is then equivalent to 
that the $(-1)$-twisted morphism $\phi^{\bullet}$ 
satisfies the following equality:
\[
g^n - f^n = \di^n\circ\phi^n + \phi^{n+1}\circ\di^n.
\]
Dually, a right homotopy $\Phi\colon f\then g$ is of the form
\[
\Phi^n = 
{\begin{pmatrix}
\phi^{n} & f^n & g^n
\end{pmatrix}}^{\rm t},
\]
and the fact that $\Phi$ is a chain morphism is then equivalent to 
that the $(-1)$-twisted morphism $\phi^{\bullet}$ 
satisfies the same equality as above.
This equality can be illustrated as the following diagram.
\[
\begin{tikzcd}[column sep=large, row sep=huge]
\cdots\ar[r] & 
C^{n-1}\ar[r,"{\di^{n-1}}"]
\ar[d,shift right,"{f^{n-1}}"']
\ar[d,shift left,pos=0.4,"{g^{n-1}}"] &
C^n\ar[r,"{\di^n}"]
\ar[d,shift right,"{f^n}"']
\ar[d,shift left,"{g^n}"]
\ar[dl,"{\phi^n}"description] &
C^{n+1}\ar[r]
\ar[dl,"{\phi^{n+1}}"description]
\ar[d,shift right,pos=0.6,"{f^{n+1}}"']
\ar[d,shift left,"{g^{n+1}}"]  & \cdots \\
\cdots\ar[r] & 
D^{n-1} \ar[r,"{\di^{n-1}}"] & 
D^n \ar[r,"{\di^n}"] & 
D^{n+1} \ar[r] & \cdots
\end{tikzcd}
\]
A $(-1)$-twisted morphism $\phi^{\bullet}$ as above 
is called a \bfblue{cochain homotopy}
from $f$ to $g$, also denoted by $\phi\colon f\then g$.

Note that these notions are equivalent and 
we'll not distinguish them if no necessary.
}
\begin{Rem}
The above definitions form the basic blocks of 
the machinery of homotopy theory. 
Obviously, if we replace the above $\otimes^{\gamma}$-version 
of closed tensor structure by $\otimes$-version, 
we can still obtained an equivalent theory. 
However, the concrete formulas would become cumbersome and 
looks far from the those in usual text of 
homological algebra.
\end{Rem}

\para{
Two (co)chain maps $f,g\colon C\tto D$ 
are said to be \bfblue{homotopic}, denoted by $f\simeq g$,
if there exists a (co)chain homotopy $\Phi\colon f\then g$. 
A (co)chain morphism $f\colon C\to D$ 
is called a \bfblue{homotopy equivalence}
if there exists another (co)chain morphism 
$g\colon D\to C$ 
such that $g\circ f\simeq \id_C$ and 
$f\circ g\simeq \id_D$.
Two (co)chain complexes $C$ and $D$ 
are said to be \bfblue{homotopy equivalent} 
if there exists a (co)chain homotopy equivalence 
$f\colon C\to D$. 

In this way, we can form a new category 
\bfblue{$K(\Aa)$} as follows:
\begin{itemize}
\item 
the objects of $K(\Aa)$ are as of $\Ch(\Aa)$,
\item 
the Hom set $\Hom_{K(\Aa)}(C,D)$ is the quotient set 
of $\Hom_{\Ch(\Aa)}(C,D)$ modulo homotopies.
\end{itemize}
This category is called the \bfblue{homotopy category} of 
$\Ch(\Aa)$ or $\Aa$ if there are no ambiguities.
In the same way, we have subcategories $K_{?}(\Aa)$ and $K^{?}(\Aa)$  
with $?$ equals $c,\ge0,\le0,+,-,b$.

Given two homotopies $\Phi\colon f\then g$ and $\Psi\colon g\then h$, 
then the \bfblue{vertical composition} of them is 
more or less the sum of them:
\[
\Psi\dotplus\Phi:=(\phi+\psi,f,h).
\] 
Note that $\Psi\dotplus\Phi\neq\Phi\dotplus\Psi$, 
the later even doesn't make sense. 
Under this composition rule, the inverse of a homotopy 
$\Phi\colon f\then g$ is the homotopy $-\Phi\colon g\then f$ defined as
\[
-\Phi:=(-\phi,g,f).
\]

Given two homotopies $\Phi,\Psi$ as below:
\[
\begin{tikzcd}[column sep=large, row sep=large]
{C}
\ar[r,bend left,"{f}",""{name=a,below}]
\ar[r,bend right,"{g}"',""{name=b}]
\ar[Rightarrow,from=a,to=b,"{\Phi}"]&
{D}
\ar[r,bend left,"{f'}",""{name=a,below}]
\ar[r,bend right,"{g'}"',""{name=b}]
\ar[Rightarrow,from=a,to=b,"{\Psi}"]&{E}
\end{tikzcd}
\]
the \bfblue{horizontal composition} is defined as
\[
\Psi\ast\Phi := \Psi\circ g \dotplus f'\circ\Phi,
\]
where the composition $f'\circ\Phi$ should be consider as given by 
\[
(I\otimes C)_{\bullet}\markar{\Phi}
D_{\bullet}\markar{f'}E_{\bullet}\quad\text{or}\quad
(\hat{I}\otimes C)^{\bullet}\markar{\Phi}
D^{\bullet}\markar{f'}E^{\bullet},
\]
while the composition $\Psi\circ g$ given by
\[
C_{\bullet}\markar{g}
D_{\bullet}\markar{\Psi}\<I,E\>_{\bullet}\quad\text{or}\quad
C^{\bullet}\markar{g}
D^{\bullet}\markar{\Psi}\<\hat{I},E\>^{\bullet}.
\]
Therefore, the definition can be reads as
\[
\Psi\ast\Phi:= \big(f'\circ\phi + \psi\circ g,f'\circ f,g'\circ g\big).
\]

Treat homotopies between chain morphisms as $2$-morphisms, 
we obtain a $2$-category structure on $\Ch(\Aa)$. 
Further, we can involves composition rules of 
homotopies between $2$-morphisms, and 
homotopies between those homotopies, etc. 
Conceptually, we should obtain an $\infty$-category structure.

However, this structure is, if it exists, at least not strict. 
To see this, consider the following diagram.
\[
\begin{tikzcd}[column sep=large, row sep=large]
{C}
\ar[r,bend left=75,"{f}",""{name=a,below}]
\ar[r,,"{g}"description,""{name=b},""{name={a'},below}]
\ar[r,bend right=75,"{h}"',""{name={b'}}]
\ar[Rightarrow,from=a,to=b,"{\Phi}"]
\ar[Rightarrow,from={a'},to={b'},"{\Psi}"]&
{D}
\ar[r,bend left=75,"{f'}",""{name=a,below}]
\ar[r,,"{g'}"description,""{name=b},""{name={a'},below}]
\ar[r,bend right=75,"{h'}"',""{name={b'}}]
\ar[Rightarrow,from=a,to=b,"{\Phi'}"]
\ar[Rightarrow,from={a'},to={b'},"{\Psi'}"]&{E}
\end{tikzcd}
\]
There are two ways to compose them:
\[
(\Psi'\dotplus\Phi')\ast(\Psi\dotplus\Phi)\qquad\text{and}\qquad
\Psi'\ast\Psi\dotplus\Phi'\ast\Phi.
\]
The \bfblue{interchange law} in the axioms of $2$-category says
that the above two compositions are the same. 
However, they do not equal. 
In fact, there is a homotopy $\Theta$ between them 
(viewed as chain morphisms) 
given by the $1$-twisted morphism
\[
\theta= (\phi'\circ\psi,0,0)\colon
I\otimes C\To E,
\]
or equivalently the $2$-twisted morphism
\[
\phi'\circ\psi \colon C\To E.
\]

}

\para{
Passing to the homotopy category $K(\Aa)$, 
one may expect the \emph{interchange law} 
as well as more \emph{coherence law} holds strictly. 
However, even the notion of homotopies itself is 
lack of sense. 
Two chain morphisms present 
the same morphism in $K(\Aa)$ if and only if there is 
a homotopy between them. 
But such a homotopy is not unique, 
even up to homotopy! 
Indeed there are non-homotopic $2$-morphisms 
between chain morphisms.
Consequently, the notion of homotopies between morphisms in 
$K(\Aa)$ is not well-defined!
}

\para{
Recall that in the homotopy theory for topological spaces, 
the key step to build a workable framework is 
to define a suitable notion of $\infty$-groupoids 
as well as the category of them. 
In particular, we choose CW complexes as such a model in \S\ 1.

Let $C_{\bullet}, D_{\bullet}$ be two chain complexes. 
First note that
\begin{proplist}
\item 
A chain morphism $f\colon C_{\bullet}\to D_{\bullet}$ 
is a homogeneous morphism between 
the underlying graded objects satisfying certain properties, 
hence an element in $\Hom_{\Gr(\Aa)}(C,D)$.
\item 
A homotopy is determined by a $1$-twisted morphism, 
hence an element in $\Hom_{\Gr(\Aa)}(C,D(1))$.
\item 
A homotopy between homotopies is determined by a $2$-twisted morphism, 
hence an element in $\Hom_{\Gr(\Aa)}(C,D(2))$.
\end{proplist}
Invested by the above, one may expect the \emph{Hom-space}, i.e. 
the $\infty$-groupoid encoding the higher homotopies of chain morphisms
from $C_{\bullet}$ to $D_{\bullet}$ is the complex 
\bfblue{$\HHom_{\Ch_{\ast}(\Aa)}(C,D)_{\bullet}$} whose underlying 
graded abelian group is precisely $\Hom_{\Gr(\Aa)}(C,D)_{\bullet}$ 
and the boundary operator reads
\[
\partial(f) = \partial^D\circ{f}-(-1)^{|f|}f\circ\partial^C.
\]
This complex is called the \bfblue{Hom-complex}. 

Similarly, let $C^{\bullet}, D^{\bullet}$ be two chain complexes. 
Then we also have the \bfblue{Hom-complex} 
\bfblue{$\HHom_{\Ch^{\ast}(\Aa)}(C,D)^{\bullet}$} with differential 
\[
\di(f) = \di^D\circ{f}-(-1)^{|f|}f\circ\di^C.
\]
It is natural to ask: after identify chain complexes and cochain complexes, 
what's the relation between the above two complexes?
Let's spell out their components first:
\begin{align*}
\HHom_{\Ch_{\ast}(\Aa)}(C,D)_n &= 
\prod_{-p+q=n}\Hom_{\Aa}(C_p,D_q), \\
\HHom_{\Ch^{\ast}(\Aa)}(C,D)^n &= 
\prod_{-p+q=n}\Hom_{\Aa}(C^p,D^q).
\end{align*}
Therefore
\begin{align*}
\HHom_{\Ch^{\ast}(\Aa)}(C,D)_n &= 
\HHom_{\Ch^{\ast}(\Aa)}(C,D)^{-n} \\&=
\prod_{-p+q=-n}\Hom_{\Aa}(C^p,D^q) \\&=
\prod_{-p+q=-n}\Hom_{\Aa}(C_{-p},D_{-q}) \\&=
\prod_{-p+q=n}\Hom_{\Aa}(C_p,D_q) =
\HHom_{\Ch_{\ast}(\Aa)}(C,D)_n.
\end{align*}
Then one can verify that they are the same complex, 
hence can be simply denoted by $\HHom_{\Ch(\Aa)}(C,D)$.
}

\para{
Recall that, for a pointed topological space $(X,x_0)$, its $n$-th homotopy group 
$\pi_n(X,x_0)$ is defined as either the set of homotopy classes of based maps 
$S^n\to X$ or the set of homotopy classes of maps $(I^n,\partial I^n)\to(X,x_0)$.

The complex corresponding to the \emph{$n$-cube} $I^n$ is 
$I^{\otimes n}_{\bullet}$, 
the $n$-fold Koszul product of $I_{\bullet}$.
Let's spell out this complex concretely. 
To do this, let's introduce the following notion:
\begin{itemize}
\item 
An object $M$ in $\Aa$ is \bfblue{free} if it is isomorphic to 
a direct sum of copies of $\unit$. 
A \bfblue{basis} of a free object $M$ is an isomorphism 
from a direct sum of copies of $\unit$ to it. 
In particular, an \bfblue{member of the basis} is a component 
$\unit\to M$ of this isomorphism. In this way, we can always present a basis 
as the collection of its members.
\end{itemize}
The complex $I_{\bullet}$ has the basis $\{v_0,v_1\}$ at degree $0$ 
and the basis $\{e\}$ at degree $1$. 
Using this \emph{basis notation}, the boundary operator can be written as 
\[
\partial(e) = v_1-v_0.
\]
Let $\alpha$ be a \bfblue{$\{v_0,v_1,e\}$-string}, i.e 
a sequence of letters consisting of $v_0$, $v_1$ and $e$. 
Then the \bfblue{length} of $\alpha$ is the number of letters in it 
and the \bfblue{total degree} $|\alpha|$ is the sum of degrees of the letters 
(where $v_0$, $v_1$ are of degree $0$ and $e$ is of degree $1$).
Therefore
\begin{itemize}
\item 
$I^{\otimes n}_{i}$ has basis consisting of 
$\{v_0,v_1,e\}$-strings of length $n$ and degree $i$;
\item 
the boundary operator reads
\[
\partial(\alpha\beta)=(\partial\alpha)\beta+(-1)^{|\alpha|}\alpha(\partial\beta).
\]

\end{itemize}

Since $\partial I^n$ is $n$-cube without its unique $n$-cell, 
the corresponding complex $\partial I^{\otimes n}_{\bullet}$ should be the complex 
$I^{\otimes n}_{\bullet}$ without its top degree 
$I^{\otimes n}_n=\unit{ee\cdots e}$.

Note that the $n$-sphere $S^n$ has a cellular decomposition: 
the $0$-cell is its base point and 
the $n$-cell is all outside that point. 
Using this cellular decomposition, 
the complex corresponding to $S^n$ is the complex 
$\unit\oplus\unit[-n]$, where the first factor presents the base point.

Let $C_{\bullet}$ be a complex. 
A \bfblue{(cubic) $n$-loop} in $C_{\bullet}$ is a chain morphism 
$\gamma\colon I^{\otimes n}_{\bullet}\to C_{\bullet}$ such that 
the composition of it with the canonical inclusion 
$\partial I^{\otimes n}_{\bullet}\into I^{\otimes n}_{\bullet}$ 
is $0$. 
Likewise, a \bfblue{(spheric) $n$-loop} in $C_{\bullet}$ is 
a chain morphism 
$\gamma\colon \unit\oplus\unit[-n]\to C_{\bullet}$ such that 
the composition of it with the canonical inclusion 
$\unit\into\unit\oplus\unit[-n]$ is $0$. 
It is clear that both of them are equivalent to a morphism 
$\gamma_n\colon\unit\to C_{n}$ such that $\partial\circ\gamma_n=0$. 
In other words, 
\[
\gamma_n\in Z_n\HHom_{\Ch(\Aa)}(\unit,C).
\] 
On the other hand, 
a homotopy $H\colon \gamma \then \eta$ between two $n$-loops 
is determined by a morphism $h\colon\unit\to C_{n+1}$ such that 
$\partial\circ h = \eta_n-\gamma_n$, i.e.
\[
\eta_n-\gamma_n\in B_n\HHom_{\Ch(\Aa)}(\unit,C).
\]
Therefore we have canonical isomorphisms
\[
\pi_n(C):=
\{\text{homotopy classes of $n$-loops in $C$ }\}
\cong
H_n\HHom_{\Ch(\Aa)}(\unit,C).
\]
This abelian group is called the 
\bfblue{$n$-th homotopy group} of $C_{\bullet}$.
}

\begin{Rem}
Be aware that $\pi_n(C)$ is in general not the underlying 
abelian group of $H_n(C)$, i.e. 
$\pi_n(C)\neq\Hom_{\Aa}(\unit,H_n(C))$. 
The reason is that the functor $\Hom_{\Aa}(\unit,-)$ is in general 
not exact. As an example, consider the category of abelian sheaves 
on a general topological space.
\end{Rem}

\para{
The above procession works for general homotopies:
\begin{proplist}
\item 
By a \bfblue{boundary condition} of $C_{\bullet}$, 
we mean a chain morphism from $\partial I^{\otimes n}_{\bullet}$ 
to $C_{\bullet}$.
By a \bfblue{$n$-cell} attaching to $C_{\bullet}$ 
via a boundary condition $\delta$, 
we mean a chain morphism from $I^{\otimes n}_{\bullet}$ 
to $C_{\bullet}$ whose restriction to 
$\partial I^{\otimes n}_{\bullet}$ is $\delta$.
Then it is clear that 
the set of all $n$-cells attaching to $C_{\bullet}$ 
via a boundary condition $\delta$ equals to the coset
\[
\{\text{$n$-loop in $C_{\bullet}$}\} + \delta.
\]
\item 
Let $\delta\colon\partial I^{\otimes n}_{\bullet}\to C_{\bullet}$ 
be a boundary condition. 
By a \bfblue{homotopy rel $\delta$}, 
we mean a homotopy whose restriction to 
$\partial I^{\otimes n}_{\bullet}$ is $I\otimes\delta$. 
Then it is clear that the quotient set of 
all $n$-cells attaching to $C_{\bullet}$ 
via a boundary condition $\delta$ 
up to homotopy rel $\delta$ 
equals to the coset
\[
\pi_n(C) + \delta.
\]
\item 
With above notions, 
a chain morphism from $C_{\bullet}$ to $D_{\bullet}$ 
can be viewed as a $0$-cell attaching to $[C,D]_{\bullet}$ 
via the empty boundary condition and a homotopy is 
a homotopy rel nothing. Hence
\begin{align*}
\Hom_{\Ch(\Aa)}(C,D) \cong 
Z_0\HHom_{\Ch(\Aa)}(C,D),\\
\Hom_{K(\Aa)}(C,D) \cong 
H_0\HHom_{\Ch(\Aa)}(C,D).
\end{align*}
\item 
A given pair of chain morphisms from $C_{\bullet}$ to $D_{\bullet}$ 
can be viewed as a boundary condition 
$\partial I_{\bullet}\to[C,D]_{\bullet}$. 
Then a homotopy between them is a 
$1$-cell attaching to $[C,D]_{\bullet}$ via that boundary condition 
and a $2$-homotopy between such homotopies is 
a homotopy rel that boundary condition.
Hence 
\begin{align*}
\{\text{homotopy between given chain morphisms}\} \cong 
Z_1\HHom_{\Ch(\Aa)}(C,D),\\
\{\text{homotopy class of above}\} \cong 
H_1\HHom_{\Ch(\Aa)}(C,D).
\end{align*}
\item 
A given pair of homotopies can be viewed as a boundary condition 
$\partial I^{\otimes 2}_{\bullet}\to[C,D]_{\bullet}$, 
where the $1$-degree encodes the two homotopies and 
$0$-degree the domain and codomains of them. 
Then a $2$-homotopy between them is a 
$2$-cell attaching to $[C,D]_{\bullet}$ via that boundary condition 
and a $3$-homotopy between such $2$-homotopies is 
a homotopy rel that boundary condition.
Hence 
\begin{align*}
\{\text{$2$-homotopy between given homotopies}\} \cong 
Z_2\HHom_{\Ch(\Aa)}(C,D),\\
\{\text{homotopy class of above}\} \cong 
H_2\HHom_{\Ch(\Aa)}(C,D).
\end{align*}
\item 
In general, a given pair of $(n-1)$-homotopies can be viewed as 
a boundary condition 
$\partial I^{\otimes n}_{\bullet}\to[C,D]_{\bullet}$, 
where the components with basis consisting of 
strings starting with $v_0$ (resp. $v_1$) comes from 
the first (resp. second) $(n-1)$-homotopy. 
Then a $n$-homotopy between them is a 
$n$-cell attaching to $[C,D]_{\bullet}$ via that boundary condition 
and a $(n+1)$-homotopy between such $n$-homotopies is 
a homotopy rel that boundary condition.
Hence 
\begin{align*}
\{\text{$n$-homotopy between given homotopies}\} \cong 
Z_n\HHom_{\Ch(\Aa)}(C,D),\\
\{\text{homotopy class of above}\} \cong 
H_n\HHom_{\Ch(\Aa)}(C,D).
\end{align*}
\[
\cdots
\quad
\begin{tikzcd}
I^{\otimes n-1}\otimes C\ar[d]\ar[dr]&\\
I^{\otimes n}\otimes C\ar[r]&D\\
I^{\otimes n-1}\otimes C\ar[u]\ar[ur]&
\end{tikzcd}
\qquad
\begin{tikzcd}
I^{\otimes n}\otimes C\ar[d]\ar[dr]&\\
I^{\otimes n+1}\otimes C\ar[r]&D\\
I^{\otimes n}\otimes C\ar[u]\ar[ur]&
\end{tikzcd}
\quad
\cdots
\]
\end{proplist}
}

\para{
It is straightforward to show that 
both the \emph{evaluation} $\hat{I}\otimes I\to\unit$ 
and the \emph{unit moprhism} $\unit\to I\otimes\hat{I}$ are 
quasi-isomorphisms. 
In this way, we may think $\hat{I}_{\bullet}$ 
as $I^{\otimes -1}_{\bullet}$ 
and more generally $\hat{I}^{\otimes n}_{\bullet}$
as $I^{\otimes -n}_{\bullet}$ for any natural number $n$.
Then the previous discussion still works.

In details. The complex $\hat{I}_{\bullet}$ has the basis 
$\{v_0^{\ast},v_1^{\ast}\}$ at degree $0$ and the basis 
$\{e^{\ast}\}$ at degree $-1$. 
The boundary operator of $I^{\otimes -1}_{\bullet}$ reads
\[
\partial(v_0^{\ast})=e^{\ast},\qquad
\partial(v_1^{\ast})=e^{\ast}.
\]
Then, the complex $I^{\otimes -n}_{\bullet}$ 
can be described as follows.
\begin{itemize}
\item 
$I^{\otimes -n}_i$ has a basis of 
$\{v_0^{\ast},v_1^{\ast},e^{\ast}\}$-strings 
of length $n$ and degree $i$;
\item 
the boundary operator reads
\[
\partial(\alpha\beta)=(\partial\alpha)\beta+(-1)^{|\alpha|}\alpha(\partial\beta).
\]
\end{itemize}
Then the complex $\partial I^{\otimes -n}_{\bullet}$ is 
the complex $I^{\otimes -n}_{\bullet}$ without its bottom degree 
$I^{\otimes -n}_{-n}=\unit{e^{\ast}e^{\ast}\cdots e^{\ast}}$.

We can also define the complex corresponding to $S^{-n}$ 
as the complex $\unit\oplus\unit[n]$, 
where the first factor presents the base point.

Then, one can define the notions of 
\bfblue{cubic} and \bfblue{spheric $(-n)$-loops} 
as before and verify the similar statements:
\begin{proplist}
\item 
a $(-n)$-loop in $C_{\bullet}$ is equivalent to 
an element in $Z_{-n}\HHom_{\Ch(\Aa)}(\unit,C)$;
\item 
two $(-n)$-loops are homotopic if they are different by 
an element in $B_{-n}\HHom_{\Ch(\Aa)}(\unit,C)$;
\item 
$\pi_{-n}(C) = H_{-n}\HHom_{\Ch(\Aa)}(\unit,C)$.
\end{proplist}
}

\para{
Through the identification of cochain complexes and chain complexes, 
the above statements can be translated as:
\begin{proplist}
\item 
a $n$-loop in $C^{\bullet}$ is equivalent to 
an element in $Z^{-n}\HHom_{\Ch(\Aa)}(\unit,C)$;
\item 
two $n$-loops of $C^{\bullet}$ are homotopic if they are different by 
an element in $B^{-n}\HHom_{\Ch(\Aa)}(\unit,C)$.
\end{proplist}
Then we can have similar statements for general homotoies, hence
\begin{align*}
\{\text{$n$-homotopy between given homotopies}\} \cong 
Z^{-n}\HHom_{\Ch(\Aa)}(C,D),\\
\{\text{homotopy class of above}\} \cong 
H^{-n}\HHom_{\Ch(\Aa)}(C,D).
\end{align*}

In this way, we can think the Hom-complex $\HHom_{\Ch(\Aa)}(C,D)$ 
\emph{encodes homotopy informations into its connective truncation.}
If one remember how homotopy groups can be viewed as negative-degree 
intrinsic cohomology groups. Then one would agree that it is more 
natural to view the complex $\HHom_{\Ch(\Aa)}(C,D)$ as a cochain complex.
}

\para{
From previous observation, 
we can encode the $\infty$-category structure on $\Ch(\Aa)$ 
into the Hom-complexes. 
To summarize, we have the followings.
\begin{proplist}
\item 
A \bfblue{$n$-morphism} from $C$ to $D$ 
is a $(1-n)$-cocycle, i.e. $(n-1)$-cycle, of $\HHom_{\Ch(\Aa)}(C,D)$.
\item 
A \bfblue{$n$-homotopy} between $n$-morphisms
$\phi\colon f\then g$ 
is an element of $\HHom_{\Ch(\Aa)}(C,D)^{-n}$ 
such that $\di\phi = g-f$.
\item 
The composition rules are encoded into the bilinear map
\[
\HHom_{\Ch(\Aa)}(D,E)\otimes\HHom_{\Ch(\Aa)}(C,D)
\To\HHom_{\Ch(\Aa)}(C,E)
\]
induced from the bilinear maps
\[
\Hom\big(D(q),E(p+q)\big)\otimes
\Hom\big(C,D(q)\big)
\To\Hom\big(C,E(p+q)\big)
\]
given by $g\otimes f\mapsto g\circ f$.
\item 
The identity morphism is encoded into a homomorphism from $\ZZ$ 
to $\HHom_{\Ch(\Aa)}(C,C)$ defined by 
$1\mapsto\id_{C}$.
\item 
The coherent axioms are encoded into the commutative diagrams
\[
\adjustbox{scale=0.7,center}{
\begin{tikzcd}
\HHom_{\Ch(\Aa)}(E,F)\otimes\HHom_{\Ch(\Aa)}(D,E)
\otimes\HHom_{\Ch(\Aa)}(C,D)
\ar[r]\ar[d]
&
\HHom_{\Ch(\Aa)}(E,F)\otimes\HHom_{\Ch(\Aa)}(C,E) 
\ar[d]
\\
\HHom_{\Ch(\Aa)}(E,D)\otimes\HHom_{\Ch(\Aa)}(C,D) 
\ar[r]
&
\HHom_{\Ch(\Aa)}(C,F)
\end{tikzcd}
}
\]
(which encodes the associativities) and 
\[
\adjustbox{scale=0.8,center}{
\begin{tikzcd}
\ZZ\otimes\HHom_{\Ch(\Aa)}(C,D)\ar[dr,"{\cong}"']\ar[r]&
\HHom_{\Ch(\Aa)}(D,D)\otimes\HHom_{\Ch(\Aa)}(C,D)\ar[d]\\&
\HHom_{\Ch(\Aa)}(C,D)\\
\HHom_{\Ch(\Aa)}(C,D)\otimes\ZZ\ar[ur,"{\cong}"]\ar[r]&
\HHom_{\Ch(\Aa)}(C,D)\otimes\HHom_{\Ch(\Aa)}(C,C)\ar[u]
\end{tikzcd}
}
\]
(which encodes the identity laws).
\end{proplist}
%In another words, $\Ch(\Aa)$ is naturally a $\Ch$-enriched category 
%and the homotopy data is encoded in the Hom-complex 
%$\HHom_{\Ch(\Aa)}(-,-)$.
}



 







\clearpage
\section{Dg-category theory}
\para{
Inspirited by previous section, the following definition arises. 

A \bfblue{dg-category} is precisely a $\Ch$-enriched category. 
(Of course, one can slightly generalize this notion 
by replacing $\Ch$ with $\Ch(k)$). 
More precisely, a dg-category $\Cc$ consists of
\begin{itemize}
\item 
a collection of \itblue{objects} $\ob\Cc$;
\item 
for any two objects $C$ and $D$, 
a \bfblue{Hom-complex} $\HHom_{\Cc}(C,D)\in\Ch$;
\begin{itemize}
\item 
an element of $\HHom_{\Cc}(C,D)^n$ is called 
a \bfblue{(general) morphism 
of (cohomological) degree $n$};
\item 
a closed morphism of degree $1-n$ is called 
a \bfblue{$n$-morphism} from $C$ to $D$, 
denoted by $f\colon C\to D$;
\item 
a \bfblue{$n$-homotopy} $\phi\colon f\then g$ 
is a morphism of degree $-n$ such that $\di\phi = g-f$;
\end{itemize}
\item 
for any three objects $C$, $D$ and $E$, 
a cochain map
\[
\HHom_{\Cc}(D,E)\otimes\HHom_{\Cc}(C,D)
\To\HHom_{\Cc}(C,E)
\]
called the \itblue{composition rule};
\item 
for any object $C$, 
a cochain map $\ZZ\to\HHom_{\Cc}(C,C)$ 
called the \itblue{identity}.
\end{itemize}
Those data must satisfies the following axioms:
\begin{enumerate}
\item 
for any objects $C,D,E,F$, the following diagram commutes;
\[
\adjustbox{scale=0.9,center}{
\begin{tikzcd}
\HHom_{\Cc}(E,F)\otimes\HHom_{\Cc}(D,E)
\otimes\HHom_{\Cc}(C,D)
\ar[r]\ar[d]
&
\HHom_{\Cc}(E,F)\otimes\HHom_{\Cc}(C,E) 
\ar[d]
\\
\HHom_{\Cc}(E,D)\otimes\HHom_{\Cc}(C,D) 
\ar[r]
&
\HHom_{\Cc}(C,F)
\end{tikzcd}
}
\]
\item 
for any objects $C,D$, the following diagram commutes.
\[
\adjustbox{scale=0.9,center}{
\begin{tikzcd}
\ZZ\otimes\HHom_{\Cc}(C,D)\ar[dr,"{\cong}"']\ar[r]&
\HHom_{\Cc}(D,D)\otimes\HHom_{\Cc}(C,D)\ar[d]\\&
\HHom_{\Cc}(C,D)\\
\HHom_{\Cc}(C,D)\otimes\ZZ\ar[ur,"{\cong}"]\ar[r]&
\HHom_{\Cc}(C,D)\otimes\HHom_{\Cc}(C,C)\ar[u]
\end{tikzcd}
}
\]
\end{enumerate}

Any dg-category $\Cc$ admits a (pre-additive) category $\Cc_0$ 
(its \bfblue{underlying category}) obtained by applying the 
\emph{change of base categories} $Z^0\colon\Ch\to\Ab$ 
(or $Z^0\colon\Ch\to\Ab\to\Set$ if one insists on an ordinary category)
and another $\h\Cc$ 
(its \bfblue{homotopy category}) obtained by applying the 
\emph{change of base categories} $H^0\colon\Ch\to\Ab$
(or $H^0\colon\Ch\to\Ab\to\Set$).
}

\begin{Eg}
Let $\Aa$ be an additive category. 
Then $\Ch(\Aa)$ is automatically a dg-category. 
The underlying category of $\Ch(\Aa)$ is 
the ordinary category of complexes. 
The homotopy category $\h\Ch(\Aa)$ is precisely $\Kk(\Aa)$.
The similar conventions apply to the subcategories 
$\Ch_{?}(\Aa)$ and $\Ch^{?}(\Aa)$ with 
$?$ equals $c,\ge0,\le0,+,-,b$.
\end{Eg}

\para{
A \bfblue{dg-functor} between dg-categories 
$F\colon\Cc\to\Dd$ is an enriched functor. 
Equivalently, a dg-functor $F$ consists of
\begin{itemize}
\item 
a mapping between objects $F_0\colon\ob\Cc\to\ob\Dd$,
\item 
a family of cochain maps 
$F_{C,D}\colon\HHom_{\Cc}(C,D)\to\HHom_{\Dd}(F(C),F(D))$, 
indexed by $C, D\in\ob\Cc$,
\end{itemize}
satisfying the following associative and unitary laws:
\begin{enumerate}
\item 
for any objects $C,D,E$, the following diagram commutes;
\[
\adjustbox{scale=0.9,center}{
\begin{tikzcd}
\HHom_{\Cc}(D,E)
\otimes\HHom_{\Cc}(C,D)
\ar[r]\ar{d}{F}
&
\HHom_{\Cc}(C,E)
\ar{d}{F}
\\
\HHom_{\Dd}(F(D),F(E))
\otimes\HHom_{\Cc}(F(C),F(D))
\ar[r]
&
\HHom_{\Dd}(F(C),F(E)) 
\end{tikzcd}
}
\]
\item 
for any object $C$, the following diagram commutes.
\[
\begin{tikzcd}
\ZZ\ar[dr]\ar[r]&
\HHom_{\Cc}(C,C)\ar{d}{F}\\&
\HHom_{\Dd}(F(C),F(C))
\end{tikzcd}
\]
\end{enumerate}

Given two dg-functors 
$F\colon\Cc\to\Dd$ and $G\colon\Dd\to\Ee$, 
their \bfblue{composition} $G\circ F$ is given as follows:
\begin{itemize}
\item 
the mapping $(G\circ F)_0\colon\ob\Cc\to\ob\Ee$ is the composition 
$G_0\circ F_0$;
\item 
the cochain maps 
$(G\circ F)_{C,D}\colon\HHom_{\Cc}(C,D)\to\HHom_{\Ee}(GF(C),GF(D))$ 
is given by the composition of  
$F_{C,D}\colon\HHom_{\Cc}(C,D)\to\HHom_{\Dd}(F(C),F(D))$ 
and $G_{F(C),F(D)}\colon\HHom_{\Dd}(F(C),F(D))\to\HHom_{\Ee}(GF(C),GF(D))$.
\end{itemize}
Then the unity of the composition is the \bfblue{identity dg-functor} 
$\Id$ which is identity on objects and each 
cochain map $\Id_{C,D}$ is 
just the identity map.

One can then define the \bfblue{isomorphisms} of dg-categories 
as those dg-functors admits an inverse. 
It is clear that this condition is equivalent to say that
the functor $F$ is \emph{surjective on objects} and 
the cochain maps $F_{C,D}$ are chain \emph{isomorphisms}.

Any dg-functor $F\colon\Cc\to\Dd$ admits a \bfblue{underlying functor} 
$F_0\colon\Cc_0\to\Dd_0$ obtained by applying the 
\emph{change of base categories} $Z^0\colon\Ch\to\Ab$ and 
a \bfblue{homotopy functor} $\h{F}\colon\h\Cc\to\h\Dd$ 
obtained by applying the 
\emph{change of base categories} $H^0\colon\Ch\to\Ab$.
}

\para{
Given two dg-functors $F,G\colon \Cc\to \Dd$. 
A \bfblue{dg-transformation} $\alpha\colon F\then G$ consists of 
a family of cochain maps $\alpha_C\colon\ZZ\to\HHom_{\Dd}(F(C),G(C))$ 
indexed by objects of $\Cc$,
satisfying that for any objects $C$ and $D$, 
the following diagram commutes.
\[
\adjustbox{scale=0.95,center}{
\begin{tikzcd}
\ZZ\otimes\HHom_{\Cc}(C,D)\ar[r,"{\alpha_D\otimes F}"] & 
\HHom_{\Dd}(F(D),G(D))\otimes\HHom_{\Dd}(F(C),F(D))\ar[d] \\
\HHom_{\Cc}(C,D)\ar[u,"{\cong}"]\ar[d,"{\cong}"'] & 
\HHom_{\Dd}(F(C),G(D)) \\
\HHom_{\Cc}(C,D)\otimes\ZZ\ar[r,"{G\otimes\alpha_C}"] &
\HHom_{\Dd}(G(C),G(D))\otimes\HHom_{\Dd}(F(D),G(D))\ar[u]
\end{tikzcd}
}
\]

Given two dg-transformations 
$\alpha\colon F\then G$, $\beta\colon G\then H$, 
their \bfblue{vertical composition} $\beta\cdot\alpha$ is given by 
\[
\begin{tikzcd}[column sep=huge, row sep=scriptsize]
{
\begin{matrix}
\ZZ \\
\otimes \\
\ZZ
\end{matrix}
}
\ar[r,"{
\begin{matrix}
\alpha_C\\
\\
\beta_C
\end{matrix}}"description]
\ar[r,"\textstyle{\otimes}"description]
&
{
\begin{matrix}
\HHom_{\Dd}(F(C),G(C)) \\
\otimes \\
\HHom_{\Dd}(G(C),H(C))
\end{matrix}
}\ar[d]\\
{\ZZ}\ar[u,"{\cong}"]\ar[r,dashed,"{(\beta\cdot\alpha)_C}"]&
\HHom_{\Dd}(F(C),H(C)).
\end{tikzcd}
\]
The unity of this composition is the 
\bfblue{identity dg-transformation} $\id$ which gives the identity for 
each object. 
A dg-transformation $\alpha$ is called a \bfblue{natural isomorphism} 
if it admits an inverse $\beta$, i.e. 
$\alpha\cdot\beta=\id$, $\beta\cdot\alpha=\id$.
It is called a \bfblue{natural equivalence} if it admits an 
weak inverse $\beta$, i.e. 
$\alpha\cdot\beta\simeq\id$, $\beta\cdot\alpha\simeq\id$.

Given two dg-transformations 
\[
\begin{tikzcd}[column sep=large, row sep=large]
{\Cc}
\ar[r,bend left,"{F}",""{name=a,below}]
\ar[r,bend right,"{G}"',""{name=b}]
\ar[Rightarrow,from=a,to=b,"{\alpha}"]&
{\Dd}
\ar[r,bend left,"{F'}",""{name=a,below}]
\ar[r,bend right,"{G'}"',""{name=b}]
\ar[Rightarrow,from=a,to=b,"{\beta}"]&
{\Ee}
\end{tikzcd}
\]
their \bfblue{horizontal composition} $\beta\ast\alpha$ is given by
the following two equivalent compositions.
\[
\begin{tikzcd}[column sep=huge, row sep=scriptsize]
{
\begin{matrix}
\ZZ \\
\otimes \\
\ZZ
\end{matrix}
}
\ar[r,"{
\begin{matrix}
F'(\alpha_C)\\
\\
\beta_{G(C)}
\end{matrix}
}"description]
\ar[r,"\textstyle{\otimes}"description]
&
{
\begin{matrix}
\HHom_{\Ee}(F'F(C),F'G(C)) \\
\otimes \\
\HHom_{\Ee}(F'G(C),G'G(C))
\end{matrix}
}\ar[d]\\
{\ZZ}\ar[u,"{\cong}"]\ar[d,"{\cong}"']
\ar[r,dashed,"{(\beta\ast\alpha)_C}"description]&
\HHom_{\Ee}(F'F(C),G'G(C)) \\
{
\begin{matrix}
\ZZ \\
\otimes \\
\ZZ
\end{matrix}
}
\ar[r,"{
\begin{matrix}
\beta_{F(C)}\\
\\
G'(\alpha_C)
\end{matrix}
}"description]
\ar[r,"\textstyle{\otimes}"description]
&
{
\begin{matrix}
\HHom_{\Ee}(F'F(C),G'F(C)) \\
\otimes \\
\HHom_{\Ee}(G'F(C),G'G(C))
\end{matrix}
}\ar[u]
\end{tikzcd}
\]
}

\para{
The previous abstract definition can be spelled out 
elementary as follows. 
\begin{proplist}
\item 
A cochain map from $\ZZ$ to a complex $C^{\bullet}$ is the 
same as a $0$-cocycle of $C^{\bullet}$.
Hence a \bfblue{dg-transformation} $\alpha\colon F\then G$ 
is the same as a family of $1$-morphisms 
$\alpha_C\colon F(C)\to F(D)$ in $\Dd$, (hence 
morphisms in $\Dd_0$), satisfying that for any element
$f\in\HHom_{\Cc}(C,D)$, the following diagram commutes.
\[
\begin{tikzcd}
F(C)\ar{r}{\alpha_C}\ar{d}[swap]{F(f)}&
G(C)\ar{d}{G(f)}\\
F(D)\ar{r}{\alpha_D}&G(D)
\end{tikzcd}
\]
Be aware that a natural transformation $\alpha\colon F_0\then G_0$ 
requires merely above commutative diagrams for $1$-morphisms 
$f\colon C\to D$. 
\item 
Given two dg-transformations 
$\alpha\colon F\then G$, $\beta\colon G\then H$, 
their \bfblue{vertical composition} $\beta\cdot\alpha$ is given by 
the family $(\beta\cdot\alpha)_C:=\beta_C\circ\alpha_C$ viewed as 
compositions of $1$-morphisms.
\item 
The \bfblue{identity dg-transformation} $\id$ is the same as 
the family of identity morphisms $\id_{F(C)}\colon F(C)\to F(C)$. 
Hence a dg-transformation $\alpha$ is 
a \bfblue{natural isomorphism} if and only if its 
each component $\alpha_C$ is an isomorphism in $\Dd_0$, 
and a \bfblue{natural equivalence} if and only if its 
each component $\alpha_C$ is an isomorphism in $\h\Dd$. 
\item 
Given two dg-transformations 
\[
\begin{tikzcd}[column sep=large, row sep=large]
{\Cc}
\ar[r,bend left,"{F}",""{name=a,below}]
\ar[r,bend right,"{G}"',""{name=b}]
\ar[Rightarrow,from=a,to=b,"{\alpha}"]&
{\Dd}
\ar[r,bend left,"{F'}",""{name=a,below}]
\ar[r,bend right,"{G'}"',""{name=b}]
\ar[Rightarrow,from=a,to=b,"{\beta}"]&
{\Ee}
\end{tikzcd}
\]
their \bfblue{horizontal composition} $\beta\ast\alpha$ is 
the family $(\beta\ast\alpha)_C$ given by 
two equivalent compositions which can be encoded into the following 
commutative diagram of $1$-morphisms.
\[
\begin{tikzcd}
{F'F(C)}\ar{r}{F'(\alpha_C)}
\ar{d}[swap]{\beta_{F(C)}}&
{F'G(C)}\ar{d}{\beta_{G(C)}}\\
{G'F(C)}\ar{r}{G'(\alpha_C)}&{G'G(C)}
\end{tikzcd}
\]
\item 
Then one can verify that the \bfblue{interchange law} holds:
whenever we have dg-transformations
\[
\begin{tikzcd}[column sep=large, row sep=large]
{\Cc}
\ar[r,bend left=75,"{F}",""{name=a,below}]
\ar[r,,"{G}"description,""{name=b},""{name={a'},below}]
\ar[r,bend right=75,"{H}"',""{name={b'}}]
\ar[Rightarrow,from=a,to=b,"{\alpha}"]
\ar[Rightarrow,from={a'},to={b'},"{\beta}"]&
{\Dd}
\ar[r,bend left=75,"{F'}",""{name=a,below}]
\ar[r,,"{G'}"description,""{name=b},""{name={a'},below}]
\ar[r,bend right=75,"{H'}"',""{name={b'}}]
\ar[Rightarrow,from=a,to=b,"{\alpha'}"]
\ar[Rightarrow,from={a'},to={b'},"{\beta'}"]&
{\Ee}
\end{tikzcd}
\]
the following two compositions are the same:
\[
(\beta'\cdot\alpha')\ast(\beta\cdot\alpha)
\quad=\quad
(\beta'\ast\beta)\cdot(\alpha'\ast\alpha).
\]
\item 
Note that, as in ordinary category theory, 
the identity transformation $\id_F$ in a formula of dg-transformations
is usually denoted as $F$. 
For example, $F'\ast\alpha$ means $\id_{F'}\ast\alpha$, 
whose components are $F'(\alpha_C)$, 
and $\beta\ast G$ means $\beta\ast\id_{G}$, 
whose components are $\beta_{G(C)}$.
Then the interchange law tells us 
\[
(\beta\ast G)\cdot(F'\ast\alpha) = \beta\ast\alpha.
\]
Likewise, we also have
\[
(G'\ast\alpha)\cdot(\beta\ast F) = \beta\ast\alpha.
\]
Hence the fact that the two ways of horizontal composition agree 
is a special case of the interchange law.
\end{proplist}
}

\para{
Let $\Cc$ and $\Dd$ be two dg-categories. 
The category \bfblue{$\Fun(\Cc,\Dd)$} consists of 
\begin{itemize}
\item dg-functors from $\Cc$ to $\Dd$ as its objects;
\item dg-transformations between those dg-functors as its morphisms. 
\end{itemize}
The natural isomorphisms are precisely isomorphisms in this category. 
Two dg-functors are said to be \bfblue{isomorphic} 
if there is a natural isomorphism between them. 
Note that Two dg-functors $F,G\colon\Cc\to\Dd$ are isomorphic 
if and only if they are isomorphic in the category 
$\Fun(\Cc,\Dd)$.

Two dg-transformations $\alpha,\beta\colon F\then G$ are said to be 
\bfblue{homotopic} if its components $\alpha_C$ and $\beta_C$ 
are homotopic 
(as cochain maps, using the abstract definition, 
or equivalently, as $1$-morphisms in $\Dd$, 
using the elementary description).
Then the category \bfblue{$\hFun(\Cc,\Dd)$} consists of 
\begin{itemize}
\item dg-functors from $\Cc$ to $\Dd$ as its objects;
\item homotopy classes of dg-transformations as its morphisms. 
\end{itemize}
The natural equivalences are precisely isomorphisms in this category. 
Two dg-functors are said to be \bfblue{equivalent}
if there is a natural equivalence between them. 
Note that Two dg-functors $F,G\colon\Cc\to\Dd$ are \bfblue{equivalent}
if and only if they are isomorphic in the category $\hFun(\Cc,\Dd)$.

As the notations suggest, the above ordinary categories 
$\Fun(\Cc,\Dd)$ and $\hFun(\Cc,\Dd)$ should be viewed as 
the underlying category and the homotopy category of the  
dg-category of dg-functors $\FFun(\Cc,\Dd)$ respectively.
This dg-category will be constructed later.
}

\para{
An \bfblue{adjunction of dg-functors} is a quadruple 
$(F,G,\eta,\epsilon)$, where
\begin{itemize}
\item 
$F\colon\Cc\to\Dd$ (the \bfblue{left adjoint}) and 
$G\colon\Dd\to\Cc$ (the \bfblue{right adjoint}) 
are two dg-functors,
\item 
$\eta\colon\Id_{\Cc}\then G\circ F$ (the \bfblue{unit}) and 
$\epsilon\colon F\circ G\then\Id_{\Dd}$ (the \bfblue{counit}) 
are two dg-transformations,
\end{itemize}
satisfying the following two commutative diagram 
(the \bfblue{triangle identities})
of dg-transformations.
\[
\begin{tikzcd}[column sep=0pt,row sep=large]
&F\circ G\circ F\ar{dr}{\epsilon\ast F}&\\
F\ar{ur}{F\ast\eta}\ar{rr}{\id_F}&&F
\end{tikzcd}
\qquad\qquad
\begin{tikzcd}[column sep=0pt,row sep=large]
&G\circ F\circ G\ar{dr}{G\ast\epsilon}&\\
G\ar{ur}{\eta\ast G}\ar{rr}{\id_G}&&G
\end{tikzcd}
\]

If this is the case, then the compositions 
(of cochain maps)
\[
\begin{tikzcd}[column sep=huge, row sep=scriptsize]
{
\begin{matrix}
\ZZ \\
\otimes \\
{\HHom_{\Dd}(F(C),D)}
\end{matrix}
}
\ar[r,"{
\begin{matrix}
\eta_C\\
\\
G
\end{matrix}
}"description]
\ar[r,"\textstyle{\otimes}"description]
&
{
\begin{matrix}
\HHom_{\Cc}(C,GF(C)) \\
\otimes \\
\HHom_{\Cc}(GF(C),G(D))
\end{matrix}
}\ar[d]\\
{\HHom_{\Dd}(F(C),D)}
\ar[u,"{\cong}"]\ar[r,dashed,"{\alpha_{C,D}}"]&
\HHom_{\Cc}(C,G(D))
\end{tikzcd}
\]
and
\[
\begin{tikzcd}[column sep=huge, row sep=scriptsize]
{
\begin{matrix}
{\HHom_{\Cc}(C,G(D))} \\
\otimes \\
{\ZZ}
\end{matrix}
}
\ar[r,"{
\begin{matrix}
F\\
\\
\epsilon_D
\end{matrix}
}"description]
\ar[r,"\textstyle{\otimes}"description]
&
{
\begin{matrix}
\HHom_{\Dd}(F(C),FG(D)) \\
\otimes \\
\HHom_{\Dd}(FG(D),D)
\end{matrix}
}\ar[d]\\
{\HHom_{\Cc}(C,G(D))}
\ar[u,"{\cong}"]\ar[r,dashed,"{\beta_{C,D}}"]&
{\HHom_{\Dd}(F(C),D)}
\end{tikzcd}
\]
give rise to a pair of natural isomorphisms
\[
\HHom_{\Dd}(F(-),-)\cong\HHom_{\Cc}(-,G(-)).
\]

Conversely, if there is a pair of natural isomorphisms
\[
\begin{tikzcd}
\HHom_{\Dd}(F(-),-)
\ar[r,shift left,"{\alpha}"]&
\ar[l,shift left,"{\beta}"]
\HHom_{\Cc}(-,G(-)),
\end{tikzcd}
\]
then the compositions
\[
\ZZ\to\HHom_{\Dd}(F(C),F(C))
\markar{\alpha_{C,F(C)}}
\HHom_{\Cc}(C,GF(C))
\]
and
\[
\ZZ\To\HHom_{\Cc}(G(D),G(D))
\markar{\beta_{G(D),D}}
\HHom_{\Dd}(FG(D),D)
\]
give rise to the unit $\eta$ and the counit $\eta$ making the 
quadruple $(F,G,\eta,\epsilon)$ an adjunction of dg-functors.
}

\para{
A dg-functor $F\colon\Cc\to\Dd$ is said to be 
\begin{itemize}
\item 
\bfblue{fully faithful}, 
if for any two objects $C,D$ of $\Cc$, the cochain map
\[
F_{C,D}\colon\HHom_{\Cc}(C,D)\to\HHom_{\Dd}(F(C),F(D))
\]
is an isomorphism;
\item 
\bfblue{homotopically fully faithful}, 
if for any two objects $C,D$ of $\Cc$, the cochain map
\[
F_{C,D}\colon\HHom_{\Cc}(C,D)\to\HHom_{\Dd}(F(C),F(D))
\]
is a homotopy equivalence;
\item 
\bfblue{essentially surjective}, 
if the {underlying functor} of it
$F_0\colon\Cc_0\to\Dd_0$ is essentially surjective;
\item 
\bfblue{homotopically essentially surjective},  
if the {homotopy functor} of it
$\h F\colon\h\Cc\to\h\Dd$ is essentially surjective;
\item 
an \bfblue{equivalence}, 
if there is another dg-functor $G\colon\Dd\to\Cc$ such that 
$F\circ G\cong\Id_{\Dd}$ and $\Id_{\Cc}\cong G\circ F$ 
($\cong$ denotes natural isomorphism);
\item 
a \bfblue{homotopically equivalence}, 
if there exists another dg-functor $G\colon\Dd\to\Cc$ such that 
$F\circ G\simeq\Id_{\Dd}$ and $\Id_{\Cc}\simeq G\circ F$
($\simeq$ denotes natural equivalence);
\item 
an \bfblue{adjoint equivalence}, 
if it admits a right adjoint $G$ such that the unit 
$\eta\colon\Id_{\Cc}\then G\circ F$ and the counit
$\epsilon\colon F\circ G\then\Id_{\Dd}$ are natural isomorphisms;
\item 
an \bfblue{adjoint homotopically equivalence}, 
if it admits a right adjoint $G$ such that the unit 
$\eta\colon\Id_{\Cc}\then G\circ F$ and the counit
$\epsilon\colon F\circ G\then\Id_{\Dd}$ are natural equivalence.
\end{itemize}
}

%As in ordinary category theory, assuming axiom of choices, 
%we have the following proposition.
\begin{proposition}\label{prop:equiv_of_dg_cats}
Let $F\colon\Cc\to\Dd$ be a dg-functor. 
Then the followings are equivalent.
\begin{proplist}
\item 
$F$ is fully faithful and essentially surjective.
\item 
$F$ is an equivalence.
\item 
$F$ is an adjoint equivalence.
\end{proplist}
Moreover, the followings are equivalent.
\begin{proplist}[resume]
\item 
$F$ is homotopically fully faithful and 
homotopically essentially surjective.
\item 
$F$ is an homotopically equivalence.
\item 
$F$ is an adjoint homotopically equivalence.
\end{proplist}
\end{proposition}
\begin{proof}
Suppose (ii), let's prove (iii). 
To do this, we need a lemma.
\begin{sublemma}
Let $\alpha\colon F\then \Id$ be a natural isomorphism between 
dg-endofunctors. Then we have 
\[
(F\ast\alpha)\cdot(\alpha^{-1}\ast F)=
(\alpha\ast F)\cdot(F\ast\alpha^{-1})=\id_F.
\]
\end{sublemma}
\begin{proof}
The result follows from the following commutative diagrams.
\[
\begin{tikzcd}
{F(-)}\ar{d}[swap]{\alpha_{(-)}}
\ar{r}{\alpha^{-1}_{F(-)}}&
{FF(-)}\ar{d}{F(\alpha_{(-)})}\\
{(-)}\ar{r}{\alpha^{-1}_{(-)}}&{F(-)}
\end{tikzcd}
\qquad
\begin{tikzcd}
{F(-)}\ar{d}[swap]{F(\alpha^{-1}_{(-)})}
\ar{r}{\alpha_{(-)}}&
{(-)}\ar{d}{\alpha^{-1}_{(-)}}\\
{FF(-)}\ar{r}{\alpha_{F(-)}}&{F(-)}
\end{tikzcd}
\]
\end{proof}

Let $G\colon\Dd\to\Cc$ be the inverse of $F$ with natural isomorphisms 
$\eta\colon\Id_{\Cc}\Isom{G\circ F}$ and 
$\varepsilon\colon{F\circ G}\Isom\Id_{\Dd}$. 
Then, let $\epsilon\colon{F\circ G}\then\Id_{\Dd}$ be the composition
\[
F\circ G\markar{F\circ G\ast\varepsilon^{-1}}
F\circ G\circ F\circ G\markar{F\ast\eta^{-1}\ast G}
F\circ G\markar{\varepsilon}
\Id_{\Dd}.
\]
Then $\epsilon$ is a natural isomorphism and by 
the following commutative diagrams
\[
\begin{tikzcd}[column sep=large,row sep=large]
F(-)\ar{d}[swap]{F(\eta_{(-)})}\ar{r}{\varepsilon^{-1}_{F(-)}}&
FGF(-)\ar{d}[swap]{FGF(\eta_{(-)})}\ar[dr,dashed,equal]&
F(-)\\
FGF(-)\ar{dr}[swap]{FG(\varepsilon^{-1}_{F(-)})}&
FGFGF(-)\ar{l}[swap]{\varepsilon_{FGF(-)}}\ar[d,dashed,equal]&
FGF(-)\ar{u}[swap]{\varepsilon_{F(-)}}\\&
FGFGF(-)\ar{ur}[swap]{F(\eta^{-1}_{GF(-)})}&
\end{tikzcd}
\]
(where the dashed identity transformations come from the lemma)
and
\[
\begin{tikzcd}[column sep=large,row sep=large]
G(-)\ar{r}{\eta_{G(-)}}\ar{d}[swap]{G(\varepsilon^{-1}_{(-)})} &
GFG(-)\ar{d}{GFG(\varepsilon^{-1}_{(-)})}\\
GFG(-)\ar{r}{\eta_{GFG(-)}}\ar[dr,dashed,equal]&
GFGFG(-)\ar{d}{GF(\eta^{-1}_{G(-)})}\\
G(-)&
GFG(-)\ar{l}{G(\varepsilon_{(-)})}
\end{tikzcd}
\]
(where the dashed identity transformation comes from the lemma), 
the quadruple $(F,G,\eta,\epsilon)$ is an adjunction of dg-functors.

Now, suppose (iii), let's prove (i). 
First, $F$ is essentially surjective. 
Indeed, for each object $D$ of $\Dd$, 
the $1$-morphism $\epsilon_D\colon FG(D)\to D$ gives the 
desired isomorphism.

To show $F$ is fully faithful, 
consider the following composition 
which gives the inverse of the cochain map $F_{C,D}$.
\[
\begin{tikzcd}[column sep=huge, row sep=scriptsize]
{
\begin{matrix}
\ZZ \\
\otimes \\
{\HHom_{\Dd}(F(C),F(D))}\\
\otimes \\
\ZZ
\end{matrix}
}
\ar[r,"{
\begin{matrix}
\eta_C\\
\textstyle{\otimes}\\
G\\
\textstyle{\otimes}\\
\eta^{-1}_D
\end{matrix}
}"description]
&
{
\begin{matrix}
\HHom_{\Cc}(C,GF(C)) \\
\otimes \\
\HHom_{\Cc}(GF(C),GF(D)) \\
\otimes \\
\HHom_{\Cc}(GF(D),D)
\end{matrix}
}\ar[d]\\
{\HHom_{\Dd}(F(C),F(D))}
\ar[u,"{\cong}"]\ar[r,dashed,"{\alpha_{C,D}}"]&
\HHom_{\Cc}(C,D)
\end{tikzcd}
\]
Indeed, $\alpha_{C,D}\circ F_{C,D}=\id$ follows from the following 
commutative diagram
\[
\begin{tikzcd}
{\HHom_{\Dd}(F(C),F(D))}\ar{r}{G}&
{\HHom_{\Cc}(GF(C),GF(D))}\ar{d}{\circ\eta_C}\\
{\HHom_{\Cc}(C,D)}\ar{u}{F}\ar{r}{\eta_D\circ}&
{\HHom_{\Cc}(C,GF(D))}
\end{tikzcd}
\]
and $F_{C,D}\circ \alpha_{C,D}=\id$ follows from the following 
commutative diagram
\[
\adjustbox{scale=0.75,center}{
\begin{tikzcd}[row sep=large]
{\HHom_{\Dd}(F(C),F(D))}\ar[blue]{d}{G}
\ar[ddd,dotted,bend right=90,"{\circ\epsilon_{F(C)}}"]\\
{\HHom_{\Cc}(GF(C),GF(D))}
\ar[blue]{r}{\circ\eta_C}\ar{d}{F}&
{\HHom_{\Cc}(C,GF(D))}
\ar{d}{F}\ar[blue]{r}{\eta_D^{-1}\circ}&
{\HHom_{\Cc}(C,D)}\ar[blue]{d}{F}\\
{\HHom_{\Dd}(FGF(C),FGF(D))}
\ar{r}{\circ F(\eta_C)}\ar{d}{\epsilon_{F(D)}\circ}&
{\HHom_{\Dd}(F(C),FGF(D))}
\ar{r}{F(\eta_D^{-1})\circ}\ar{d}{\epsilon_{F(D)}\circ}&
{\HHom_{\Dd}(F(C),F(D))}\\
{\HHom_{\Dd}(FGF(C),F(D))}\ar[dotted]{r}{\circ F(\eta_C)}&
{\HHom_{\Dd}(F(C),F(D))}\ar[ur,bend right=15,dashed,equal]
\end{tikzcd}
}
\]
where the dashed identity as well as that 
the composition of dotted arrow is identity 
follows from the triangle identities, 
and the blue arrows emphasize the desired composition.

Next, suppose (i), let's prove (ii). 
First, let's construct the dg-functor $G\colon\Dd\to\Cc$.
\begin{enumerate}
\item 
For any object $D$ of $\Dd$, CHOOSE an object $C$ of $\Cc$ 
such that $F(C)\cong D$. Then put $G(D)=C$ and denote 
this isomorphism by $\epsilon_D$.
\item 
For any pair of objects $D,D'$ of $\Dd$, 
the cochain map $G_{D,D'}$ 
is given by the composition:
\[
\adjustbox{scale=0.85,center}{
\begin{tikzcd}
\HHom_{\Dd}(D,D')
\ar[r,"{
\begin{matrix}
\circ\epsilon_D\\
\\
\epsilon_{D'}^{-1}\circ
\end{matrix}
}"description]
\ar[r,"\textstyle{\circ}"description]&
\HHom_{\Dd}(FG(D),FG(D'))\ar{r}{F^{-1}_{C,D}}&
\HHom_{\Cc}(G(D),G(D')).
\end{tikzcd}
}
\]
\item 
Then $G$ is a dg-functor by straightforward verification using elements.
\item 
Now $\epsilon$ form a dg-transformation by the construction of $G$ 
and it is clear a natural isomorphism.
\item 
For each object $C$ of $\Cc$, define 
$\eta_C\colon C\to GF(C)$ as the preimage of 
$\epsilon^{-1}_{F(C)}\colon F(C)\to FGF(C)$ 
under the cochain map 
\[
F_{C,GF(C)}\colon\HHom_{\Cc}(C,GF(C))\To\HHom_{\Dd}(F(C),FGF(C)).
\]
\item 
Now $\eta$ form another dg-transformation since 
$\epsilon^{-1}$ is a dg-transformation and $F$ is fully faithful. 
It is clear that $\eta$ is a natural isomorphism.
\end{enumerate}

Finally, the proofs of (iv) implies (v) implies (vi) implies (iv) 
are similar as above, but:
\begin{enumerate}
\item 
instead of working in the category $\Fun(\Cc,\Dd)$,
one works in the category $\hFun(\Cc,\Dd)$;
\item 
instead of using inverse dg-transformations, 
one needs to use weak inverse;
\item 
instead of CHOOSE isomorphisms, 
one has to CHOOSE homotopy equivalences.
\end{enumerate}
\end{proof}
\begin{Rem}
It would be helpful if one is familiar with 
the proof of similar statements in ordinary category theory. 
One can also try to use elements to drop above proof down to earth.
\end{Rem}

\para{
Let $\Cc$ and $\Dd$ be two dg-categories, then their 
\bfblue{(tensor) product} is the dg-category $\Cc\otimes\Dd$ 
\begin{itemize}
\item 
whose collection of objects is the product $\ob\Cc\times\ob\Dd$, 
\item 
each Hom object $\HHom_{\Cc\otimes\Dd}((C,D),(C',D'))$ is 
the tensor product 
\[
\HHom_{\Cc}(C,C')\otimes\HHom_{\Dd}(D,D'),
\] 
\item 
the composition rule and the unity is given 
by obverse constructions.
\end{itemize}
One should think $\Cc\otimes\Dd$ as a dg-version of $\Cc\times\Dd$. 
This tensor product has a unit $\unit$, which is the dg-category 
having one object $\ast$ with the Hom-complex 
$\HHom_{\unit}(\ast,\ast)=\ZZ$, and 
is symmetric with the braiding given by 
the braiding $\gamma$ of $\Ch$. 
In this way, the $2$-category $\dg\Cat$ 
becomes a tensor $2$-category.

The \bfblue{opposite dg-category} of 
a dg-category $\Cc$, 
denoted by $\Cc^{\opp}$, is the dg-category with 
the same objects as $\Cc$ and Hom-complex 
\[
\HHom_{\Cc^{\opp}}(C,D):=\HHom_{\Cc}(D,C).
\]
Clearly, $(\Cc^{\opp})^{\opp} = \Cc$ 
and $(\Cc\otimes\Dd)^{\opp}=\Cc^{\opp}\otimes\Dd^{\opp}$. 
One should think $\Cc^{\opp}$ as a dg-version 
of opposite category $\Cc^{\opp}$.
In this way, $\dg\Cat$ becomes a 
tensor $2$-category with an involution $(-)^{\opp}$.
}

\para{
The change of base categories $Z^0$ and $H^0$ 
gives rise to two $2$-functors between $2$-categories 
$\dg\Cat\to\Cat$ (more precisely, 
they land in the $2$-category of 
pre-additive categories). 
It is clear that $(\Cc^{\opp})_0 = \Cc^{\opp}_0$ 
and that $\h(\Cc^{\opp}) = (\h\Cc)^{\opp}$. 
So they respect the involution structures. 
However, in general, 
$(\Cc\otimes\Dd)_0$ is not isomorphic to 
$\Cc_0\times\Dd_0$ (as product of categories) or 
$\Cc_0\otimes\Dd_0$ (as product of pre-additive categories). 
Similarly for $\h(\Cc\otimes\Dd)$. 

However, there is a canonical additive functor
\begin{align*}
\Cc_0\otimes\Dd_0\To(\Cc\otimes\Dd)_0,\\
\h\Cc\otimes\h\Dd\To\h(\Cc\otimes\Dd).
\end{align*}
Hence both $Z^0$ and $H^0$ induce 
lax $2$-functors between tensor $2$-categories. 
This is because $Z^0$ and $H^0$ 
are lax functors, i.e. for any complexes 
$C^{\bullet}$ and $D^{\bullet}$, there are 
homomorphisms
\[
Z^0(C)\otimes Z^0(D) \To Z^0(C\otimes D),\qquad
H^0(C)\otimes H^0(D) \To H^0(C\otimes D).
\]
Indeed, they are just the homomorphisms 
induced by the inclusion 
\[
C^0\otimes D^0\To (C\otimes D)^0.
\]

}

\para{
A \bfblue{dg-bifunctor} is simply a dg-functor 
from a product $\Cc\otimes\Dd$. One should think it as the 
dg-version of bifunctor. 

As in ordinary case, given two dg-functors $F\colon\Cc\to\Dd$ 
and $G\colon\Cc'\to\Dd'$, 
we always have the dg-functor 
\[
F\otimes G\colon \Cc\otimes\Cc'\To\Dd\otimes\Dd'.
\]
One should think it as the dg-version of $F\times G$.

As in ordinary case, any dg-bifunctor 
$T\colon\Cc\otimes\Dd\to\Ee$ 
induces \bfblue{partial functors} 
$T(C,-)\colon\Dd\to\Ee$ and 
$T(-,D)\colon\Cc\to\Ee$ by evaluating $T$ 
at objects $C$ of $\Cc$ and 
$D$ of $\Dd$ respectively. 
In other words, $T(C,-)$ is precisely the composition 
\[
\Dd\cong\unit\otimes\Dd
\markar{C\otimes\Id_{\Dd}}
\Cc\otimes\Dd\markar{T}\Ee,
\]
and $T(-,D)$ is the composition 
\[
\Cc\cong\Cc\otimes\unit
\markar{\Id_{\Cc}\otimes D}
\Cc\otimes\Dd\markar{T}\Ee.
\]

Conversely, if we have two families of dg-functors 
$\{F_C\colon\Dd\to\Ee\}_{C\in\ob\Cc}$ and 
$\{G_D\colon\Cc\to\Ee\}_{D\in\ob\Dd}$ 
such that 
\[
F_C(D)=G_D(C)
\]
for any $C\in\ob\Cc$, $D\in\ob\Dd$. 
Then we can put $T(C,D)=F_C(D)=G_D(C)$. 
To make this a dg-functor, it remains to define cochain maps
\[
\HHom_{\Cc}(C,C')\otimes\HHom_{\Dd}(D,D')\To\HHom_{\Ee}(T(C,D),T(C',D')).
\]
There are two way to define it:
\[
\begin{tikzcd}[column sep=huge, row sep=scriptsize]
{
\begin{matrix}
\HHom_{\Dd}(D,D')\\
\otimes\\
\HHom_{\Cc}(C,C')
\end{matrix}
}
\ar[r,"{
\begin{matrix}
F_{C}\\
\\
G_{D'}
\end{matrix}
}"description]
\ar[r,"\textstyle{\otimes}"description]
\ar[dd,leftrightarrow,"{\cong}"description]
&
{
\begin{matrix}
\HHom_{\Ee}(T(C,D),T(C,D')) \\
\otimes \\
\HHom_{\Ee}(T(C,D'),T(C',D'))
\end{matrix}
}\ar[d]\\
&\HHom_{\Ee}(T(C,D),T(C',D')) \\
{
\begin{matrix}
\HHom_{\Cc}(C,C')\\
\otimes\\
\HHom_{\Dd}(D,D')
\end{matrix}
}
\ar[r,"{
\begin{matrix}
G_D\\
\otimes\\
F_{C'}
\end{matrix}
}"description]
\ar[r,"\textstyle{\otimes}"description]
&
{
\begin{matrix}
\HHom_{\Ee}(T(C,D),T(C',D)) \\
\otimes \\
\HHom_{\Ee}(T(C',D),T(C',D'))
\end{matrix}
}\ar[u]
\end{tikzcd}
\]
Hence the two families defines a dg-bifunctor $T$ such that 
$T(C,-)=F_C$ and $T(-,D)=G_D$ if and only if the above diagram commutes.

Given two dg-bifunctors $T,S\colon \Cc\otimes\Dd\to\Ee$.
Using above characterization of dg-bifunctors, 
it is easy to show that 
a family of $1$-morphisms 
$\{\alpha_{C,D}\}_{(C,D)\in\ob(\Cc\otimes\Dd)}$ 
forms a dg-transformation from $T$ to $S$ if and only if 
\begin{enumerate}
\item 
for any $C\in\Cc$, 
the family $\{\alpha_{C,D}\}_{D\in\ob\Dd}$ 
forms a dg-transformation from $T(C,-)$ to $S(C,-)$, 
and 
\item
for any $D\in\Dd$, 
the family $\{\alpha_{C,D}\}_{C\in\ob\Cc}$ 
forms a dg-transformation from $T(-,D)$ to $S(-,D)$.
\end{enumerate}
In other words, dg-transformation can be verified 
variable by variable.
}
\begin{Rem}
On can verify that: 
if $T\colon\Cc\otimes\Dd\To\Ee$ is a dg-bifunctor, 
then the partial functors of the bifunctor
\[
\Cc_0\times\Dd_0\To(\Cc\otimes\Dd)_0\markar{T_0}\Ee_0
\]
are precisely the underlying functors of 
the partial functors of $T$.
\end{Rem}

\begin{example}
For any dg-category $\Cc$, 
there is a natural \emph{dg-bifunctor} 
\[
\HHom_{\Cc}(-,-)\colon\Cc^{\opp}\otimes\Cc\To\Ch.
\]
To see this, it suffices to give the canonical cochain maps
\[
\HHom_{\Cc^{\opp}\otimes\Cc}((C,D),(C',D'))
\To
[\HHom_{\Cc}(C,D),\HHom_{\Cc}(C',D')].
\]
But this is just the adjunct of 
\[
\HHom_{\Cc}(C',C)\otimes\HHom_{\Cc}(D,D')\otimes\HHom_{\Cc}(C,D)
\To
\HHom_{\Cc}(C',D').
\]

In this way, any dg-functor $F\colon\Cc\to\Dd$ gives 
a dg-transformation 
\[
F_{-,-}\colon\HHom_{\Cc}(-,-)\To\HHom_{\Dd}(F(-),F(-)).
\]
\end{example}

\para{
Consider the category $\Fun(\Cc,\Dd)$ 
of dg-functors from a small dg-category $\Cc$ 
to another dg-category $\Dd$. 
To enhance it into a dg-category, notice that 
for any two dg-functors $F$ and $G$ and 
any pair of objects $(C,D)$ in $\Cc$, there is a Hom-complex 
\[
\HHom_{\Dd}(F(C),G(D)).
\]
Hence the Hom-complex $\HHom_{\Fun(\Cc,\Dd)}(F,G)$ has to be certain 
universal construction from them.

Note that, the condition for a family $\{\alpha_C\}_{C\in\ob\Cc}$ 
from a dg-transformation from $F$ to $G$ 
can be translated into the following diagram 
\[
\begin{tikzcd}
\ZZ\ar{d}[swap]{\alpha_C}\ar{r}{\alpha_D}&
\HHom_{\Dd}(F(D),G(D))\ar{d}{\rho_{C,D}}\\
\HHom_{\Dd}(F(C),G(C))\ar{r}{\lambda_{C,D}}&
{[\HHom_{\Cc}(C,D),\HHom_{\Dd}(F(C),G(D))]}
\end{tikzcd}
\]
where $\lambda_{C,D}$ is given by the adjunct of
\[
\HHom_{\Cc}(C,D)\otimes\HHom_{\Dd}(F(C),G(C))\To
\HHom_{\Dd}(F(C),G(D)),
\]
which is the adjunct of
\[
\HHom_{\Cc}(C,D)\markar{\HHom_{\Dd}(F(C),G(-))}
[\HHom_{\Dd}(F(C),G(C)),\HHom_{\Dd}(F(C),G(D))];
\]
similarly, $\rho_{C,D}$ is given by the adjunct of
\[
\HHom_{\Dd}(F(D),G(D))\otimes\HHom_{\Cc}(C,D)\To
\HHom_{\Dd}(F(C),G(D)),
\]
which is the adjunct of
\[
\HHom_{\Cc}(C,D)\markar{\HHom_{\Dd}(F(-),G(D))}
[\HHom_{\Dd}(F(D),G(D)),\HHom_{\Dd}(F(C),G(D))].
\]

Inspired by this, for $T(-,-)\colon\Cc^{\opp}\otimes\Cc\to\Ch$ 
a bifunctor (for instant, $T(-,-)=\HHom_{\Dd}(F(-),G(-))$), 
an \bfblue{extraordinary naturality} of $T$ 
is a family of cochain maps $\{\alpha_C\}_{C\in\ob\Cc}$ fitting 
the following commutative diagram
\[
\begin{tikzcd}
X\ar{d}[swap]{\alpha_C}\ar{r}{\alpha_D}&
T(D,D)\ar{d}{\rho_{C,D}}\\
T(C,C)\ar{r}{\lambda_{C,D}}&
{[\HHom_{\Cc}(C,D),T(C,D)]}
\end{tikzcd}
\]
where $\lambda_{C,D}$ is given by the functor
$T(C,-)$ and $\rho_{C,D}$ by $T(-,D)$. 
Then the \bfblue{end} of $T$ is the universal extraordinary naturality 
of $T$. The complex representing the end is 
denoted by $\int_{C\in\Cc}T(C,C)$ and the canonical cochain maps 
$\pi_C\colon\int_{C\in\Cc}T(C,C)\to T(C,C)$ is called the 
\bfblue{counit} at $C$.

One should notice that an extraordinary naturality is nothing than 
a cone over the diagram consisting of $\lambda$ and $\rho$. 
Hence, the universal extraordinary naturality 
is the limit of this diagram. 
Therefore we have the following equalizer diagram.
\[
\begin{tikzcd}
{\displaystyle \int_{C\in\Cc}T(C,C)}
\ar{r}{\alpha}&
{\displaystyle \prod_{C\in\Cc}T(C,C)}
\ar[shift left]{r}{\lambda}
\ar[shift right]{r}[swap]{\rho}&
{\displaystyle \prod_{C,D\in\Cc}[\HHom_{\Cc}(C,D),T(C,D)]}.
\end{tikzcd}
\]

With the notion of ends, we can enhance $\Fun(\Cc,\Dd)$ 
into a dg-category $\FFun(\Cc,\Dd)$ as follows. 
\begin{proplist}
\item 
The Hom-complex is 
\[
\HHom_{\FFun(\Cc,\Dd)}(F,G):=
\int_{C\in\Cc}\HHom_{\Dd}(F(C),G(C)).
\]
\item 
The composition rule is given by the dashed arrow in
the following commutative diagrams 
(with $C$ goes through all the objects of $\Cc$)
uniquely determined by the universal property of end.
\[
\begin{tikzcd}[column sep=huge, row sep=scriptsize]
{
\begin{matrix}
\HHom_{\FFun(\Cc,\Dd)}(F,G) \\
\otimes \\
\HHom_{\FFun(\Cc,\Dd)}(G,H)
\end{matrix}
}
\ar[d,dashed]
\ar[r,"{
\begin{matrix}
\pi_C\\
\\
\pi_C
\end{matrix}}"description]
\ar[r,"\textstyle{\otimes}"description]
&
{
\begin{matrix}
\HHom_{\Dd}(F(C),G(C)) \\
\otimes \\
\HHom_{\Dd}(G(C),H(C))
\end{matrix}
}\ar[d]\\
\HHom_{\FFun(\Cc,\Dd)}(F,H)
\ar{r}{\pi_C}
&\HHom_{\Dd}(F(C),H(C))
\end{tikzcd}
\]
\item 
The identity is determined similarly, 
which turns out to be the \emph{identity dg-transformation}.
\end{proplist}
}

\begin{proposition}
The underlying category of $\FFun(\Cc,\Dd)$ is 
$\Fun(\Cc,\Dd)$ while the homotopy category is 
$\hFun(\Cc,\Dd)$.
\end{proposition}
\begin{proof}
First we have
\begin{align*}
Z^0\HHom_{\FFun(\Cc,\Dd)}(F,G) &= 
\Hom_{\Ch}\big(\ZZ,\HHom_{\Fun(\Cc,\Dd)}(F,G)\big),\\
H^0\HHom_{\FFun(\Cc,\Dd)}(F,G) &= 
\Hom_{K(\Ab)}\big(\ZZ,\HHom_{\FFun(\Cc,\Dd)}(F,G)\big).
\end{align*}
By the universal property of 
$\int_{C\in\Cc}\HHom_{\Dd}(F(C),G(C))$, 
a cochain map from $\ZZ$ to $\HHom_{\FFun(\Cc,\Dd)}(F,G)$ 
is equivalent to a family of cochain maps $\{\alpha_C\}$ 
fitting the commutative diagrams
\[
\begin{tikzcd}
\ZZ\ar{d}[swap]{\alpha_C}\ar{r}{\alpha_D}&
\HHom_{\Dd}(F(D),G(D))\ar{d}{\rho_{C,D}}\\
\HHom_{\Dd}(F(C),G(C))\ar{r}{\lambda_{C,D}}&
{[\HHom_{\Cc}(C,D),\HHom_{\Dd}(F(C),G(D))]}
\end{tikzcd}
\]
hence a dg-transformation from $F$ to $G$. 
Moreover, a cochain homotopy between such cochain maps 
is equivalent to a family of cochain homotopies between 
the components of the two families, hence 
a homotopy between dg-transformations.
\end{proof}

\begin{proposition}
We have natural equivalences of categories
\[
\Fun(\Cc\otimes\Dd,\Ee)\cong\Fun(\Cc,\FFun(\Dd,\Ee))
\]
for any dg-categories $\Cc$, $\Dd$ and $\Ee$ with $\Dd$ small. 
Moreover, if both $\Cc$ and $\Dd$ are small, we have 
natural equivalences of dg-categories
\[
\FFun(\Cc\otimes\Dd,\Ee)\cong\FFun(\Cc,\FFun(\Dd,\Ee)).
\]
\end{proposition}
\begin{proof}
We'll construct the an adjunction 
of transformations as follows.
\begin{enumerate}
\item 
The unit 
$\eta\colon\Id\then\FFun(\Dd,-\otimes\Dd)$
and 
\item 
the evaluation 
$\ev\colon\FFun(\Dd,-)\otimes\Dd\then\Id$.
\end{enumerate}
Then natural equivalence are given by the pair
\begin{align*}
\alpha\colon 
\FFun(\Cc\otimes\Dd,\Ee) & \longrightarrow 
\FFun(\Cc,\FFun(\Dd,\Ee)) \\
F & \longmapsto \widehat{F}, \\
\beta\colon 
\FFun(\Cc,\FFun(\Dd,\Ee)) & \longrightarrow 
\FFun(\Cc\otimes\Dd,\Ee) \\
G & \longmapsto \widetilde{G}.
\end{align*}
Where $\widehat{F}$ is the composition
\[
\Cc\markar{\eta_{\Cc}}
\FFun(\Dd,\Cc\otimes\Dd)
\markar{\FFun(\Dd,F)}
\FFun(\Dd,\Ee)
\]
and $\widetilde{G}$ is the composition
\[
\Cc\otimes\Dd
\markar{G\otimes\Dd}
\Fun(\Dd,\Ee)\otimes\Dd
\markar{\ev_{\Ee}}.
\Ee
\]

The components of $\eta$ are dg-functors
\[
\eta_{\Cc}\colon\Cc\To\FFun(\Dd,\Cc\otimes\Dd)
\]
which takes an object $C$ of $\Cc$ to the dg-functor 
\[
\widehat{C}\colon\Dd\To\Cc\otimes\Dd
\]
which takes an object $D$ of $\Dd$ to the object
\[
(C,D)\in\ob(\Cc\otimes\Dd).
\]
For any $D,D'\in\ob\Dd$, the map 
\[
\widehat{C}_{D,D'}\colon
\HHom_{\Dd}(D,D')\To
\HHom_{\Cc}(C,C)\otimes\HHom_{\Dd}(D,D')
\]
is given by $g\mapsto\id_C\otimes{g}$.
For any $C,C'\in\ob\Cc$, the map
\[
\eta_{\Cc,C,C'}\colon
\HHom_{\Cc}(C,C')\To
\int_{D\in\Dd}\HHom_{\Cc}(C,C')\otimes\HHom_{\Dd}(D,D)
\]
is induced by $f\mapsto{f}\otimes\id_D$.

The components of $\ev$ are dg-functors
\[
\ev_{\Cc}\colon\FFun(\Dd,\Cc)\otimes\Dd\To\Cc
\]
which takes a pair of a dg-functor $F\colon\Dd\to\Cc$ 
and an object $D$ of $\Dd$ 
to the object $F(D)$ of $\Cc$. 
For any pairs $(F_1,D_1)$ and $(F_2,D_2)$ 
the map 
\[
\int_{D\in\Dd}\HHom_{\Cc}(F_1(D),F_2(D))
\otimes\HHom_{\Dd}(D_1,D_2)
\markar{\ev_{\Cc}}\HHom_{\Cc}(F_1(D_1),F_2(D_2))
\]
is induced by the map
\[
\int_{D\in\Dd}\HHom_{\Cc}(F_1(D),F_2(D))
\To
[\HHom_{\Dd}(D_1,D_2),\HHom_{\Cc}(F_1(D_1),F_2(D_2))]
\]
which is precisely the commutative diagram in 
the definition of the end
$\int_{D\in\Dd}\HHom_{\Cc}(F_1(D),F_2(D))$.

Once finish above setup, it is straightforward 
to verify that the pair $(\eta,\ev)$ exhibits the 
adjunction $-\otimes\Dd\dashv\FFun(\Dd,-)$ 
and then the fact that $(\alpha,\beta)$ form 
an equivalence of dg-categories follows.
\end{proof}

\para{
Let $\Cc$ be a dg-category. 
A \bfblue{dg-module} over $\Cc$ is 
a dg-functor $M\colon\Cc^{\opp}\to\Ch$. 
In other words, it consists of 
\begin{itemize}
\item 
complexes $M(C)$ for each $C\in\ob\Cc$,
\item 
cochain maps 
\[
M_{C,D}\colon\HHom_{\Cc}(C,D)\otimes M(D) \To M(C),
\]
\end{itemize}
and satisfies certain axioms looks like those for modules.

Let's denote the category of dg-modules over $\Cc$ by 
$\dg(\Cc)$. 
For any object $C$ of $\Cc$, 
the construction $\HHom_{\Cc}(-,C)$ 
can be made into a dg-module $\Upsilon(C)$ by 
letting the cochain map $\Upsilon(C)_{C_1,C_2}$ be 
\[
\HHom_{\Cc}(C_1,C_2)\otimes\HHom_{\Cc}(C_2,C)\To\HHom_{\Cc}(C_1,C).
\]

This construction can be made into a dg-functor by 
letting 
\[
\Upsilon_{C,D}\colon\HHom_{\Cc}(C,D)\To
\HHom_{\dg(\Cc)}(\Upsilon(C),\Upsilon(D))
\]
be induced by the natural cochain maps (the adjunct of composition rule)
\[
\HHom_{\Cc}(C,D)\To[\HHom_{\Cc}(X,C),\HHom_{\Cc}(X,D)].
\]


We have the following Yoneda lemmas.
}



\begin{theorem}[Yoneda Lemma]\label{thm:Yoneda}
Let $\Cc$ be a dg-category.
For any object $C$ of $\Cc$ and any dg module $M$ over $\Cc$, 
we have an isomorphism
\[
\HHom_{\dg(\Cc)}(\Upsilon(C),M) \cong M(C)
\]
both natural in $C$ and $M$.
\end{theorem}
\begin{proof}
For any object $D$ of $\Cc$, we have a canonical cochain map
\[
M_{D,C}\colon\HHom_{\Cc}(D,C)\otimes M(C)\To M(D)
\]
hence a canonical cochain map
\[
M(C)\To[\HHom_{\Cc}(D,C), M(D)].
\]
It is easy to verify that these cochain maps 
give rise to an extraordinary naturality of 
$[\HHom_{\Cc}(-,C), M(-)]$. 
Hence, it remains to show that 
this extraordinary naturality 
satisfies the universal property of 
$\HHom_{\dg(\Cc)}(\Upsilon(C),M)$.

Let $\alpha_{(-)}\colon X^{\bullet}\to[\HHom_{\Cc}(-,C), M(-)]$ 
be any extraordinary naturality. 
Then taking the composition of $\widehat{\alpha_C}$, 
the adjunct of $\alpha_C$, with the identity of $C$, 
we get a cochain map
\[
X^{\bullet}\cong 
X^{\bullet}\otimes\ZZ\To
X^{\bullet}\otimes\HHom_{\Cc}(C,C)
\markar{\widehat{\alpha_C}}
M(C).
\]
Then it is not difficult to show this is 
the desired unique cochain map in the 
universal property of $\HHom_{\dg(\Cc)}(\Upsilon(C),M)$. 
\end{proof}
\begin{corollary}
The dg-functor 
\[
\Upsilon\colon\Cc\To\dg(\Cc)
\]
is fully faithful.
\end{corollary}
\begin{Rem}
This dg-functor is called the \bfblue{dg-Yoneda embedding}.
\end{Rem}

\begin{corollary}
For any objects $C$ and $D$ of $\Cc$ the followings are equivalent
\begin{proplist}
\item 
$\Upsilon(C)$ and $\Upsilon(D)$ are isomorphic;
\item 
$\Upsilon(C)_0$ and $\Upsilon(D)_0$ are isomorphic;
\item 
$C$ and $D$ are isomorphic.
\end{proplist}
Moreover, if this is the case, 
any natural isomorphism between 
$\Upsilon(C)$ and $\Upsilon(D)$ 
is induced by an isomorphism 
between $C$ and $D$.
\end{corollary}
\begin{proof}
Combine the dg-version of Yoneda lemma with 
the ordinary Yoneda lemma, this is clear.
\end{proof}

\para{
A dg-transformation $\alpha$ between dg-functors to $\Ch(\Aa)$ 
is called a \bfblue{natural quasi-isomorphism} 
if its each component $\alpha_C$ is a quasi-isomorphism. 
Two dg-functors are said to be \bfblue{quasi-isomorphic} if 
there is a zigzag of quasi-isomorphisms between them.

Apply this notion to the dg-transformation given by a dg-functor, 
we have the following notions:
\begin{itemize}
\item 
a dg-functor $F\colon\Cc\to\Dd$ is
\bfblue{quasi-fully faithful}, 
if for any two objects $C,D$ of $\Cc$, the cochain map
\[
F_{C,D}\colon\HHom_{\Cc}(C,D)\To\HHom_{\Dd}(F(C),F(D))
\]
is a quasi-isomorphism;
\item 
a quasi-fully faithful and homotopically essentially surjective 
dg-functor is called a \bfblue{quasi-equivalence}.
\end{itemize}

Note that since natural quasi-isomorphism are not 
invertible, even up to homotopy, in the dg-category 
of dg-functors, the notion of quasi-equivalences 
doesn't have a similar characterization as in 
\Cref{prop:equiv_of_dg_cats}. 
This drawback suggests that $\FFun(\Cc,\Dd)$ 
is not a good model for the full higher category of 
functors.
}

\begin{proposition}\label{prop:homotopyYoneda}
For any objects $C$ and $D$ of $\Cc$ the followings are equivalent
\begin{proplist}
\item 
$\Upsilon(C)$ and $\Upsilon(D)$ are equivalent;
\item 
$\Upsilon(C)$ and $\Upsilon(D)$ are quasi-isomorphic;
\item 
$\h\Upsilon(C)$ and $\h\Upsilon(D)$ are isomorphic;
\item 
$C$ and $D$ are equivalent.
\end{proplist}
Moreover, if this is the case, 
any natural equivalence between 
$\Upsilon(C)$ and $\Upsilon(D)$ 
is induced by an equivalence 
between $C$ and $D$.
\end{proposition}
\begin{proof}
(i) $\then$ (ii) $\then$ (iii) is clear. 

(iii) $\Leftrightarrow$ (iv) follows by 
apply the ordinary Yoneda to the category $\h\Cc$.

(i) $\Leftrightarrow$ (iv) follows by 
apply the dg-Yoneda to the category $\dg(\Cc)$, 
and then apply the lax functor $H^0$.
\end{proof}

\para{
A dg-module is said to be 
\bfblue{representable} 
(resp. \bfblue{weakly representable}, 
\bfblue{quasi-representable}) if it is 
isomorphic (resp. equivalent, quasi-isomorphic)
to some $\Upsilon(C)$.
}















\clearpage
\section{Homotopy limits}
\para{
Let $\Cc$ be a dg-category. 
A \bfblue{diagram} is a functor $D$
from a small category $\Ii$ to the underlying category of $\Cc$. 
We can simply denote it by $D\colon\Ii\to\Cc$. 
One can then talk about the notions of limits/colimits in $\Cc$. 

Of course $D$ is merely a functor not a dg-functor. 
So how can one get a dg-functor from it? 
This comes from the fact that taking underlying category admits 
a left adjoint: taking the \bfblue{free dg-category} of an 
ordinary/pre-additive category. 
This operation can be easily built as long as one knows the 
following adjunctions:
\begin{align*}
(\iota\dashv Z_0)\colon&\Ab\rightleftarrows\Ch,\\
(\text{Free abelian group}\dashv\text{Forgetful})\colon&
\Set\rightleftarrows\Ab.
\end{align*}

It may be ambiguous as we write the dg-functor 
induced by the functor $D\colon\Ii\to\Cc_0$ also 
by $D\colon\Ii\to\Cc$ since the free dg-category of 
$\Ii$ usually has different underlying category than $\Ii$. 
However, this notation is meaningful since 
the dg-functor $D$ and the functor $D$ are just 
a pair of adjuncts. 
As a comparison, the functor 
\[
\Fun(\Cc,\Dd)\To\Fun(\Cc_0,\Dd_0)
\]
sending a dg-functor to its underlying functor 
is neither fully faithful 
nor essentially surjective.
}

\para{
A \bfblue{dg-cone} 
from an object $X$ of $\Cc$ to $D$ 
is a dg-transformation from 
the constant functor with value $X$, 
hence also denoted by $X$, to $D$.
Then the \bfblue{complex of dg-cones} 
from $X$ to $D$ is 
merely the complex 
\[
\HHom_{\FFun(\Ii,\Cc)}(X,D).
\] 
An object $L$ of $\Cc$ is said to be 
a \bfblue{dg-limit} of $D$, if there is 
a natural quasi-isomorphism
\[
\HHom_{\Cc}(-,L) \markar{qis} 
\HHom_{\FFun(\Ii,\Cc)}(-,D).
\]
Note that, if this is the case, 
then by Yoneda lemma (\ref{thm:Yoneda}), 
this quasi-isomorphism is given by 
a dg-cone $\phi$ from $L$ to $D$. 
We say that $\phi$ \itblue{exhibits $L$ 
as a dg-limit of $D$}.
If the natural quasi-isomorphism $\phi_{\ast}$ 
is a natural isomorphism, then we say $L$ is a 
\bfblue{strong dg-limit}, or simply a \bfblue{limit}, of $D$. 
Note that, by Yoneda lemma (\ref{prop:homotopyYoneda}), 
the dg-limit of $D$ is unique up to equivalences and 
the limit of $D$ is unique up to unique isomorphism. 
In this since, we simply denote the dg-limit/limit of $D$ 
by $\lim D$. 

Dually, a \bfblue{dg-cocone} 
from $D$ to an object $X$ of $\Cc$
is a dg-transformation from $D$ to 
the constant functor with value $X$.
Then the \bfblue{complex of dg-cocones} 
from $D$ to $X$ is 
merely the complex 
\[
\HHom_{\FFun(\Ii,\Cc)}(D,X).
\] 
An object $C$ of $\Cc$ is said to be 
a \bfblue{dg-colimit} of $D$, if there is 
a natural quasi-isomorphism
\[
\HHom_{\Cc}(C,-) \markar{qis} 
\HHom_{\FFun(\Ii,\Cc)}(D,-).
\]
Note that, if this is the case, 
then by Yoneda lemma (\ref{thm:Yoneda}), 
this quasi-isomorphism is given by 
a dg-cocone $\psi$ from $D$ to $C$. 
We say that $\psi$ \itblue{exhibits $C$ 
as a dg-colimit of $D$}.
If the natural quasi-isomorphism $\psi_{\ast}$ 
is a natural isomorphism, then we say $C$ is a 
\bfblue{strong dg-colimit}, or simply a \bfblue{colimit}, of $D$. 
Note that, by Yoneda lemma (\ref{prop:homotopyYoneda}), 
the dg-colimit of $D$ is unique up to equivalences and 
the colimit of $D$ is unique up to unique isomorphism. 
In this since, we simply denote the dg-colimit/colimit of $D$ 
by $\colim D$. 

Let $\phi$ be a dg-cone exhibits 
an object $L$ as a limit of $D$, then 
the following composition is an isomorphism.
\[
\Hom_{\Cc_0}(-,L)\markar{Z^0(\phi_\ast)}
\Hom_{\Fun(\Ii,\Cc)}(-,D)\cong
\Hom_{\Fun(\Ii,\Cc_0)}(-,D)
\]
(notice that the isomorphism comes from the adjunction 
of building free dg-category and 
taking underlying category). 
Hence, a limit in $\Cc$ is just an ordinary limit in $\Cc_0$. 
Similar holds for colimits.
}

\para{
The notion of limits/colimits, or even dg-limits/dg-colimits 
doesn't involve much higher structures: 
limits/colimits are merely ordinary limits/colimits, 
while dg-limits/dg-colimits are objects equivalent to 
them. 

Contrarily, \itblue{homotopy limits/colimits} 
are rarely limits/colimits. 
First, a \itblue{homotopy cone/cocone} of $D$ 
is not a dg-transformation from/to a constant functor, 
or a commutative diagram in more concrete words. 
Rather, it is a diagram commuting only up to homotopy! 
Then, a \itblue{homotopy limits/colimit} of $D$
is the \emph{universal homotopy cone/cocone}.
More precisely, there should be a notion of 
\bfblue{homotopy cone} from an object $X$ to $D$ 
and a complex $\mathrm{HoCone}(X,D)$ 
of homotopy cones from $X$ to $D$. 
Then the \bfblue{homotopy limit} of $D$ 
is characterized by the quasi-isomorphism
\[
\HHom_{\Cc}(-,\holim D)
\markar{qis} 
\mathrm{HoCone}(-,D).
\]
Similarly, there should be a notion of 
\bfblue{homotopy cocone} from $D$ to an object $X$ 
and a complex $\mathrm{HoCocone}(D,X)$
of homotopy cocones from $D$ to $X$. 
Then the \bfblue{homotopy colimit} of $D$ 
is characterized by the quasi-isomorphism
\[
\HHom_{\Cc}(\hocolim D,-)
\markar{qis}  
\mathrm{HoCocone}(D,-).
\]

Be aware that a homotopy cone/cocone in $\Cc$ is not 
simply a cone/cocone in the homotopy category $\h\Cc$ 
since the later forgets higher homotopies. 
In practice is even worse, 
there are many reasonable homotopy categories 
just don't have enough limits/colimits!
}

\para{
From now on, I'll spell out 
some homotopy limits/colimits in details. 
The general paten is like the following:
\begin{proplist}
\item 
First, homotopies in $\Ch$ can be presented by cochain maps.
\item 
Hence, by spelling out the data of a homotopy cone 
over a diagram $D$ of certain shape $\Ii$ in the 
dg-category $\Ch$, we can translate it into 
the data of a cone over certain diagram $\s D$ 
in the ordinary category $\Ch$ 
in a canonical way.
\item 
Then, we can spell out the 
\itblue{strong homotopy limit} 
$\holim D$ of $D$ by realize it as 
the limit of $\s D$.
\item 
By the functoriality of $\lim \s D$, 
the above construction 
gives rise to a dg-functor 
\[
\holim\colon\FFun(\Ii,\Ch)\to\Ch.
\]
\item 
Then the homotopy limits/colimts 
can be be defined via the natural quasi-isomorphism 
\begin{align*}
\HHom_{\Cc}(-,\holim D) &\markar{qis} 
\holim\HHom_{\Cc}(-,D), \\
\HHom_{\Cc}(\hocolim D,-) &\markar{qis}  
\holim\HHom_{\Cc}(D,-).
\end{align*}
\end{proplist}
}

\para{
Consider the empty diagram. 
Then a homotopy cone/cocone is merely an object without extra data. 
Hence the complex of homotopy cones/cocones is 
the complex of dg-cones/dg-cocones. 
Hence, a homotopy limit of empty diagram is 
an object equivalent to the terminal object 
and is called a \bfblue{homotopy terminal object}.
Dually, a homotopy colimit of empty diagram is 
an object equivalent to the initial object 
and is called a \bfblue{homotopy initial object}.
}

\para{
Consider the diagram consisting of merely an object $D$.
Then a homotopy cone is a $1$-morphism to $D$, 
hence a dg-cone, and 
a homotopy cocone is a $1$-morphism from $D$, 
hence a dg-cocone. 
Therefore, a homotopy limit/colimit of an object $D$ is 
an object equivalent to $D$. 
}

\para{
Consider a family of objects $\{D_i\}_{i\in I}$. 
Then a homotopy cone is 
a family of morphisms to each $D_i$ without extra data, 
hence a dg-cone, and 
a homotopy cocone is 
a family of morphisms from each $D_i$ without extra data, 
hence a dg-cocone. 
Therefore, a homotopy limit of 
a family of objects $\{D_i\}_{i\in I}$ 
is an object equivalent to the product $\prod_{i\in I}D_i$ 
and is called the \bfblue{homotopy product}.
Dually, a homotopy colimit of 
a family of objects $\{D_i\}_{i\in I}$ 
is an object equivalent to the coproduct $\coprod_{i\in I}D_i$ 
and is called the \bfblue{homotopy coproduct}.
}

\para{%\label{mapping_path}
Let $f\colon C^{\bullet}\to D^{\bullet}$ 
be a cochain map between complexes of abelian groups. 
Then the data of a homotopy cone 
of this diagram 
(a \itblue{homotopy triangle}) 
consists of
\begin{itemize}
\item 
a complex $X^{\bullet}$;
\item 
two cochain maps $x_1\colon X^{\bullet}\to C^{\bullet}$ 
and $x_0\colon X^{\bullet}\to D^{\bullet}$; and
\item 
a homotopy $\Phi\colon x_0\then f\circ x_1$.
\end{itemize}
The above data can be organized 
into the following commutative diagram 
of complexes 
\[
\begin{tikzcd}
X^{\bullet}\ar{d}[swap]{x_1}\ar{r}{\Phi} & 
{\<I,D\>^{\bullet}}\ar{d}{\ev_1} \\
C^{\bullet}\ar{r}{f} & D^{\bullet}
\end{tikzcd}
\]
where the cochain map $x_0$ is hidden in 
the diagram by $x_0=\ev_0\circ\Phi$. 
Therefore, a homotopy triangle is equivalent to 
a commutative square as above, 
hence equivalent to a cochain map 
\[
x\colon X^{\bullet}\To\Path(f)^{\bullet}
\] 
where $\Path(f)^{\bullet}$ is the fiber product of 
$C^{\bullet}$ and $\<I,D\>^{\bullet}$ over $D^{\bullet}$. 
More elementarily, $\Path(f)^{\bullet}$ is the complex
\begin{align*}
\Path(f)^n &= 
D^{n-1}e^{\ast}\oplus D^nv_0^{\ast}\oplus C^nv_1^{\ast}, \\
\di^n &= 
\begin{pmatrix}
-\di_D^{n-1} & -1 & f^n \\
 & \di_D^n & \\
 &  & \di_C^n
\end{pmatrix}.
\end{align*}
Under this description, the cochain map $x$ has components
\[
x^n=(\phi^n,x_0^n,x_1^n)^{\rm t}
\]
where $\phi$ is the cochain homotopy presenting $\Phi$.
One can think this as a kind of \itblue{universal property}: 
whenever one has a homotopy triangle $(X,x_1,x_0,\phi)$, 
one gets a unique cochain map 
$x\colon X^{\bullet}\to\Path(f)^{\bullet}$ 
such that the compositions of $x$ with the three projections 
from $\Path(f)^{\bullet}$ give $\phi$, $x_0$ and $x_1$ 
respectively. 
Then the complex of homotopy triangles with vertex $X$ 
is essentially the complex 
\[
[X,\Path(f)]^{\bullet}.
\]
Let $[X,f]$ denote the cochain map 
\[
[X,C]^{\bullet}\To[X,D]^{\bullet}
\]
induced by $f$. 
Then it is easy to show that 
\[
[X,\Path(f)]^{\bullet} \cong \Path([X,f])^{\bullet}.
\]

Now, let $f\colon C\to D$ be a $1$-morphism in any dg-category $\Cc$. 
For any object $X$, let $\HHom_{\Cc}(X,f)$ denote 
the cochain map 
\[
\HHom_{\Cc}(X,C)\To\HHom_{\Cc}(X,D)
\]
induced by $f$. 
Then a \bfblue{mapping path object} of $f$ 
is an object $\Path(f)$ of $\Cc$ together with a natural quasi-isomorphism 
\[
\HHom_{\Cc}\big(-,\Path(f)\big)\markar{qis} 
\Path\big(\HHom_{\Cc}(-,f)\big).
\]
By Yoneda lemma (\Cref{thm:Yoneda}), 
this is equivalent to a $0$-cocycle $\pi$ of 
$\Path\big(\HHom_{\Cc}(\Path(f),f)\big)$ 
(called the \bfblue{universal homotopy triangle} 
exhibiting $\Path(f)$ as a mapping path object of $f$),
such that $\pi$ induces above natural quasi-isomorphism. 
If the above natural quasi-isomorphism is further an 
natural isomorphism, then we say $\Path(f)$ is 
a \bfblue{strong mapping path object} of $f$.

Dually, for any object $X$, let $\HHom_{\Cc}(f,X)$ denote 
the cochain map 
\[
\HHom_{\Cc}(D,X)\To\HHom_{\Cc}(C,X)
\]
induced by $f$. 
Then a \bfblue{mapping cylinder} of $f$ 
is an object $\Cyl(f)$ of $\Cc$ together with 
a natural quasi-isomorphism 
\[
\HHom_{\Cc}\big(\Cyl(f),-\big)\markar{qis} 
\Path\big(\HHom_{\Cc}(f,-)\big).
\]
This is equivalent to a $0$-cocycle $\pi$ of 
$\Cyl\big(\HHom_{\Cc}(f,\Cyl(f))\big)$ 
(called the \bfblue{universal homotopy cotriangle} 
exhibiting $\Cyl(f)$ as a mapping cylinder of $f$),
such that $\pi$ induces above natural quasi-isomorphism. 
If the above natural quasi-isomorphism is further an 
natural isomorphism, then we say $\Cyl(f)$ is 
a \bfblue{strong mapping cylinder} of $f$.

Let $f\colon C^{\bullet}\to D^{\bullet}$ 
be a cochain morphism of complexes in arbitrary 
abelian category $\Aa$. 
Then, the \itblue{strong mapping path object} of $f$ 
is the complex 
\begin{align*}
\Path(f)^n &= 
D^{n-1}e^{\ast}\oplus D^nv_0^{\ast}\oplus C^nv_1^{\ast}, \\
\di^n &= 
\begin{pmatrix}
-\di_D^{n-1} & -1 & f^n \\
 & \di_D^n & \\
 &  & \di_C^n
\end{pmatrix}.
\end{align*}
The \itblue{strong mapping cylinder} of $f$ 
is the complex 
\begin{align*}
\Cyl(f)^n &= 
C^{n+1}e\oplus C^nv_0\oplus D^nv_1, \\
\di^n &= 
\begin{pmatrix}
-\di_C^{n+1} & & \\
-1 & \di_C^n & \\
f^n &  & \di_D^n
\end{pmatrix}.
\end{align*}

Dually, if $f\colon C_{\bullet}\to D_{\bullet}$ 
is a chain morphism of complexes in $\Aa$, then 
the \itblue{strong mapping path object} of $f$ 
is the complex 
\begin{align*}
\Path(f)_n &= 
D_{n+1}e^{\ast}\oplus D_nv_0^{\ast}\oplus C_nv_1^{\ast}, \\
\partial_n &= 
\begin{pmatrix}
-\partial^D_{n+1} & -1 & f_n \\
 & \partial^D_n & \\
 &  & \partial^C_n
\end{pmatrix},
\end{align*}
and the \itblue{strong mapping cylinder} of $f$ 
is the complex 
\begin{align*}
\Cyl(f)_n &= 
C_{n-1}e\oplus C_nv_0\oplus D_nv_1, \\
\partial_n &= 
\begin{pmatrix}
-\partial^C_{n-1} & & \\
-1 & \partial^C_n & \\
f_n &  & \partial^D_n
\end{pmatrix}.
\end{align*}

}

\para{
One can also consider the \emph{reversed} version of previous: 
the data of a \itblue{reversed homotopy triangle} consists of
\begin{itemize}
\item 
a complex $X^{\bullet}$;
\item 
two cochain maps $x_0\colon X^{\bullet}\to C^{\bullet}$ 
and $x_1\colon X^{\bullet}\to D^{\bullet}$; and
\item 
a homotopy $\Phi\colon f\circ x_0\then x_1$.
\end{itemize}
The above data can be organized 
into the following commutative diagram 
of complexes 
\[
\begin{tikzcd}
C^{\bullet}\ar{r}{f} & D^{\bullet} \\
X^{\bullet}\ar{u}{x_0}\ar{r}{\Phi} & 
{\<I,D\>^{\bullet}}\ar{u}[']{\ev_0} 
\end{tikzcd}
\]
where the cochain map $x_1$ is hidden in 
the diagram by $x_1=\ev_1\circ\Phi$. 
Therefore, a homotopy triangle is equivalent to 
a commutative square as above, 
hence equivalent to a cochain map 
\[
x\colon X^{\bullet}\To\widetilde\Path(f)^{\bullet}
\] 
where $\widetilde\Path(f)^{\bullet}$ is the complex
\begin{align*}
\widetilde\Path(f)^n &= 
D^{n-1}e^{\ast}\oplus C^nv_0^{\ast}\oplus D^nv_1^{\ast}, \\
\di^n &= 
\begin{pmatrix}
-\di_D^{n-1} & -f^n & 1 \\
 & \di_C^n & \\
 &  & \di_D^n
\end{pmatrix}.
\end{align*}
Under this description, the cochain map $x$ has components
\[
x^n=(\phi^n,x_0^n,x_1^n)^{\rm t}
\]
where $\phi$ is the cochain homotopy presenting $\Phi$.
One can think this as a kind of \itblue{universal property}: 
whenever one has a homotopy triangle $(X,x_0,x_1,\phi)$, 
one gets a unique cochain map 
$x\colon X^{\bullet}\to\widetilde\Path(f)^{\bullet}$ 
such that the compositions of $x$ with the three projections 
from $\Path(f)^{\bullet}$ give $\phi$, $x_0$ and $x_1$ 
respectively. 
Then the complex of homotopy triangles with vertex $X$ 
is essentially the complex 
\[
[X,\widetilde\Path(f)]^{\bullet}.
\]
Then it is easy to show that 
\[
[X,\widetilde\Path(f)]^{\bullet} \cong \widetilde\Path([X,f])^{\bullet}.
\]

Now, let $f\colon C\to D$ be a $1$-morphism in any dg-category $\Cc$. 
Then a \bfblue{reversed mapping path object} of $f$ 
is an object $\widetilde\Path(f)$ of $\Cc$ 
together with a natural quasi-isomorphism 
\[
\HHom_{\Cc}\big(-,\widetilde\Path(f)\big)\markar{qis} 
\widetilde\Path\big(\HHom_{\Cc}(-,f)\big).
\]
This is equivalent to a $0$-cocycle $\pi$ of 
$\Path\big(\HHom_{\Cc}(\widetilde\Path(f),f)\big)$ 
(called the \bfblue{universal reversed homotopy triangle} 
exhibiting $\widetilde\Path(f)$ as a mapping path object of $f$),
such that $\pi$ induces above natural quasi-isomorphism. 
If the above natural quasi-isomorphism is further an 
natural isomorphism, then we say $\widetilde\Path(f)$ is 
a \bfblue{strong reversed mapping path object} of $f$.

Dually, a \bfblue{reversed mapping cylinder} of $f$ 
is an object $\widetilde\Cyl(f)$ of $\Cc$ together with 
a natural quasi-isomorphism 
\[
\HHom_{\Cc}\big(\widetilde\Cyl(f),-\big)\markar{qis} 
\widetilde\Path\big(\HHom_{\Cc}(f,-)\big).
\]
This is equivalent to a $0$-cocycle $\pi$ of 
$\Cyl\big(\HHom_{\Cc}(f,\widetilde\Cyl(f))\big)$ 
(called the \bfblue{universal reversed homotopy cotriangle} 
exhibiting $\widetilde\Cyl(f)$ as a mapping cylinder of $f$),
such that $\pi$ induces above natural quasi-isomorphism. 
If the above natural quasi-isomorphism is further an 
natural isomorphism, then we say $\widetilde\Cyl(f)$ is 
a \bfblue{strong reversed mapping cylinder} of $f$.

Let $f\colon C^{\bullet}\to D^{\bullet}$ 
be a cochain morphism of complexes in arbitrary 
abelian category $\Aa$. 
Then, the \itblue{strong reversed mapping path object} of $f$ 
is the complex 
\begin{align*}
\widetilde\Path(f)^n &= 
D^{n-1}e^{\ast}\oplus C^nv_0^{\ast}\oplus D^nv_1^{\ast}, \\
\di^n &= 
\begin{pmatrix}
-\di_D^{n-1} & -f^n & 1 \\
 & \di_C^n & \\
 &  & \di_C^n
\end{pmatrix}.
\end{align*}
The \itblue{strong reversed mapping cylinder} of $f$ 
is the complex 
\begin{align*}
\widetilde\Cyl(f)^n &= 
C^{n+1}e\oplus D^nv_0\oplus C^nv_1, \\
\di^n &= 
\begin{pmatrix}
-\di_C^{n+1} & & \\
-f^n & \di_D^n & \\
1 &  & \di_C^n
\end{pmatrix}.
\end{align*}

Dually, if $f\colon C_{\bullet}\to D_{\bullet}$ 
is a chain morphism of complexes in $\Aa$, then 
the \itblue{strong reversed mapping path object} of $f$ 
is the complex 
\begin{align*}
\widetilde\Path(f)_n &= 
D_{n+1}e^{\ast}\oplus C_nv_0^{\ast}\oplus D_nv_1^{\ast}, \\
\partial_n &= 
\begin{pmatrix}
-\partial^D_{n+1} & -f_n & 1 \\
 & \partial^C_n & \\
 &  & \partial^D_n
\end{pmatrix},
\end{align*}
and the \itblue{strong reversed mapping cylinder} of $f$ 
is the complex 
\begin{align*}
\widetilde\Cyl(f)_n &= 
C_{n-1}e\oplus D_nv_0\oplus C_nv_1, \\
\partial_n &= 
\begin{pmatrix}
-\partial^C_{n-1} & & \\
-f_n & \partial^D_n & \\
1 &  & \partial^C_n
\end{pmatrix}.
\end{align*}

}

\para{\label{homotopy_pullback}
Let $f\colon C^{\bullet}\to D^{\bullet}$ 
and $g\colon E^{\bullet}\to D^{\bullet}$
be two cochain maps between complexes of abelian groups. 
Then the data of a homotopy cone (a \itblue{homotopy square}) 
consists of
\begin{itemize}
\item 
a complex $X^{\bullet}$;
\item 
two cochain maps $x_1\colon X^{\bullet}\to C^{\bullet}$, 
and $x_0\colon X^{\bullet}\to E^{\bullet}$; and
\item 
a homotopy $\Phi\colon g\circ x_0\then f\circ x_1$.
\end{itemize}
The above data can be organized 
into the following commutative diagram 
of complexes 
\[
\begin{tikzcd}
E^{\bullet}\ar{r}{g}& D^{\bullet} \\
X^{\bullet}\ar[r,"{\Phi}"description]
\ar{d}[swap]{x_1}\ar{u}{x_0} & 
{\<I,D\>^{\bullet}}\ar{u}[swap]{\ev_0}\ar{d}{\ev_1} \\
C^{\bullet}\ar{r}{f} & D^{\bullet}
\end{tikzcd}
\]
hence is equivalent to a cochain map
\[
x\colon X^{\bullet}\To (C\times^h_DE)^{\bullet}
\]
where $(C\times^h_DE)^{\bullet}$ is the limit of the diagram
\[
\begin{tikzcd}
C^{\bullet}\ar{r}{f} & D^{\bullet} & 
{\<I,D\>^{\bullet}}\ar{l}[swap]{\ev_1}\ar{r}{\ev_0} & 
D^{\bullet} & E^{\bullet}\ar{l}[swap]{g}
\end{tikzcd}
\]
More elementarily, $(C\times^h_DE)^{\bullet}$ is the complex
\begin{align*}
(C\times^h_DE)^n &= 
D^{n-1}e^{\ast}\oplus E^nv_0^{\ast}\oplus C^nv_1^{\ast}, \\
\di^n &= 
\begin{pmatrix}
-\di_D^{n-1} & -g^n & f^n \\
 & \di_E^n & \\
 &  & \di_C^n
\end{pmatrix}.
\end{align*}
Under this description, the cochain map $x$ has components
\[
x^n=(\phi^n,x_0^n,x_1^n)^{\rm t},
\]
where $\phi$ is the cochain homotopy presenting $\Phi$.
Again, one can think this as a \itblue{universal property}:
whenever one has a homotopy square $(X,x_1,x_0,\phi)$, 
one gets a unique cochain map 
$x\colon X^{\bullet}\to(C\times^h_DE)^{\bullet}$ 
such that the compositions of $x$ with the three projections 
from $(C\times^h_DE)^{\bullet}$ give $\phi$, $x_0$ and $x_1$ 
respectively. 
Then the complex of homotopy square with vertex $X$ is essentially 
the complex 
\[
\big[X,C\times^h_DE\big]^{\bullet}.
\]
Then it is easy to show that 
\[
\big[X,C\times^h_DE\big]^{\bullet} \cong 
\Big([X,C]\times^h_{[X,D]}[X,E]\Big)^{\bullet}.
\]
Let $g'$ and $f'$ 
be the projections from $(C\times^h_DE)^{\bullet}$ 
to $C^{\bullet}$ and $E^{\bullet}$ respectively. 
Then we have 
\[
[X,f'] = [X,f]',
\qquad\text{and}\qquad
[X,g'] = [X,g]'.
\]

Now, let $f\colon C\to D$ 
and $g\colon E\to D$ be 
two $1$-morphisms in 
any dg-category $\Cc$. 
Then a \bfblue{homotopy fiber product} of $C$ and $E$ over $D$  
is an object $C\times^h_DE$ of $\Cc$ together with a natural quasi-isomorphism 
\[
\HHom_{\Cc}\Big(-,C\times^h_DE\Big)\markar{qis} 
\HHom_{\Cc}(-,C)\times^h_{\HHom_{\Cc}(-,D)}\HHom_{\Cc}(-,E).
\]
This is equivalent to a $0$-cocycle $\pi$ of the complex
$\HHom_{\Cc}(-,C)\times^h_{\HHom_{\Cc}(-,D)}\HHom_{\Cc}(-,E)$ 
(called the \bfblue{homotopy Cartesian diagram} 
exhibiting $C\times^h_DE$ 
as a homotopy fiber product 
of $C$ and $D$ over $E$),
such that $\pi$ induces above 
natural quasi-isomorphism. 
If the above natural quasi-isomorphism 
is further an natural isomorphism, 
then we say $C\times^h_DE$ is 
a \bfblue{strong homotopy fiber product}. 
The $1$-morphism $g'\colon C\times^h_DE\to C$ 
given by the projection 
\[
\HHom_{\Cc}(C\times^h_DE,C)
\times^h_{\HHom_{\Cc}(C\times^h_DE,D)}
\HHom_{\Cc}(C\times^h_DE,E)
\To
\HHom_{\Cc}(C\times^h_DE,C)
\]
is called the \bfblue{homotopy pullback of} $g$ along $f$.

Dually, let $f\colon D\to C$ 
and $g\colon D\to E$ be 
two $1$-morphisms in $\Cc$. 
Then a \bfblue{homotopy fiber coproduct} of $C$ and $E$ rel $D$  
is an object $C\amalg^h_DE$ of $\Cc$ together with a natural quasi-isomorphism 
\[
\HHom_{\Cc}\Big(C\amalg^h_DE,-\Big)\markar{qis} 
\HHom_{\Cc}(C,-)\times^h_{\HHom_{\Cc}(D,-)}\HHom_{\Cc}(E,-).
\]
This is equivalent to a $0$-cocycle $\pi$ of the complex
$\HHom_{\Cc}(C,-)\times^h_{\HHom_{\Cc}(D,-)}\HHom_{\Cc}(E,-)$ 
(called the \bfblue{homotopy co-Cartesian diagram} 
exhibiting the object $C\amalg^h_DE$ 
as a homotopy fiber coproduct 
of $C$ and $D$ rel $E$),
such that $\pi$ induces above 
natural quasi-isomorphism. 
If the above natural quasi-isomorphism 
is further an natural isomorphism, 
then we say $C\amalg^h_DE$ is 
a \bfblue{strong homotopy fiber coproduct}. 
The $1$-morphism $g'\colon C\to C\amalg^h_DE$ 
given by the projection 
\[
\HHom_{\Cc}(C,C\amalg^h_DE)
\times^h_{\HHom_{\Cc}(D,C\amalg^h_DE)}
\HHom_{\Cc}(E,C\amalg^h_DE)
\To
\HHom_{\Cc}(C,C\amalg^h_DE)
\]
is called the \bfblue{homotopy pushout of} $g$ along $f$.

Let $f\colon C^{\bullet}\to D^{\bullet}$ 
and $g\colon E^{\bullet}\to D^{\bullet}$
be two cochain morphisms between complexes in arbitrary 
abelian category $\Aa$. 
Then, the \itblue{strong homotopy fiber product} $C\times^h_DE$ 
is the complex 
\begin{align*}
(C\times^h_DE)^n &= 
D^{n-1}e^{\ast}\oplus E^nv_0^{\ast}\oplus C^nv_1^{\ast}, \\
\di^n &= 
\begin{pmatrix}
-\di_D^{n-1} & -g^n & f^n \\
 & \di_E^n & \\
 &  & \di_C^n
\end{pmatrix}.
\end{align*}
Dually, let $f\colon D^{\bullet}\to C^{\bullet}$ 
and $g\colon D^{\bullet}\to E^{\bullet}$
be two cochain morphisms between complexes in $\Aa$. 
Then, the \itblue{strong homotopy fiber coproduct} $C\amalg^h_DE$ 
is the complex 
\begin{align*}
(C\amalg^h_DE)^n &= 
D^{n+1}e\oplus E^nv_0\oplus C^nv_1, \\
\di^n &= 
\begin{pmatrix}
-\di_D^{n+1} & & \\
-g^n & \di_E^n & \\
f^n &  & \di_C^n
\end{pmatrix}.
\end{align*}

Let $f\colon C_{\bullet}\to D_{\bullet}$ 
and $g\colon E_{\bullet}\to D_{\bullet}$
be two chain morphisms between complexes in $\Aa$. 
Then, the \itblue{strong homotopy fiber product} $C\times^h_DE$ 
is the complex 
\begin{align*}
(C\times^h_DE)_n &= 
D_{n+1}e^{\ast}\oplus E_nv_0^{\ast}\oplus C_nv_1^{\ast}, \\
\partial_n &= 
\begin{pmatrix}
-\partial^D_{n+1} & -g_n & f_n \\
 & \partial^E_n & \\
 &  & \partial^C_n
\end{pmatrix}.
\end{align*}
Dually, let $f\colon D^{\bullet}\to C^{\bullet}$ 
and $g\colon D^{\bullet}\to E^{\bullet}$
be two cochain maps between complexes in $\Aa$. 
Then, the \itblue{strong homotopy fiber coproduct} $C\amalg^h_DE$ 
is the complex 
\begin{align*}
(C\amalg^h_DE)_n &= 
D_{n-1}e\oplus E_nv_0\oplus C_nv_1, \\
\partial_n &= 
\begin{pmatrix}
-\partial^D_{n-1} & & \\
-g_n & \partial^E_n & \\
f_n &  & \partial^C_n
\end{pmatrix}.
\end{align*}

}

\begin{proposition}[Pasting lemma]\label{prop:Pasting}
Suppose we have the following diagram in a dg-category $\Cc$:
\[
\begin{tikzcd}
\bullet\ar{r}\ar{d} & 
\bullet\ar{r}\ar{d}\ar[dl,Rightarrow,dashed] & 
\bullet\ar{d}\ar[dl,Rightarrow,dashed] \\
\bullet\ar{r} & 
\bullet\ar{r} & 
\bullet
\end{tikzcd}
\]
where the dashed arrows denote homotopies. 
\begin{proplist}
\item 
If both square are homotopy Cartesian diagrams 
(resp. homotopy co-Cartesian diagrams), 
then so is the rectangle.
\item 
If both the right square and the rectangle 
are homotopy Cartesian diagrams, 
then so is the left square.
\item 
If both the left square and the rectangle 
are homotopy co-Cartesian diagrams, 
then so is the right square.
\end{proplist}
\end{proposition}
\begin{proof}
The key point is: the two squares and 
the rectangle are homotopy squares. 
Hence the statements follows from the 
universal property. 
If one is satisfied with this argument, 
one can just skip the following proof.

To spell out an explicit proof, 
first note that it is sufficient 
to prove the statements 
of homotopy Cartesian diagrams in $\Ch$. 

Suppose we have the diagram
\[
\begin{tikzcd}
{C\times^h_D(D\times^h_EF)}\ar{r}{f'}\ar{d}[']{h''} & 
{D\times^h_EF}\ar{r}{g'}\ar{d}[']{h'}
\ar[dl,Rightarrow,dashed,"\alpha"description] & 
F\ar{d}{h}
\ar[dl,Rightarrow,dashed,"\beta"description] \\
C\ar{r}[']{f} & 
D\ar{r}[']{g} & 
E
\end{tikzcd}
\]
where both squares are 
homotopy Cartesian diagrams. 
Then, by composing 
$g'$ with $f$, 
$g$ with $f$ and
$\alpha$ with $\beta$, 
we get a homotopy square
\[
\begin{tikzcd}
{C\times^h_D(D\times^h_EF)}\ar{r}{g'\circ f'}\ar{d}[']{h''} & 
F\ar{d}{h}
\ar[dl,Rightarrow,dashed,"{\Phi}"description] \\
C\ar{r}[']{g\circ f} & 
E
\end{tikzcd}
\]
where the homotopy $\Phi$ is 
\[
(g\ast\alpha)\dotplus(\beta\ast f').
\]
Therefore, the corresponding cochain map 
\[
p\colon{C\times^h_D(D\times^h_EF)}\To{C\times^h_EF}
\]
is $(\phi,g'\circ f',h'')^{\rm t}$ which can be spelled out as
\[
p^n=
\begin{pmatrix}
g^n & 1 & & & \\
& 0 & 1 & 0 & \\
& & & & 1
\end{pmatrix}
\]
under the decomposition
\begin{align*}
(C\times^h_D(D\times^h_EF))^n &=
D^{n-1}e^{\ast}\oplus 
E^{n-1}e^{\ast}v_0^{\ast}\oplus F^nv_0^{\ast}v_0^{\ast}
\oplus D^nv_1^{\ast}v_0^{\ast}
\oplus C^nv_1^{\ast},\\
(C\times^h_EF)^n &=
E^{n-1}e^{\ast}\oplus F^nv_0^{\ast}\oplus C^nv_1^{\ast}.
\end{align*}

Conversely, 
suppose we have homotopy Cartesian diagrams. 
\[
\begin{tikzcd}
{C\times^h_EF}\ar{r}{(g\circ f)'}\ar{d}[']{h'''} & 
F\ar{d}{h}
\ar[dl,Rightarrow,dashed,"{\Psi}"description] \\
C\ar{r}[']{g\circ f} & 
E
\end{tikzcd}
\qquad\qquad
\begin{tikzcd}
{D\times^h_EF}\ar{r}{g'}\ar{d}[']{h'} & 
F\ar{d}{h}
\ar[dl,Rightarrow,dashed,"{\beta}"description] \\
D\ar{r}[']{g} & 
E
\end{tikzcd}
\]
Then, the left one gives a homotopy square
\[
\begin{tikzcd}
{C\times^h_EF}\ar{r}{(g\circ f)'}\ar{d}[']{f\circ h'''} & 
F\ar{d}{h}
\ar[dl,Rightarrow,dashed,"{\Psi}"description] \\
D\ar{r}[']{g} & 
E
\end{tikzcd}
\]
The corresponding cochain map
\[
f''\colon C\times^h_EF\To D\times^h_EF
\]
is $(\psi,(g\circ f)',f\circ h''')^{\rm t}$ 
which can be spelled out as
\[
f^{\prime\prime n}=
\begin{pmatrix}
1 & & \\
& 1 & \\
& & f^n
\end{pmatrix}
\]
under the decomposition
\begin{align*}
(C\times^h_EF)^n &=
E^{n-1}e^{\ast}\oplus F^nv_0^{\ast}\oplus C^nv_1^{\ast},\\
(D\times^h_EF)^n &=
E^{n-1}e^{\ast}\oplus F^nv_0^{\ast}\oplus D^nv_1^{\ast}.
\end{align*}
Now, we have a commutative square 
\[
\begin{tikzcd}
{C\times^h_EF}\ar{r}{f''}\ar{d}[']{h'''} & 
{D\times^h_EF}\ar{d}{h'} \\
C\ar{r}[']{f} & 
D
\end{tikzcd}
\]
Therefore, the corresponding cochain map 
\[
q\colon {C\times^h_EF}\To{C\times^h_D(D\times^h_EF)}
\]
is $(0,f'',h''')^{\rm t}$ which can be spelled out as 
\[
q^n=
\begin{pmatrix}
0 & & \\
1 & & \\
& 1 & \\
& & f^n \\
& & 1
\end{pmatrix}.
\] 
Now, it is clear that 
\[
p\circ q = \id_{C\times^h_EF}
\]
On the other hand, $\alpha$ gives a homotopy 
from $\id_{C\times^h_D(D\times^h_EF)}$ 
to $q\circ p$ by composing with a section of $h'\circ f'$
\end{proof}

\para{\label{homotopy_fiber}
Let $f\colon C^{\bullet}\to D^{\bullet}$ 
be a cochain map between complexes of abelian groups. 
Consider the diagram 
$C^{\bullet}\markar{f} D^{\bullet}\longleftarrow 0$. 
Note that this is a special case of \ref{homotopy_pullback}. 
However, since the special properties of $0$, 
one expects a more concentrate expression of 
the strong homotopy limit of this diagram.
A homotopy cone of this diagram (a \itblue{homotopy annihilation}) 
consists of the following data
\begin{itemize}
\item 
a complex $X^{\bullet}$;
\item 
a cochain map $x_1\colon X^{\bullet}\to C^{\bullet}$; and
\item 
a homotopy $\Phi\colon 0\then f\circ x_0$.
\end{itemize}
The above data can be organized 
into the following commutative diagram 
of complexes 
\[
\begin{tikzcd}
{}&D^{\bullet}\\
X^{\bullet}\ar{ur}{0}\ar[r,"{\Phi}"description]
\ar{d}[swap]{x_1} & 
{\<I,D\>^{\bullet}}\ar{u}[']{\ev_0}\ar{d}{\ev_1} \\
C^{\bullet}\ar{r}{f} & D^{\bullet}
\end{tikzcd}
\]
hence is equivalent to a cochain map
\[
x\colon X^{\bullet}\To \Fib(f)^{\bullet}
\]
where $\Fib(f)^{\bullet}$ is the limit of the diagram
\[
\begin{tikzcd}
C^{\bullet}\ar{r}{f} & D^{\bullet} & 
{\<I,D\>^{\bullet}}\ar{l}[swap]{\ev_1}\ar{r}{\ev_0} & 
D^{\bullet} & 0\ar{l}
\end{tikzcd}
\]
More elementarily, $\Fib(f)^{\bullet}$ is the complex
\begin{align*}
\Fib(f)^n &= 
D^{n-1}e^{\ast}\oplus C^nv_1^{\ast}, \\
\di^n &= 
\begin{pmatrix}
-\di_D^{n-1} & f^n \\
 & \di_C^n
\end{pmatrix}.
\end{align*}
Under this description, the cochain map $x$ has components
\[
x^n=(\phi^n,x_1^n)^{\rm t},
\]
where $\phi$ is the cochain homotopy presenting $\Phi$.
Again, one can think this as a \itblue{universal property}:
whenever one has a homotopy annihilation $(X,x_1,\phi)$, 
one gets a unique cochain map 
$x\colon X^{\bullet}\to\Fib(f)^{\bullet}$ 
such that the compositions of $x$ with the two projections 
from $\Fib(f)^{\bullet}$ give $\phi$ and $x_1$ 
respectively. 
Then the complex of homotopy annihilation 
with vertex $X$ is the complex 
\[
[X,\Fib(f)]^{\bullet}.
\]
Then it is easy to show that 
\[
[X,\Fib(f)]^{\bullet} \cong 
\Fib([X,f])^{\bullet}.
\]

Now, let $f\colon C\to D$ be 
a $1$-morphisms in 
any dg-category $\Cc$. 
Then a \bfblue{homotopy fiber} of $f$  
is an object $\Fib(f)$ of $\Cc$ together with a natural quasi-isomorphism 
\[
\HHom_{\Cc}\big(-,\Fib(f)\big)\markar{qis} 
\Fib\big(\HHom_{\Cc}(-,f)\big).
\]
This is equivalent to a $0$-cocycle $\pi$ of the complex
$\Fib\big(\HHom_{\Cc}(-,f)\big)$
such that $\pi$ induces above 
natural quasi-isomorphism. 
If the above natural quasi-isomorphism 
is further an natural isomorphism, 
then we say $\Fib(f)$ is 
a \bfblue{strong homotopy fiber}. 

Dually, a \bfblue{homotopy cofiber}, 
or \bfblue{mapping cone} of $f$ 
is an object $\Cofib(f)$ of $\Cc$ together with a natural quasi-isomorphism 
\[
\HHom_{\Cc}\big(\Cofib(f),-\big)\markar{qis} 
\Fib\big(\HHom_{\Cc}(f,-)\big).
\]
This is equivalent to a $0$-cocycle $\pi$ of the complex
$\Fib\big(\HHom_{\Cc}(f,-)\big)$ 
such that $\pi$ induces above 
natural quasi-isomorphism. 
If the above natural quasi-isomorphism 
is further an natural isomorphism, 
then we say $\Cofib(f)$ is 
a \bfblue{strong homotopy cofiber}.

Let $f\colon C^{\bullet}\to D^{\bullet}$ 
be a cochain morphism between complexes 
in arbitrary abelian category $\Aa$. 
Then, the \itblue{strong homotopy fiber} $\Fib(f)$ 
is the complex 
\begin{align*}
\Fib(f)^n &= 
D^{n-1}e^{\ast}\oplus C^nv_1^{\ast}, \\
\di^n &= 
\begin{pmatrix}
-\di_D^{n-1} & f^n \\
 & \di_C^n
\end{pmatrix},
\end{align*}
and the \itblue{strong homotopy cofiber} $\Cofib(f)$ 
is the complex 
\begin{align*}
\Cofib(f)^n &= 
C^{n+1}e\oplus D^nv_1, \\
\di^n &= 
\begin{pmatrix}
-\di_C^{n+1} & \\
f^n & \di_D^n
\end{pmatrix}.
\end{align*}

Dually, let $f\colon C_{\bullet}\to D_{\bullet}$ 
be a chain morphism between complexes in $\Aa$. 
Then, the \itblue{strong homotopy fiber} $\Fib(f)$ 
is the complex 
\begin{align*}
\Fib(f)_n &= 
D_{n+1}e^{\ast}\oplus C_nv_1^{\ast}, \\
\partial_n &= 
\begin{pmatrix}
-\partial^D_{n+1} & f_n \\
 & \partial^C_n
\end{pmatrix},
\end{align*}
and the \itblue{strong homotopy cofiber} $\Cofib(f)$ 
is the complex 
\begin{align*}
\Cofib(f)_n &= 
C_{n-1}e\oplus D_nv_1, \\
\partial_n &= 
\begin{pmatrix}
-\partial^C_{n-1} & \\
f_n & \partial^D_n
\end{pmatrix}.
\end{align*}

}

\para{
One can also consider the \emph{reversed} version of \ref{homotopy_fiber}: 
now the diagram is $0 \To D^{\bullet} \markal{f} C^{\bullet}$. 
So, the data of a \itblue{reversed homotopy annihilation} consists of 
the following data
\begin{itemize}
\item 
a complex $X^{\bullet}$;
\item 
a cochain map $x_0\colon X^{\bullet}\to C^{\bullet}$; and
\item 
a homotopy $\Phi\colon f\circ x_0\then 0$.
\end{itemize}
The above data can be organized 
into the following commutative diagram 
of complexes 
\[
\begin{tikzcd}
C^{\bullet}\ar{r}{f} & D^{\bullet}\\
X^{\bullet}\ar{dr}{0}\ar[r,"{\Phi}"description]
\ar{u}{x_0} & 
{\<I,D\>^{\bullet}}\ar{u}[']{\ev_0}\ar{d}{\ev_1} \\
& D^{\bullet}
\end{tikzcd}
\]
hence is equivalent to a cochain map
\[
x\colon X^{\bullet}\To \widetilde\Fib(f)^{\bullet}
\]
where $\widetilde\Fib(f)^{\bullet}$ is the limit of the diagram
\[
\begin{tikzcd}
C^{\bullet}\ar{r}{f} & D^{\bullet} & 
{\<I,D\>^{\bullet}}\ar{l}[swap]{\ev_0}\ar{r}{\ev_1} & 
D^{\bullet} & 0\ar{l}
\end{tikzcd}
\]
More elementarily, $\widetilde\Fib(f)^{\bullet}$ is the complex
\begin{align*}
\widetilde\Fib(f)^n &= 
D^{n-1}e^{\ast}\oplus C^nv_0^{\ast}, \\
\di^n &= 
\begin{pmatrix}
-\di_D^{n-1} & -f^n \\
 & \di_C^n
\end{pmatrix}.
\end{align*}
Under this description, the cochain map $x$ has components
\[
x^n=(\phi^n,x_0^n)^{\rm t},
\]
where $\phi$ is the cochain homotopy presenting $\Phi$.
Again, one can think this as a \itblue{universal property}:
whenever one has a homotopy annihilation $(X,x_0,\phi)$, 
one gets a unique cochain map 
$x\colon X^{\bullet}\to\widetilde\Fib(f)^{\bullet}$ 
such that the compositions of $x$ with the two projections 
from $\widetilde\Fib(f)^{\bullet}$ give $\phi$ and $x_0$ 
respectively. 
Then the complex of homotopy annihilation 
with vertex $X$ is the complex 
\[
[X,\widetilde\Fib(f)]^{\bullet}.
\]
Then it is easy to show that 
\[
[X,\widetilde\Fib(f)]^{\bullet} \cong 
\widetilde\Fib([X,f])^{\bullet}.
\]

Now, let $f\colon C\to D$ be 
a $1$-morphisms in 
any dg-category $\Cc$. 
Then a \bfblue{reversed homotopy fiber} of $f$  
is an object $\widetilde\Fib(f)$ of $\Cc$ 
together with a natural quasi-isomorphism 
\[
\HHom_{\Cc}\big(-,\widetilde\Fib(f)\big)\markar{qis} 
\widetilde\Fib\big(\HHom_{\Cc}(-,f)\big).
\]
This is equivalent to a $0$-cocycle $\pi$ of the complex
$\widetilde\Fib\big(\HHom_{\Cc}(-,f)\big)$
such that $\pi$ induces above 
natural quasi-isomorphism. 
If the above natural quasi-isomorphism 
is further an natural isomorphism, 
then we say $\widetilde\Fib(f)$ is 
a \bfblue{strong reversed homotopy fiber}. 

Dually, a \bfblue{reversed homotopy cofiber}, 
or \bfblue{reversed mapping cone} of $f$ 
is an object $\widetilde\Cofib(f)$ of $\Cc$ \
together with a natural quasi-isomorphism 
\[
\HHom_{\Cc}\big(\widetilde\Cofib(f),-\big)\markar{qis} 
\widetilde\Fib\big(\HHom_{\Cc}(f,-)\big).
\]
This is equivalent to a $0$-cocycle $\pi$ of the complex
$\widetilde\Fib\big(\HHom_{\Cc}(f,-)\big)$ 
such that $\pi$ induces above 
natural quasi-isomorphism. 
If the above natural quasi-isomorphism 
is further an natural isomorphism, 
then we say $\widetilde\Cofib(f)$ is 
a \bfblue{strong reversed homotopy cofiber}.

Let $f\colon C^{\bullet}\to D^{\bullet}$ 
be a cochain morphism between complexes 
in arbitrary abelian category $\Aa$. 
Then, the \itblue{strong reversed homotopy fiber} $\Fib(f)$ 
is the complex 
\begin{align*}
\widetilde\Fib(f)^n &= 
D^{n-1}e^{\ast}\oplus C^nv_0^{\ast}, \\
\di^n &= 
\begin{pmatrix}
-\di_D^{n-1} & f^n \\
 & \di_C^n
\end{pmatrix},
\end{align*}
and the \itblue{strong reversed homotopy cofiber} $\Cofib(f)$ 
is the complex 
\begin{align*}
\widetilde\Cofib(f)^n &= 
C^{n+1}e\oplus D^nv_0, \\
\di^n &= 
\begin{pmatrix}
-\di_C^{n+1} & \\
f^n & \di_D^n
\end{pmatrix}.
\end{align*}

Dually, let $f\colon C_{\bullet}\to D_{\bullet}$ 
be a chain morphism between complexes in $\Aa$. 
Then, the \itblue{strong reversed homotopy fiber} $\Fib(f)$ 
is the complex 
\begin{align*}
\widetilde\Fib(f)_n &= 
D_{n+1}e^{\ast}\oplus C_nv_0^{\ast}, \\
\partial_n &= 
\begin{pmatrix}
-\partial^D_{n+1} & f_n \\
 & \partial^C_n
\end{pmatrix},
\end{align*}
and the \itblue{strong reversed homotopy cofiber} $\Cofib(f)$ 
is the complex 
\begin{align*}
\widetilde\Cofib(f)_n &= 
C_{n-1}e\oplus D_nv_0, \\
\partial_n &= 
\begin{pmatrix}
-\partial^C_{n-1} & \\
f_n & \partial^D_n
\end{pmatrix}.
\end{align*}
}

\para{
Let $C^{\bullet}$ 
be a cochain complex of abelian groups. 
Consider the diagram 
$0\to C^{\bullet}\from 0$. 
Note that this is a special case of \ref{homotopy_fiber}. 
However, since the special properties of $0$, 
one expects a more concentrated expression of 
the strong homotopy limit of this diagram.
A homotopy cone of this diagram (a \itblue{homotopy loop}) 
consists of the following data
\begin{itemize}
\item 
a complex $X^{\bullet}$; and
\item 
a homotopy $\Phi\colon 0\then 0\colon X^{\bullet}\to C^{\bullet}$.
\end{itemize}
The above data can be organized 
into the following commutative diagram 
of complexes 
\[
\begin{tikzcd}
& C^{\bullet} \\
X^{\bullet}\ar{ur}{0}\ar[r,"{\Phi}"description]
\ar{dr}[']{0} & 
{\<I,C\>^{\bullet}}\ar{u}[']{\ev_0}\ar{d}{\ev_1} \\
& C^{\bullet}
\end{tikzcd}
\]
hence is equivalent to a cochain map
\[
x\colon X^{\bullet}\To \Omega C^{\bullet}
\]
where $\Omega C^{\bullet}$ is the limit of the diagram
\[
\begin{tikzcd}
{\<I,C\>^{\bullet}}\ar{d}[swap]{\ev_1}\ar{r}{\ev_0} &
C^{\bullet} \\
C^{\bullet} & 0\ar{l}\ar{u}
\end{tikzcd}
\]
More elementarily, $\Omega C^{\bullet}$ is the complex
\[
\Omega C^n = 
C^{n-1}e^{\ast}, \qquad
\di^n = -\di_C^{n-1}.
\]
In other words, $\Omega C^{\bullet} = C[-1]^{\bullet}$. 
Under this description, the cochain map $x$ is precisely 
the cochain homotopy presenting $\Phi$.
Then the complex of homotopy loops 
with vertex $X$ is the complex 
\[
[X,\Omega C]^{\bullet}.
\]
Then it is easy to show that 
\[
[X,\Omega C]^{\bullet} \cong 
\Omega[X,C]^{\bullet}.
\]

Now, let $C$ be 
an object in a dg-category $\Cc$. 
Then a \bfblue{loop space object}, 
or \bfblue{looping}, of $C$  
is an object $\Omega C$ of $\Cc$ together with a natural quasi-isomorphism 
\[
\HHom_{\Cc}(-,\Omega C)\markar{qis} 
\Omega\HHom_{\Cc}(-,C).
\]
This is equivalent to a $0$-cocycle $\pi$ of the complex
$\Omega\HHom_{\Cc}(-,C)$
such that $\pi$ induces above 
natural quasi-isomorphism. 
If the above natural quasi-isomorphism 
is further an natural isomorphism, 
then we say $\Omega C$ is 
a \bfblue{strong loop space object}. 

Dually, a \bfblue{suspension} of $C$ 
is an object $\Sigma C$ of $\Cc$ together with a natural quasi-isomorphism 
\[
\HHom_{\Cc}(\Sigma C,-)\markar{qis} 
\Omega\HHom_{\Cc}(C,-).
\]
This is equivalent to a $0$-cocycle $\pi$ of the complex
$\Omega\HHom_{\Cc}(C,-)$ 
such that $\pi$ induces above 
natural quasi-isomorphism. 
If the above natural quasi-isomorphism 
is further an natural isomorphism, 
then we say $\Sigma C$ is 
a \bfblue{strong suspension}.

Let $C^{\bullet}$ be a cochain complex 
in arbitrary abelian category $\Aa$. 
Then, the \itblue{strong loop space object} $\Omega C$ 
is the complex 
\[
\Omega C^n = C^{n-1}e^{\ast}, \qquad
\di^n = -\di_C^{n-1}, 
\]
i.e. $\Omega C^{\bullet}=C[-1]^{\bullet}$. 
The \itblue{strong suspension} $\Sigma C$ 
is the complex 
\[
\Sigma C^n = C^{n+1}e, \qquad
\di^n = -\di_C^{n+1}, 
\]
i.e. $\Sigma C^{\bullet}=C[1]^{\bullet}$. 

Dually, let $C_{\bullet}$ 
be a chain complex in $\Aa$. 
Then, the \itblue{strong loop space object} $\Omega C$ 
is the complex 
\[
\Omega C_n = C_{n+1}e^{\ast},\qquad
\partial_n = -\partial^C_{n+1},
\]
i.e. $\Omega C_{\bullet}=C[1]_{\bullet}$.
The \itblue{strong suspension} $\Sigma C$ 
is the complex 
\[
\Sigma C_n = C_{n-1}e, \qquad
\partial_n = -\partial^C_{n-1},
\]
i.e. $\Sigma C_{\bullet}=C[-1]_{\bullet}$.
}












\clearpage
\section{Fiber sequences and exact sequences}
\para{
Let $\Cc$ be a dg-category.
A sequence 
$C\markar{f} D\markar{g} E$ is called a 
\bfblue{fiber sequence} 
(resp. \bfblue{cofiber sequence}) if $E$ is a homotopy fiber of $g$ 
(resp. $E$ is a homotopy cofiber of $f$). 
A \bfblue{long fiber sequence} 
(resp. \bfblue{long cofiber sequence}) 
is a sequence in which any two adjoining morphisms 
form a fiber sequence (resp. cofiber sequence).

Suppose $C\markar{f} D\markar{g} E$ is a fiber sequence.
Consider the following diagram
\[
\begin{tikzcd}
\widetilde\Fib(f)\ar{r}{h}\ar{d} & 
C\ar{r}\ar{d}{f}
\ar[dl,Rightarrow,dashed] & 
0\ar{d}
\ar[dl,Rightarrow,dashed] \\
0\ar{r} & 
D\ar{r}[']{g} & 
E
\end{tikzcd}
\]
where both squares are homotopy Cartesian diagrams. 
By pasting lemma (\ref{prop:Pasting}), 
the rectangle is also a homotopy Cartesian diagram. 
Hence we have equivalence 
\[
\widetilde\Fib(f)\simeq \Omega E.
\]
Furthermore, consider the following diagram 
\[
\begin{tikzcd}
\Fib(h)\ar{d}[']{\pi}\ar{r}
&0\ar[dl,Rightarrow,dashed]\ar{d}\\
\widetilde\Fib(f)\ar{r}{h}\ar{d} & 
C\ar{d}{f}
\ar[dl,Rightarrow,dashed]  \\
0\ar{r} & 
D
\end{tikzcd}
\]
with both squares are homotopy Cartesian diagrams. 
By pasting lemma (\ref{prop:Pasting}), 
the rectangle is also a homotopy Cartesian diagram. 
Hence we have equivalence
\[
\Fib(h)\simeq \Omega D.
\]
By the functoriality of $\Omega$ and since we have reversed 
$\Fib(f)$, the morphism $\pi$ is precisely $-\Omega g$.
Keep the previous processes, we get a long fiber sequence
\[
\cdots\To \Omega C
\markar{-\Omega f}\Omega D
\markar{-\Omega g}\Omega E
\markar{h}C
\markar{f}D
\markar{g}E.
\]

Dually, suppose $C\markar{f} D\markar{g} E$ is a cofiber sequence. 
Then we get a long cofiber sequence
\[
C\markar{f}
D\markar{g}
E\markar{h}
\Sigma C\markar{-\Sigma f}
\Sigma D\markar{-\Sigma g}
\Sigma E\To\cdots.
\]

Combine the aboves, for any morphism $f\colon C\to D$, 
we have a sequence
\[
\Fib(f)\To C\markar{f} D\To\Cofib(f)
\]
extending to both directions.
}








\printbibliography
\clearpage
\appendix
\section*{Abandoned drafts}
\begin{proposition}
Let $f\colon C\to D$ be a $1$-morphism in a dg-category. 
Let $f_{\ast}$ be the induced dg-transformation
\[
f_{\ast}\colon\HHom_{\Cc}(-,C)\To\HHom_{\Cc}(-,D).
\]
Then the followings are equivalent.
\begin{proplist}
\item 
$f_{\ast}$ is a natural equivalence.
\item 
$f_{\ast}$ is a natural quasi-isomorphism.
\item 
$H^0(f_{\ast})$ is a natural isomorphism.
\item 
$f$ is a homotopy equivalence.
\end{proplist}
The similar statement also holds for $f^{\ast}$.
\end{proposition}
\begin{proof}
Let's first prove (iii) implies (iv). 
Indeed, if $H^0(f_{\ast})$ is a natural isomorphism, then 
in particular we have isomorphisms
\begin{align*}
\Hom_{\h\Cc}(D,C) &\markar{f_{\ast}} \Hom_{\h\Cc}(D,D),\\
\Hom_{\h\Cc}(C,C) &\markar{f_{\ast}} \Hom_{\h\Cc}(C,D).
\end{align*}
The first isomorphism gives a $1$-morphism $g\colon D\to C$ 
such that $f\circ g\simeq\id_D$. 
The second isomorphism deduces $g\circ f\simeq\id_C$ 
from $f\circ g\circ f\simeq f$.

It remains to prove (iv) implies (i). 
So, let $f\colon C\to D$ be a homotopy equivalence with 
weak inverse $g\colon D\to C$ and a pair of homotopies 
$\phi\colon\id\then g\circ f$ and 
$\psi\colon f\circ g\then\id$.
Then let's prove that $\phi_{\ast}$ and $\psi_{\ast}$ 
are cochain homotopies. 

Indeed, for any object $X$ and any general morphism 
$h\colon X\to C$ of degree $n$, we have
(note that the composition rule in $\Cc$ is a cochain map 
and that $\phi$ is of degree $-1$)
\begin{align*}
&\quad
(\di_{\HHom_{\Cc}(X,D)}^{n-1}\circ\phi_{\ast}^n+
\phi_{\ast}^{n+1}\circ\di_{\HHom_{\Cc}(X,C)}^n)
(h)\\
&= 
\di_{\HHom_{\Cc}(X,D)}^{n-1}(\phi\circ h) + 
\phi\circ\di_{\HHom_{\Cc}(X,C)}^n(h) \\
&= 
\di_{\HHom_{\Cc}(C,D)}^{-1}(\phi)\circ h - 
\phi\circ\di_{\HHom_{\Cc}(X,C)}^n(h) + 
\phi\circ\di_{\HHom_{\Cc}(X,C)}^n(h) \\
&=
\di_{\HHom_{\Cc}(C,D)}^{-1}(\phi)\circ h\\
&=
(g\circ f-\id)\circ h\\
&=
(g_{\ast}\circ f_{\ast} - \id_{\HHom_{\Cc}(X,X)})(h).
\end{align*}
Hence $\phi_{\ast}$ is a cochain homotopy 
from $\id$ to $g_{\ast}\circ f_{\ast}$. 
The proof for $\psi_{\ast}$ being cochain homotopy 
is similar. 
\end{proof}
\end{document}