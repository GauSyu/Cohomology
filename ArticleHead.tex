% !Mode:: "TeX:UTF-8"
%%%%%%%%%%%%%%%%%%%%%%%%%%%%%%%%%%%%%%%%%%%%%%
%              A Common Setting File Made by Gau, Syu               %
%                                version = 2018.April                     %
%                        contact = GauSyu@Gmail.com                        %
%%%%%%%%%%%%%%%%%%%%%%%%%%%%%%%%%%%%%%%%%%%%%%
%
% @ PRELOAD
%\RequirePackage{luatex85}	
%%	"luatex85" is needed for xy-envs if the typesetting engine is lualatex
%
% @ Document Class Settings
\documentclass[UTF-8,11pt,a4paper]{article}
%%		UTF-8: use UTF-8 encode
%%		11pt: main font size, 10pt is dedault
%%		a4paper: paper size
%
% @ Layout Settings
\usepackage{marginnote}
\usepackage{fancyhdr}		%	header and footer
\usepackage{indentfirst}	%	indent at each first line
\usepackage{pdfpages}		%	input pages from given pdf files.
% !Mode:: "TeX:UTF-8"
%%%%%%%%%%%%%%%%%%%%%%%%%%%%%%%%%%%%%%%%%%%%%%
%                Decorate Setting File Made by Gau, Syu                 %
%                                version = 2017.314                                 %
%                        contact = GauSyu@Gmail.com                        %
%%%%%%%%%%%%%%%%%%%%%%%%%%%%%%%%%%%%%%%%%%%%%%
%
% @ Colors
\usepackage{xcolor}		%	A packagefor colors
\newcommand{\black}{\color{black}}
\newcommand{\gray}{\color{gray}}
\newcommand{\red}{\color{red}}
\newcommand{\blue}{\color{blue}}
\newcommand{\green}{\color{green}}
\newcommand{\bfred}[1]{\textbf{\red #1}}
\newcommand{\bfblue}[1]{\textbf{\blue #1}}
\newcommand{\bired}[1]{\textbf{\it \red #1}}
\newcommand{\biblue}[1]{\textbf{\it \blue #1}}
\newcommand{\itred}[1]{\textit{\red #1}}
\newcommand{\itblue}[1]{\textit{\blue #1}}
%
% @ Typefaces
\usepackage{fontspec}	%	Provides an automatic and unified interface for loading fonts in LaTeX
%	Note that one may need the corresponding fonts in system to enable settings like the followings:
\newfontfamily{\Edward}{Edwardian Script ITC}
%\newfontfamily{\Frak}{Euclid Fraktur}
\newcommand{\Giant}{\fontsize{72pt}{\baselineskip}\selectfont}
\newcommand{\giant}{\fontsize{27pt}{\baselineskip}\selectfont}
%
% @ MultiColumn
\usepackage{multicol}	%	defines a multicols environment which typesets text in multiple columns
%
% @ Enumerate environment
\usepackage{enumerate}
%	This package gives the enumerate environment an optional argument which determines the style in which the counter is printed.
\usepackage[shortlabels]{enumitem}
\newlist{steps}{enumerate}{3}
\setlist[steps,1]{label=\emph{Step \arabic*.}}
\setlist[steps]{align=left, leftmargin=0pt, listparindent=\parindent, labelwidth=0pt, itemindent=!}
\setlist[steps,2]{label=\emph{Step \arabic{stepsi}.\arabic*.}}
\setlist[steps,3]{label=\emph{\alph*.}}	%	colors, typefaces etc.
% @ Latex Graphics
% !Mode:: "TeX:UTF-8"
%%%%%%%%%%%%%%%%%%%%%%%%%%%%%%%%%%%%%%%%%%%%%%
%           Latex Graphics Setting File Made by Gau, Syu            %
%                                version = 2017.314                                 %
%                        contact = GauSyu@Gmail.com                        %
%%%%%%%%%%%%%%%%%%%%%%%%%%%%%%%%%%%%%%%%%%%%%%
\usepackage{graphicx}	%	Enable to Insert Pictures
\usepackage{epic}			%	Extending Latex Graphics
%\usepackage{xy}   			% Enable to Draw Xy-Diagrams
%%	Start from 2017, I decive to use tikzcd instead of xy-enivs
\usepackage{tikz}			%	Enable to Draw Pictures 
\usetikzlibrary{cd}				%	Load the tikzcd library
\usetikzlibrary{arrows}		%	Load the arrows
\usetikzlibrary{tikzmark,quotes}	%	Enable subnode
\usepackage{braids}			%	An easy way to draw braids, need tikz
\usepackage{adjustbox}
%
% @ Indexes, Glossaries and Refs
% !Mode:: "TeX:UTF-8"
%%%%%%%%%%%%%%%%%%%%%%%%%%%%%%%%%%%%%%%%%%%%%%
%                   Index Setting File Made by Gau, Syu                   %
%                                version = 2017.314                                 %
%                        contact = GauSyu@Gmail.com                        %
%%%%%%%%%%%%%%%%%%%%%%%%%%%%%%%%%%%%%%%%%%%%%%
%
% @ MakeIndex
\makeindex
%
% @ Indexed Terminology
\usepackage{xifthen}			%	provides \isempty test
\newcommand{\termin}[2][]{%
\ifthenelse{\isempty{#1}}%
{\emph{\textbf{{#2}}}\index{#2}}% if #1 is empty
{\emph{\textbf{{#1}}}\index{#2}}% if #1 is not empty
}	%	Used as: \termin[arg1]{arg2}
%%		arg1 is optional, it is what this command display, default is same as arg2
%%		arg2 is what this command generates in the Index table
\newcommand{\bfredin}[2][]{%
\ifthenelse{\isempty{#1}}%
{\bfred{{#2}}\index{#2}}% if #1 is empty
{\bfred{{#1}}\index{#2}}% if #1 is not empty
}	%	Same as above, except the layout style
\newcommand{\bfbluein}[2][]{%
\ifthenelse{\isempty{#1}}%
{\bfblue{{#2}}\index{#2}}% if #1 is empty
{\bfblue{{#1}}\index{#2}}% if #1 is not empty
}	%	Same as above, except the layout style	%	index settings
%% !Mode:: "TeX:UTF-8"
%%%%%%%%%%%%%%%%%%%%%%%%%%%%%%%%%%%%%%%%%%%%%%
%               Glossaries Setting File Made by Gau, Syu               %
%                                version = 2017.314                                 %
%                        contact = GauSyu@Gmail.com                        %
%%%%%%%%%%%%%%%%%%%%%%%%%%%%%%%%%%%%%%%%%%%%%%
%
% @ MakeGlossaries
    \usepackage[
               symbols,                %list of symbols
              %nonumberlist,           %do not show page numbers
               seeautonumberlist,      %
               hyperfirst=false,       %
               toc,                    %show listings as entries in table of contents
               section=chapter,        %use section level for toc entries
               counter=section         %countered by section
               ]{glossaries}
    % & Add new glossary listing
        \altnewglossary{categories}{cat}{Categories}
    % & Enable hyperref on gls
        \glsenablehyper
    % & Make glossaries
        \makeglossaries

    % & <Example> Usage of glossaryentry
    %   \newglossaryentry{<\label>}
    %                    {name=<\what occurs in the glossary>,
    %                     description=<\>,
    %                     text=<\what occurs in the context>,
    %                     sort=<\How this term by sorted>,
    %                     type=<\>}

	%	glossary settings
% !Mode:: "TeX:UTF-8"
%%%%%%%%%%%%%%%%%%%%%%%%%%%%%%%%%%%%%%%%%%%%%%
%                 BibTeX Setting File Made by Gau, Syu                %
%                                version = 2017.314                                 %
%                        contact = GauSyu@Gmail.com                        %
%%%%%%%%%%%%%%%%%%%%%%%%%%%%%%%%%%%%%%%%%%%%%%
%
\usepackage[sorting=none,giveninits=true]{biblatex}
%% !Mode:: "TeX:UTF-8"
%%%%%%%%%%%%%%%%%%%%%%%%%%%%%%%%%%%%%%%%%%%%%%
%                 AMSRefs Setting File Made by Gau, Syu                %
%                                version = 2017.314                                 %
%                        contact = GauSyu@Gmail.com                        %
%%%%%%%%%%%%%%%%%%%%%%%%%%%%%%%%%%%%%%%%%%%%%%
%
% @ AMSRefs
\usepackage[alphabetic,abbrev]{amsrefs}	%	A Refs Package for AMS
%%	citation-order: ordered as the citation-order
%%	alphabetic: generates alphabetic labels
%%	y2k: use the full year instead of the last two digits of the year
%%	shortalphabetic: generates a shorter alphabetic label using only the first letter of each author name
%%	abbrev: abbrev: equivalent to requesting all four of the following options
%%		initials: replace the given names of all authors, editors, and translators with their initials
%%		short-journals: print short form instead of full form for journal names
%%		short-months: print short version of month names (e.g., Jan. instead of January)
%%		short-publishers: print short form instead of full form for publisher names
%%	backrefs: causes “back-references” to be printed at the end of each bibliography entry to show what page it was cited on
\newcommand{\CITE}[1]{%
\hyperlink{#1}{{\upshape [\textbf{\red{#1}}]}}
}
	% AMSRef settings, NEED manual modifications
%
% @ Chapters and Sections
% !Mode:: "TeX:UTF-8"
%%%%%%%%%%%%%%%%%%%%%%%%%%%%%%%%%%%%%%%%%%%%%%
%                 Section Setting File Made by Gau, Syu                  %
%                                version = 2017.314                                 %
%                        contact = GauSyu@Gmail.com                        %
%%%%%%%%%%%%%%%%%%%%%%%%%%%%%%%%%%%%%%%%%%%%%%
%
% @ Section name
\usepackage{titlesec}
\titleformat{\section}
{\normalfont\Large\bfseries}
{\S\ \thesection}
{1em}
{}
%	\titleformat{<command>}[<shape>]{<format>}{<label>}{<sep>}{<before-code>}[<after-code>]
%%	<command>: the sectioning command to be redefined
%%	<shape>: (optional) paragraph shape
%%	<format>: the format to be applied to the whole title
%%	<label>: the label with numbering
%%	<sep>: the horizontal separation between label and title body
%%	<before-code>: code preceding the title body
%%	<after-code>: (optional) code following the title body
%
% @ Section Numbering
%\renewcommand{\thesubsection}{\thesection.\roman{subsection}}	%	abbrev and romanize the counter labels
%	The following code remove the whole label part of subsection if it doesn't have number, otherwise there will be an empty label at head
\makeatletter
\def\@subseccntformat#1{\csname #1ignore\expandafter\endcsname\csname the#1\endcsname\quad}
\let\subsectionignore\@gobbletwo
\let\latex@numberline\numberline
\def\numberline#1{\if\relax#1\relax\else\latex@numberline{#1}\fi}
\makeatother
%
% @ Set the depths
\setcounter{secnumdepth}{2}			%	depth of section numbering levels
%	0=chapter, 1=section, 2=subsection
\setcounter{tocdepth}{2}				% 	depth of contents	%	section settings for articles
%
% @ Math Support
%\usepackage{stix}					%	The STIX fonts are a suite of uni­code OpenType fonts con­tain­ing a com­plete set of math­e­mat­i­cal glyphs
\usepackage[leqno]{amsmath}
%	Standard AMS Package & AMS Theorem Env
%%	leqno means put number of formulas on the left
%
\usepackage{mathtools}
%	provides a series of packages designed to enhance the appearance of documents containing a lot of mathematics
%
\usepackage{hyperref}	%	basic package for hyperrefs
% @@@@@@@@@@@@@ MUST Put hyperref AFTER amsmath and BEFORE cleveref
\usepackage[amsmath,thmmarks,hyperref]{ntheorem}
%	Pacakage for for handling theorem-like environments
%%	[thmmarks] enables the automatical placement of endmarks
%%	[thref] enables the extended reference features
%%	[hyperref] provides compability with the hyperref-package
%%	[amsthm] provides compatibility with the theorem-layout commands of the amsthm-package
%%	[standard] uses the amssymb and latexsym (automatically loaded) packages and defines the frequently used environments in english and german documents
%
\usepackage{thmtools}
%	a collection of packages that is designed to provide an easier interface to theorems, and to facilitate some more advanced tasks
%
%
%
% @ fix the bug in aliasctr.sty
\usepackage{etoolbox}
\makeatletter
\patchcmd{\@counteralias}
 {\@ifdefinable{c@#1}}
 {\expandafter\@ifdefinable\csname c@#1\endcsname}
 {}{}
\makeatother
%
%
%
% @ Math Fonts
\usepackage{amssymb}		%	defined math symbols in msam and mabm, amsfonts is automatically included
%\usepackage{eucal}				%	makes \mathcal use Euler script instead of the usual Computer Modern calligraphic alphabet
%\usepackage{mathrsfs}		%	support for using RSFS fonts (\mathscr) in maths
\usepackage[cal=euler,scr=boondox]{mathalfa}
%	make \mathcal use Euler script and \mathscr use boondoxo (from Stix with Calligraphic Oblique)
\usepackage{mathdots}		%	commands to produce dots in math that respect font size
\usepackage{upgreek}		%	upright Greek letters
\usepackage{manfnt}			%	Knuth's ‘manual’ font
\usepackage{stmaryrd}		%	St Mary Road symbols for theoretical computer science
%
% @ Abbrevated Command for Letters
% !Mode:: "TeX:UTF-8"
%%%%%%%%%%%%%%%%%%%%%%%%%%%%%%%%%%%%%%%%%%%%%%
%              Letter Abbreviation File Made by Gau, Syu              %
%                                version = 2017.314                                 %
%                        contact = GauSyu@Gmail.com                        %
%%%%%%%%%%%%%%%%%%%%%%%%%%%%%%%%%%%%%%%%%%%%%%
%
% @ \mathcal
\def\Aa{{\cal A}}
\def\Bb{{\cal B}}
\def\Cc{{\cal C}}
\def\Dd{{\cal D}}
\def\Ee{{\cal E}}
\def\Ff{{\cal F}}
\def\Gg{{\cal G}}
\def\Hh{{\cal H}}
\def\Ii{{\cal I}}
\def\Jj{{\cal J}}
\def\Kk{{\cal K}}
\def\Ll{{\cal L}}
\def\Mm{{\cal M}}
\def\Nn{{\cal N}}
\def\Oo{{\cal O}}
\def\Pp{{\cal P}}
\def\Qq{{\cal Q}}
\def\Rr{{\cal R}}
\def\Ss{{\cal S}}
\def\Tt{{\cal T}}
\def\Uu{{\cal U}}
\def\Vv{{\cal V}}
\def\Ww{{\cal W}}
\def\Xx{{\cal X}}
\def\Yy{{\cal Y}}
\def\Zz{{\cal Z}}			%	\mathcal{A} -> \Aa
% !Mode:: "TeX:UTF-8"
%%%%%%%%%%%%%%%%%%%%%%%%%%%%%%%%%%%%%%%%%%%%%%
%              Letter Abbreviation File Made by Gau, Syu              %
%                                version = 2017.314                                 %
%                        contact = GauSyu@Gmail.com                        %
%%%%%%%%%%%%%%%%%%%%%%%%%%%%%%%%%%%%%%%%%%%%%%
%
% @ \mathbb
\def\AA{{\mathbb A}}
\def\BB{{\mathbb B}}
\def\CC{{\mathbb C}}
\def\DD{{\mathbb D}}
\def\EE{{\mathbb E}}
\def\FF{{\mathbb F}}
\def\GG{{\mathbb G}}
\def\HH{{\mathbb H}}
\def\II{{\mathbb I}}
\def\JJ{{\mathbb J}}
\def\KK{{\mathbb K}}
\def\LL{{\mathbb L}}
\def\MM{{\mathbb M}}
\def\NN{{\mathbb N}}
\def\OO{{\mathbb O}}
\def\PP{{\mathbb P}}
\def\QQ{{\mathbb Q}}
\def\RR{{\mathbb R}}
\def\SS{{\mathbb S}}
\def\TT{{\mathbb T}}
\def\UU{{\mathbb U}}
\def\VV{{\mathbb V}}
\def\WW{{\mathbb W}}
\def\XX{{\mathbb X}}
\def\YY{{\mathbb Y}}
\def\ZZ{{\mathbb Z}}			%	\mathbb{A} -> \AA
% !Mode:: "TeX:UTF-8"
%%%%%%%%%%%%%%%%%%%%%%%%%%%%%%%%%%%%%%%%%%%%%%
%              Letter Abbreviation File Made by Gau, Syu              %
%                                version = 2017.314                                 %
%                        contact = GauSyu@Gmail.com                        %
%%%%%%%%%%%%%%%%%%%%%%%%%%%%%%%%%%%%%%%%%%%%%%
%
% @ \mathfrak (little)
\def\aa{\mathfrak{a}}
\def\bb{\mathfrak{b}}
\def\cc{\mathfrak{c}}
\def\dd{\mathfrak{d}}
\def\ee{\mathfrak{e}}
\def\ff{\mathfrak{f}}
\def\gg{\mathfrak{g}}  % Knuth uses $\gg$ for ``>>''.
\def\hh{\mathfrak{h}}
\def\ii{\mathfrak{i}}
\def\jj{\mathfrak{j}}
\def\kk{\mathfrak{k}}
\def\ll{\mathfrak{l}}  % Knuth uses $\ll$ for ``<<''.
\def\mm{\mathfrak{m}}
\def\nn{\mathfrak{n}}
\def\oo{\mathfrak{o}}
\def\pp{\mathfrak{p}}
\def\qq{\mathfrak{q}}
\def\rr{\mathfrak{r}}
\def\ss{\mathfrak{s}}
\def\tt{\mathfrak{t}}
\def\uu{\mathfrak{u}}
\def\vv{\mathfrak{v}}
\def\ww{\mathfrak{w}}
\def\xx{\mathfrak{x}}
\def\yy{\mathfrak{y}}
\def\zz{\mathfrak{z}}
			%	\mathfrak{a} -> \aa
% !Mode:: "TeX:UTF-8"
%%%%%%%%%%%%%%%%%%%%%%%%%%%%%%%%%%%%%%%%%%%%%%
%              Letter Abbreviation File Made by Gau, Syu              %
%                                version = 2017.314                                 %
%                        contact = GauSyu@Gmail.com                        %
%%%%%%%%%%%%%%%%%%%%%%%%%%%%%%%%%%%%%%%%%%%%%%
%
% @ \mathscr
\def\Aaa{\mathscr{A}}
\def\Bbb{\mathscr{B}}
\def\Ccc{\mathscr{C}}
\def\Ddd{\mathscr{D}}
\def\Eee{\mathscr{E}}
\def\Fff{\mathscr{F}}
\def\Ggg{\mathscr{G}}
\def\Hhh{\mathscr{H}}
\def\Iii{\mathscr{I}}
\def\Jjj{\mathscr{J}}
\def\Kkk{\mathscr{K}}
\def\Lll{\mathscr{L}}
\def\Mmm{\mathscr{M}}
\def\Nnn{\mathscr{N}}
\def\Ooo{\mathscr{O}}
\def\Ppp{\mathscr{P}}
\def\Qqq{\mathscr{Q}}
\def\Rrr{\mathscr{R}}
\def\Sss{\mathscr{S}}
\def\Ttt{\mathscr{T}}
\def\Uuu{\mathscr{U}}
\def\Vvv{\mathscr{V}}
\def\Www{\mathscr{W}}
\def\Xxx{\mathscr{X}}
\def\Yyy{\mathscr{Y}}
\def\Zzz{\mathscr{Z}}			%	\mathscr{A} -> \Aaa
%% !Mode:: "TeX:UTF-8"
%%%%%%%%%%%%%%%%%%%%%%%%%%%%%%%%%%%%%%%%%%%%%%
%              Letter Abbreviation File Made by Gau, Syu              %
%                                version = 2017.314                                 %
%                        contact = GauSyu@Gmail.com                        %
%%%%%%%%%%%%%%%%%%%%%%%%%%%%%%%%%%%%%%%%%%%%%%
%
% @ \mathscr
\def\scA{\mathscr{A}}
\def\scB{\mathscr{B}}
\def\scC{\mathscr{C}}
\def\scD{\mathscr{D}}
\def\scE{\mathscr{E}}
\def\scF{\mathscr{F}}
\def\scG{\mathscr{G}}
\def\scH{\mathscr{H}}
\def\scI{\mathscr{I}}
\def\scJ{\mathscr{J}}
\def\scK{\mathscr{K}}
\def\scL{\mathscr{L}}
\def\scM{\mathscr{M}}
\def\scN{\mathscr{N}}
\def\scO{\mathscr{O}}
\def\scP{\mathscr{P}}
\def\scQ{\mathscr{Q}}
\def\scR{\mathscr{R}}
\def\scS{\mathscr{S}}
\def\scT{\mathscr{T}}
\def\scU{\mathscr{U}}
\def\scV{\mathscr{V}}
\def\scW{\mathscr{W}}
\def\scX{\mathscr{X}}
\def\scY{\mathscr{Y}}
\def\scZ{\mathscr{Z}}		%	\mathscr{A} -> \scA
% !Mode:: "TeX:UTF-8"
%%%%%%%%%%%%%%%%%%%%%%%%%%%%%%%%%%%%%%%%%%%%%%
%              Letter Abbreviation File Made by Gau, Syu              %
%                                version = 2017.314                                 %
%                        contact = GauSyu@Gmail.com                        %
%%%%%%%%%%%%%%%%%%%%%%%%%%%%%%%%%%%%%%%%%%%%%%
%
% @ \mathfrak (capital)
\def\AAa{\mathfrak{A}}
\def\BBb{\mathfrak{B}}
\def\CCc{\mathfrak{C}}
\def\DDd{\mathfrak{D}}
\def\EEe{\mathfrak{E}}
\def\FFf{\mathfrak{F}}
\def\GGg{\mathfrak{G}}
\def\HHh{\mathfrak{H}}
\def\IIi{\mathfrak{I}}
\def\JJj{\mathfrak{J}}
\def\KKk{\mathfrak{K}}
\def\LLl{\mathfrak{L}}
\def\MMm{\mathfrak{M}}
\def\NNn{\mathfrak{N}}
\def\OOo{\mathfrak{O}}
\def\PPp{\mathfrak{P}}
\def\QQq{\mathfrak{Q}}
\def\RRr{\mathfrak{R}}
\def\SSs{\mathfrak{S}}
\def\TTt{\mathfrak{T}}
\def\UUu{\mathfrak{U}}
\def\VVv{\mathfrak{V}}
\def\WWw{\mathfrak{W}}
\def\XXx{\mathfrak{X}}
\def\YYy{\mathfrak{Y}}
\def\ZZz{\mathfrak{Z}}	%	\mathfrak{A} -> \AAa
% !Mode:: "TeX:UTF-8"
%%%%%%%%%%%%%%%%%%%%%%%%%%%%%%%%%%%%%%%%%%%%%%
%              Letter Abbreviation File Made by Gau, Syu              %
%                                version = 2017.314                                 %
%                        contact = GauSyu@Gmail.com                        %
%%%%%%%%%%%%%%%%%%%%%%%%%%%%%%%%%%%%%%%%%%%%%%
%
% @ \mathfrak (capital)
\def\frA{\mathfrak{A}}
\def\frB{\mathfrak{B}}
\def\frC{\mathfrak{C}}
\def\frD{\mathfrak{D}}
\def\frE{\mathfrak{E}}
\def\frF{\mathfrak{F}}
\def\frG{\mathfrak{G}}
\def\frH{\mathfrak{H}}
\def\frI{\mathfrak{I}}
\def\frJ{\mathfrak{J}}
\def\frK{\mathfrak{K}}
\def\frL{\mathfrak{L}}
\def\frM{\mathfrak{M}}
\def\frN{\mathfrak{N}}
\def\frO{\mathfrak{O}}
\def\frP{\mathfrak{P}}
\def\frQ{\mathfrak{Q}}
\def\frR{\mathfrak{R}}
\def\frS{\mathfrak{S}}
\def\frT{\mathfrak{T}}
\def\frU{\mathfrak{U}}
\def\frV{\mathfrak{V}}
\def\frW{\mathfrak{W}}
\def\frX{\mathfrak{X}}
\def\frY{\mathfrak{Y}}
\def\frZ{\mathfrak{Z}}	%	\mathfrak{A} -> \frA
% !Mode:: "TeX:UTF-8"
%%%%%%%%%%%%%%%%%%%%%%%%%%%%%%%%%%%%%%%%%%%%%%
%              Letter Abbreviation File Made by Gau, Syu              %
%                                version = 2017.314                                 %
%                        contact = GauSyu@Gmail.com                        %
%%%%%%%%%%%%%%%%%%%%%%%%%%%%%%%%%%%%%%%%%%%%%%
%
% @ \mathrm (little)
\def\aaa{\mathrm{a}}
\def\bbb{\mathrm{b}}
\def\ccc{\mathrm{c}}
\def\ddd{\mathrm{d}}
\def\eee{\mathrm{e}}
\def\fff{\mathrm{f}}
\def\ggg{\mathrm{g}}  % Knuth uses $\gg$ for ``>>''.
\def\hhh{\mathrm{h}}
\def\iii{\mathrm{i}}
\def\jjj{\mathrm{j}}
\def\kkk{\mathrm{k}}
\def\lll{\mathrm{l}}  % Knuth uses $\ll$ for ``<<''.
\def\mmm{\mathrm{m}}
\def\nnn{\mathrm{n}}
\def\ooo{\mathrm{o}}
\def\ppp{\mathrm{p}}
\def\qqq{\mathrm{q}}
\def\rrr{\mathrm{r}}
\def\sss{\mathrm{s}}
\def\ttt{\mathrm{t}}
\def\uuu{\mathrm{u}}
\def\vvv{\mathrm{v}}
\def\www{\mathrm{w}}
\def\xxx{\mathrm{x}}
\def\yyy{\mathrm{y}}
\def\zzz{\mathrm{z}}		%	\mathrm{a} -> \aaa
%
% @ Notations
% !Mode:: "TeX:UTF-8"
%%%%%%%%%%%%%%%%%%%%%%%%%%%%%%%%%%%%%%%%%%%%%%
%                      Notations File Made by Gau, Syu                       %
%                                version = 2017.314                                 %
%                        contact = GauSyu@Gmail.com                        %
%%%%%%%%%%%%%%%%%%%%%%%%%%%%%%%%%%%%%%%%%%%%%%
%
% @ Arrows
\def\acts{\curvearrowright}
\def\actson{\curvearrowright}
\def\actsby{\curvearrowleft}
\def\sendto{\rightsquigarrow}
%
\def\mono{\rightarrowtail}
\def\epi{\twoheadrightarrow}
\def\into{\hookrightarrow}
\def\onto{\twoheadrightarrow}
\def\isom{\overset{\sim}{\to}}
\def\Isom{\overset{\sim}{\longrightarrow}}
%
\def\from{\leftarrow}
\def\To{\longrightarrow}
\def\From{\longleftarrow}
\def\tofrom{\leftrightarrow}
\def\ToFrom{\longleftrightarrow}
\def\tto{\rightrightarrows}
\def\ffrom{\leftleftarrows}
%
\def\then{\Rightarrow}
\def\Then{\Longrightarrow}
\def\down{\downarrow}
\def\up{\uparrow}			%	Alternative commands for arrows
% !Mode:: "TeX:UTF-8"
%%%%%%%%%%%%%%%%%%%%%%%%%%%%%%%%%%%%%%%%%%%%%%
%                      Notations File Made by Gau, Syu                       %
%                                version = 2017.314                                 %
%                        contact = GauSyu@Gmail.com                        %
%%%%%%%%%%%%%%%%%%%%%%%%%%%%%%%%%%%%%%%%%%%%%%
%
% @ Functions
\newcommand{\norm}[1]{\lVert #1\rVert}    %	Norm
\newcommand{\dual}[1]{{#1}^{\wedge}}      %	Dual
\newcommand{\codual}[1]{{#1}^{\vee}}      %	Codual
\newcommand{\pfrac}[2]{\frac{\partial{#1}}{\partial{#2}}}		%	fraction of partial differentials
\newcommand{\bfrac}[2]{\left(\frac{#1}{#2}\right)}					%	Legendre cymbol
\newcommand{\udsum}[3]{\left.{#1}\right|^{#2}_{#3}}				%	1^2_3
\newcommand{\local}[2]{\left.{#1}\right|_{#2}}						%	Local #1 at #2
%
\newcommand{\markar}[1]{\stackrel{{#1}}{\longrightarrow}}	%	marked rightarrow
\newcommand{\markal}[1]{\stackrel{{#1}}{\longleftarrow}}	%	marked leftarrow
\newcommand{\iso}[1]{\stackrel{#1}{\cong}}								%	marked isomorphism
\newcommand{\defeq}{\stackrel{\text{def}}{=}}							%	definition equality
\newcommand{\defen}{\stackrel{\text{def}}{\iff}}						%	definition condition
%
\newcommand{\tdtlog}[1]{t\frac{\rm d}{{\rm d} t}\log\left({#1}\right)}
\newcommand{\kongwei}{\underline{\hphantom{X}}}
%
\newcommand{\tp}[1]{\prescript{\rm t}{}{#1}}
%
% @ Map Descriptions
\newcommand{\mapdes}[4]
{
	\begin{align*}
	#1 & \longrightarrow #2 \\
	#3 & \longmapsto #4
	\end{align*}
}
\newcommand{\longmapdes}[5]
{
	\begin{align*}
	#1\colon  #2 & \longrightarrow  #3 \\
	#4 & \longmapsto  #5
	\end{align*}
}
\newcommand{\isodes}[4]
{
	\begin{align*}
	#1 & \cong  #2 \\
	#3 & \leftrightarrow  #4
	\end{align*}
}		%	Functions and Map Descriptions
% !Mode:: "TeX:UTF-8"
%%%%%%%%%%%%%%%%%%%%%%%%%%%%%%%%%%%%%%%%%%%%%%
%                      Notations File Made by Gau, Syu                       %
%                                version = 2017.314                                 %
%                        contact = GauSyu@Gmail.com                        %
%%%%%%%%%%%%%%%%%%%%%%%%%%%%%%%%%%%%%%%%%%%%%%
%
%	\ensuremath{---}\xspace may be helpful
%
% @ common Operators
\DeclareMathOperator{\id}{id}			%	identity
\DeclareMathOperator{\Id}{Id}			%	Identity
\DeclareMathOperator{\obj}{obj}			%	identity
\DeclareMathOperator{\pr}{pr}			%	projection
\DeclareMathOperator{\pt}{pt}			%	point
\DeclareMathOperator{\res}{res}			%	restriction
%
% @ Operators in a category
\DeclareMathOperator{\coim}{coim}
\DeclareMathOperator{\Coim}{Coim}
\DeclareMathOperator{\coker}{coker}
\DeclareMathOperator{\Coker}{Coker}
\DeclareMathOperator{\im}{im}
\DeclareMathOperator{\Image}{Im}
\DeclareMathOperator{\Ker}{Ker}
\DeclareMathOperator{\Sat}{Sat}			%	Saturated closure
\DeclareMathOperator{\sk}{sk}
\DeclareMathOperator{\ob}{ob}%Object
%
% @ Operators in linear algebra
\DeclareMathOperator{\Ad}{Ad}			%	Adjunction
\DeclareMathOperator{\ad}{ad}			%	adjuntion
\DeclareMathOperator{\diag}{diag}
\DeclareMathOperator{\Det}{Det}			%	Determinant
\DeclareMathOperator{\Exp}{Exp}			%	Expotential
\DeclareMathOperator{\Hess}{Hess}		%	Hessian
\DeclareMathOperator{\Span}{span} 		%	Annoyingly, \span is already a command in TeX, and redefining it leads to other problems
\DeclareMathOperator{\Tr}{Tr}				%	trace
\DeclareMathOperator{\tr}{tr}				%	trace
\DeclareMathOperator{\wt}{wt}				%	trace
%
% @ Operators in sub(p)scripts
\DeclareMathOperator{\ab}{ab}		%	abelian
\DeclareMathOperator{\an}{an}		%	analytic
\DeclareMathOperator{\conv}{conv}%	convergent
\DeclareMathOperator{\ev}{ev}		%	evaluation
\DeclareMathOperator{\opp}{op}	%	opposite
\DeclareMathOperator{\pre}{pre}
\DeclareMathOperator{\rev}{rev}		%	reverse
\DeclareMathOperator{\sep}{sep}		%	separated
\DeclareMathOperator{\tor}{tor}		%	torsion
%
% @ Operators in commutative algebra
\DeclareMathOperator{\Ann}{Ann}			%	Annihilator
\DeclareMathOperator{\Char}{Char}		%	Characteristic
\DeclareMathOperator{\gr}{gr}			%	associated graded object
\DeclareMathOperator{\Gr}{Gr}			%	associated graded object
\DeclareMathOperator{\SGr}{\mathscr{G}\!\mathit{r}}%	associated graded object
\DeclareMathOperator{\lcm}{lcm}			%	least common multiple
\DeclareMathOperator{\ord}{ord}			%	order
\DeclareMathOperator{\rad}{rad}			%	radical
\DeclareMathOperator{\Rad}{Rad}			%	Radical
\DeclareMathOperator{\rank}{rank}		%	rank
\DeclareMathOperator{\mult}{mult}		%	multiplicity
\DeclareMathOperator{\Length}{Length}		%	length
\DeclareMathOperator{\sgn}{sgn}			%	signature
\DeclareMathOperator{\supp}{supp}	%	support
\DeclareMathOperator{\Supp}{Supp}	%	Support
\DeclareMathOperator{\vol}{vol}				%	volume
\DeclareMathOperator{\Vol}{Vol}				%	Volume
%
% @ Operators of constructions
\DeclareMathOperator{\Alt}{Alt}				%	Alternative
\DeclareMathOperator{\Sym}{Sym}			%	Symmetric
\DeclareMathOperator{\Ext}{Ext}				%	Extensions
\DeclareMathOperator*{\EExt}{\mathcal{E}xt}	%	Cal Ext
\DeclareMathOperator{\Proj}{Proj}			%	Proj Spectrum
\DeclareMathOperator{\Quot}{Qout}		%	Quotient
\DeclareMathOperator{\Spec}{Spec}		%	Spectrum
\DeclareMathOperator{\Spm}{Spm}		%	Maximal Spectrum
\DeclareMathOperator{\Tor}{Tor}				%	Torsors
%
\DeclareMathOperator{\Der}{Der}			%	Derivations
\DeclareMathOperator{\SDer}{\mathscr{D}\!\mathit{er}}			%	Sheaf Derivations
\DeclareMathOperator{\Diff}{Diff}			%	Differential operators
\DeclareMathOperator{\SDiff}{\mathscr{D}\!\mathit{iff}}			%	Differential operators
\DeclareMathOperator{\Gal}{Gal}			%	Galois group
\DeclareMathOperator{\Pic}{Pic}			%	Picard group
\DeclareMathOperator{\Res}{Res}			%	Residue
%
% @ Hom-like Operators
\DeclareMathOperator{\Aut}{Aut}		%	automorphisms
\DeclareMathOperator{\Bil}{Bil}				%	Bilinear maps
\DeclareMathOperator{\End}{End}		%	Endomorphisms
\DeclareMathOperator{\Fun}{Fun}			%	Functors
\DeclareMathOperator{\hFun}{hFun}			%	Functors
\DeclareMathOperator{\FFun}{\mathcal{F}un}			%	Functors
\DeclareMathOperator{\GrHom}{GrHom}	%	Graded Hom
\DeclareMathOperator{\Hom}{Hom}			%	Hom
\DeclareMathOperator{\inEnd}{\underline{End}}			%	Internal End
\DeclareMathOperator{\inHom}{\underline{Hom}}			%	Internal Hom
\DeclareMathOperator{\inGrHom}{\underline{GrHom}}	%	Internal Graded Hom
\DeclareMathOperator{\Nat}{Nat}				%	Nature transformations
\DeclareMathOperator{\Mor}{Mor}		%	Morphisms
\DeclareMathOperator{\Map}{Map}		%	Maps
\DeclareMathOperator{\HHom}{\mathcal{H}\!\mathit{om}}			%		Hom space
\DeclareMathOperator{\SHom}{\mathscr{H}\!\!\mathit{om}}			%	sheaf Hom
\DeclareMathOperator{\SEnd}{\mathscr{E}\!\mathit{nd}}			%	sheaf Hom
\def\SheafHom{\SHom}
\DeclareMathOperator{\cont}{\mathit{cont}}
%
% @ Limit Operators
\DeclareMathOperator*{\holim}{\mathrm{holim}}
\DeclareMathOperator*{\hocolim}{\mathrm{hocolim}}		
\DeclareMathOperator*{\dirlim}{\underrightarrow{\lim}}			%	direct limit
\DeclareMathOperator*{\colim}{\mathrm{colim}}			%	direct limit
\DeclareMathOperator*{\invlim}{\underleftarrow{\lim}}			%	inverse limit
\DeclareMathOperator*{\Rinvlim}{\underleftarrow{\lim}^1}	%	derived inverse limit
\DeclareMathOperator*{\Lan}{Lan}				%	Left Kan extension
\DeclareMathOperator*{\Ran}{Ran}				%	Right Kan extension
%
%
% @ Operators in number theory
\DeclareMathOperator{\Frob}{Frob}		%	Frobenuis
\DeclareMathOperator*{\sump}{\sum\nolimits^\prime}			%	sum outside origin
\renewcommand{\mod}{\mathop{\mathrm{mod}}}
\makeatletter
\renewcommand{\pod}[1]{\allowbreak\mathchoice
	{\if@display \mkern 18mu\else \mkern 8mu\fi (#1)}
	{\if@display \mkern 18mu\else \mkern 8mu\fi (#1)}
	{\mkern4mu(#1)}
	{\mkern4mu(#1)}
}
\makeatother		%	Math Operators
% !Mode:: "TeX:UTF-8"
%%%%%%%%%%%%%%%%%%%%%%%%%%%%%%%%%%%%%%%%%%%%%%
%                      Notations File Made by Gau, Syu                       %
%                                version = 2017.314                                 %
%                        contact = GauSyu@Gmail.com                        %
%%%%%%%%%%%%%%%%%%%%%%%%%%%%%%%%%%%%%%%%%%%%%%
%
% @ Some notations
\newcommand{\Calrm}[2]{\mathcal{#1}\mathrm{#2}}
\newcommand{\<}{\langle}
\renewcommand{\>}{\rangle}
\renewcommand{\le}{\leqslant}
\renewcommand{\ge}{\geqslant}
\def\di{\mathrm{d}}
\def\dt{\frac{\di}{\di t}}
\def\st{\textrm{ s.t. }}
%
% @ Spaces
\def\RP{\mathbb{R}\mathbf{P}}
\def\CP{\mathbb{C}\mathbf{P}}
\def\Real{\mathbb{R}}
\def\dR{\mathrm{dR}}
\def\DB{\mathscr{D}\mathscr{b}}
%
% @ Groups
\def\GL{\mathrm{GL}}	%	general linear group
\def\SL{\mathrm{SL}}	%	special linear group
\def\GO{\mathrm{O}}		%	general orthogonal group
\def\SO{\mathrm{SO}}	%	special orthogonal group
\def\GU{\mathrm{U}}		%	general unitary group
\def\SU{\mathrm{SU}}	%	special unitary group
\def\Sp{\mathrm{Sp}}	%	symplectic group
\def\Spin{\mathrm{Spin}}%	Spin group
\def\PGL{\mathrm{PGL}}	%	proj GL
\def\PSL{\mathrm{PSL}}	%	proj SL
%
% @ Lie algebras
\def\gl{\mathfrak{gl}}		%	general linear Lie algebra
\def\sl{\mathfrak{sl}}		%	special linear Lie algebra
\def\go{\mathfrak{o}}	%	orthogonal
\def\so{\mathfrak{so}}	%	special orthogonal
\def\sp{\mathfrak{sp}}	%	symplectic
\def\gu{\mathfrak{u}}	%	unitary
\def\su{\mathfrak{su}}	%	special unitary
%
% @ Number theory staffs
\def\dedekind{{\scriptstyle\mathcal{O}}}
\def\Dedekind{\mathcal{O}}
%
% @ Categories
\def\Set{\mathbf{Set}}
\def\Sub{\mathbf{Sub}}
\def\Cat{\mathbf{Cat}}
\def\Top{\mathbf{Top}}
\def\Et{\mathbf{Et}}
\def\EtBund{\mathbf{EtBund}}
\def\PSh{\mathbf{PSh}}
\def\Sh{\mathbf{Sh}}
\def\Sch{\mathbf{Sch}} %Schemes
\def\Aff{\mathbf{Aff}}
\def\El{\mathbf{El}}
\def\Fin{\mathbf{Fin}}
\def\Alg{\mathbf{Alg}}
\def\Vect{\mathbf{Vect}}
\def\CRing{\mathbf{CRing}}
\def\Ring{\mathbf{Ring}}%Ring
\def\Ab{\mathbf{Ab}}%Abelian group 
\def\PAb{\mathbf{PAb}}
\def\Grp{\mathbf{Grp}}%Group
\def\Grpd{\mathbf{Grpd}}%Groupoid
\def\Mon{\mathbf{Mon}}%Monoid
\def\CMon{\mathbf{CMon}}%Monoid
\def\CoMon{\mathbf{CoMon}}%Monoid
\def\BiMon{\mathbf{BiMon}}%Monoid
\def\Mod{\mathbf{Mod}}%Module
\def\PMod{\mathbf{PMod}}%Module
\def\delooping{\mathbf{B}}
\def\Rep{\mathbf{Rep}}%Representations
\def\Loc{\mathbf{Loc}}
\def\SpSeq{\mathbf{SpSeq}}
\def\QCoh{\mathbf{QCoh}}
\def\Coh{\mathbf{Coh}}
\def\SmoothMfd{\mathbf{SmoothMfd}}
\def\Fil{\mathrm{Fil}}
\def\fib{\mathrm{fib}}	%Fibrations
\def\cofib{\mathrm{cofib}}	%Cofibrations
\def\Fib{\mathrm{Fib}}	%Fibrations
\def\Cofib{\mathrm{Cofib}}	%Cofibrations
\def\Cyl{\mathrm{Cly}}	%Cylinder
\def\Path{\mathrm{Path}}	%Path
\def\unit{\mathbf{1}}
\def\MC{\mathbf{MC}}
\def\Ch{\mathbf{Ch}}

\def\Ho{\mathrm{Ho}}%Homotopy
\def\h{\mathrm{h}}%Homotopy
\def\s{\mathrm{s}}%Homotopy
\def\dg{\mathrm{dg}}

% @ Miscellana
\def\vac{|0\>}
\newcommand{\nlab}[1]{\href{http://ncatlab.org/nlab/show/#1}{{#1}} }
\newcommand{\arxiv}[1]{\href{https://arxiv.org/abs/#1}{{#1}} }
\def\qbface{({\red·}‿{\red·})}			%	Math Names
%\input{Letters_and_Notations}
