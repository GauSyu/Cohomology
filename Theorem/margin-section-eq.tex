% !Mode:: "TeX:UTF-8"
%%%%%%%%%%%%%%%%%%%%%%%%%%%%%%%%%%%%%%%%%%%%%%
%               Theorems Setting File Made by Gau, Syu                %
%                                version = 2018April                       %
%                        contact = GauSyu@Gmail.com                        %
%%%%%%%%%%%%%%%%%%%%%%%%%%%%%%%%%%%%%%%%%%%%%%
%
% @	Known keys to \declaretheorem
%%	[parent=<counter>] (alias: numberwithin, within) : reset whenever that counter is incremented
%%	[sibling=<counter>] (alias: numberlike, sharenumber) : use this counter for numbering
%%	[title=<TEX>] (alias: name, heading) : title of the theorem, default is the name of the environment
%%	[numbered=<yes/no>] : will be numbered or not
%%	[style=<theoremstyle>] : use the settings of this style
%%	[refname=<string>] : the name of the theorem as used by \autoref, \cref and friends
%%	[refname=<string1,string2>] : the second is the plural form used by \cref
%%	[Refname=<string>] : the name of the theorem as used by \Autoref, \Cref and friends
%%	[Refname=<string1,string2>] : the second is the plural form used by \Cref
%
% @ Known keys to \declaretheoremstyle
%%	[spaceabove=<length>]
%%	[spacebelow=<length>]
%%	[headfont=<font switches>]
%%	[bodyfont=<font switches>]
%%	[headpunct=<TEX>] : theoremseparator
%%	[postheadspace=<length>] : horizontal space inserted after the entire head of the theorem, before the body
%%	[headindent=<length>] : horizontal space inserted before the head
%%	[headformat=<TEX> or <keywords>] : see manuel
%%	[headformat=margin] : number protude in the (left) margin
%%	[headformat=swapnumber] : puts number before the name
%
% @ Thm-type
\theoremstyle{margin}	%	Put the number at left margin
\numberwithin{equation}{section}	%	reset eq. number for each section
%
\declaretheorem[title=Theorem,numberlike=equation,%
refname={theorem,theorems},Refname={Theorem,Theorems}%
]{theorem}
\def\thm{\theorem}
\declaretheorem[title=Theorem,numbered=no]{Theorem}
\def\Thm{\Theorem}
%
\declaretheorem[title=Lemma,numberlike=equation,%
refname={lemma,lemmas},Refname={Lemma,Lemmas}%
]{lemma}
\def\lem{\lemma}
\declaretheorem[title=Lemma,within=equation,%
refname={lemma,lemmas},Refname={Lemma,Lemmas}%
]{sublemma}
\def\slem{\sublemma}
\declaretheorem[title=Lemma,numbered=no]{Lemma}
\def\Lem{\Lemma}
%
\declaretheorem[title=Proposition,numberlike=equation,%
refname={proposition,propositions},Refname={Proposition,Propositions}%
]{proposition}
\def\prop{\proposition}
\declaretheorem[title=Proposition,numberlike=sublemma,%
refname={proposition,propositions},Refname={Proposition,Propositions}%
]{subproposition}
\def\sprop{\subproposition}
\declaretheorem[title=Proposition,numbered=no]{Proposition}
\def\Prop{\Proposition}
%
\declaretheorem[title=Corollary,numberlike=equation,%
refname={corollary,corollaries},Refname={Corollary,Corollaries}%
]{corollary}
\def\cor{\corollary}
\declaretheorem[title=Corollary,numberlike=sublemma,%
refname={corollary,corollaries},Refname={Corollary,Corollaries}%
]{subcorolloary}
\def\scor{\subcorollary}
\declaretheorem[title=Corollary,numbered=no]{Corollary}
\def\Cor{\Croposition}
%
% @ Defn-type
\theorembodyfont{\upshape}	%	following envs will use upshape bodyfont
%
\declaretheorem[title={},numberlike=equation]{para}
\declaretheorem[title={},numberlike=sublemma]{subpara}
%
\declaretheorem[title={Definition},numberlike=equation,%
refname={definition,definitions},Refname={Definition,Definitions}%
]{definition}
\def\defn{\definition}
\declaretheorem[title={Definition},numbered=no]{Definition}
\def\Defn{\Definition}
%
\declaretheorem[title=Example,numberlike=equation,%
refname={example,examples},Refname={Example,Examples}%
]{example}
\def\eg{\example}
\declaretheorem[title=Example,numberlike=sublemma,%
refname={example,examples},Refname={Example,Examples}%
]{subexample}
\def\seg{\subexample}
\declaretheorem[title=Example,numbered=no]{Example}
\def\Eg{\Example}
%
\declaretheorem[title=Remark,numberlike=theorem,%
refname={remark,remarks},Refname={Remark,Remarks}%
]{remark}
\def\rem{\remark}
\declaretheorem[title=Remark,numberlike=sublemma,%
refname={remark,remarks},Refname={Remark,Remarks}%
]{subremark}
\def\srem{\subremark}
\declaretheorem[title=Remark,numbered=no]{Remark}
\def\Rem{\Remark}
%
\declaretheorem[title=Recall,numbered=no]{recall}
%
% @ Proof Env
\def\qedsymbol{\ensuremath{\Box}}	%	desige the qed symbol
%
\declaretheoremstyle[%
headfont=\sc,%
bodyfont=\normalfont,%
headpunct={:},%
qed=\qedsymbol%
]{nonumberproof}
%
\declaretheorem[%
title=Proof,%
numbered=no,%
style=proof%
]{proof}
%	proof
